% Created 2022-01-20 Thu 12:56
% Intended LaTeX compiler: pdflatex
\documentclass[usenatbib,twocolumn]{mnras}
\usepackage[utf8]{inputenc}
\usepackage[T1]{fontenc}
\usepackage{graphicx}
\usepackage{longtable}
\usepackage{wrapfig}
\usepackage{rotating}
\usepackage[normalem]{ulem}
\usepackage{amsmath}
\usepackage{amssymb}
\usepackage{capt-of}
\usepackage{hyperref}
\date{}
\title{OmEff Notebook}
\hypersetup{
 pdfauthor={JT Laune},
 pdftitle={OmEff Notebook},
 pdfkeywords={},
 pdfsubject={},
 pdfcreator={Emacs 27.2 (Org mode 9.5.2)}, 
 pdflang={English}}
\begin{document}

\maketitle
\section{Results}
\label{sec:org2c38111}
\subsection{External Test Particle with $\omega_{\rm eff}$}
\label{sec:orgacdee2a}
\begin{figure} \centering
\includegraphics[width=0.45\textwidth]{{oldTpResults/total-gbehaviors-eq5.62e-02}.png}
\caption{Here we have plotted the numerical average eccentricity of
the test particle over the last 10\% of simulation time, while holding
the disk forces constant to set $e_{\rm eq} = 0.056$. Each line
corresponds to a different value of $e_p$, while holding disk and
initial conditions constant. The dashed parts of the line indicate
standard test particle behavior, which is dependent on the specific
values of $e_p$ and $e_{\rm eq}$. The new behavior in the presence of
$\omega_{\rm eff}$ is indicated by solid lines, where
$\Delta\varpi\to\pi$.}
\label{fig:omeffbehaviors}
\end{figure}

\begin{figure} \centering
\includegraphics[width=0.45\textwidth]{{oldTpResults/regularbehavior}.png}
      \caption{Here we have plotted a typical outcome of test particle
capture with weak external precession. This plot corresponds to the
dashed segment of the pink line in Figure \ref{fig:omeffbehaviors}. As
we can see, because $e_p>e_{\rm eq}$, apsides become aligned.}
      \end{figure}

\begin{figure} \centering
    \includegraphics[width=0.45\textwidth]{{oldTpResults/newbehavior}.png}
    \caption{Here we have plotted a new phenomena of test particle capture
      with weak external precession. This plot corresponds to the solid segment
    of the pink line in Figure \ref{fig:omeffbehaviors}. As we can see,
  the apsides become anti-aligned. This is not previously a feature of
TP capture with $\omega_{\rm eff}=0$.}
    \end{figure}
\end{document}