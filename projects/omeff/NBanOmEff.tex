% Created 2022-02-16 Wed 23:12
% Intended LaTeX compiler: pdflatex
\documentclass[11pt]{article}
\usepackage[utf8]{inputenc}
\usepackage[T1]{fontenc}
\usepackage{graphicx}
\usepackage{longtable}
\usepackage{wrapfig}
\usepackage{rotating}
\usepackage[normalem]{ulem}
\usepackage{amsmath}
\usepackage{amssymb}
\usepackage{capt-of}
\usepackage{hyperref}
\usepackage{amsthm}
\newtheorem{defn}{Definition}
\author{JT Laune}
\date{\textit{<2022-02-08 Tue>}}
\title{Coplanar MMR Notes}
\hypersetup{
 pdfauthor={JT Laune},
 pdftitle={Coplanar MMR Notes},
 pdfkeywords={},
 pdfsubject={},
 pdfcreator={Emacs 27.2 (Org mode 9.5.2)}, 
 pdflang={English}}
\begin{document}

\maketitle

\section{Start}
\label{sec:org775029a}
Consider \(n\) point \textbf{orbiters} with finite masses
\(m_1,m_2,m_3,\ldots\ll M\), where \(M\) is the mass of the primary about
which they orbit. In the absence of dissipation, its Hamiltonian is:
\begin{align}
  \mathcal H
  &\simeq \sum^{n}_{i=1}\left(\frac12 \frac{\vec p_i\cdot\vec p_i}{m_i}
    -\frac{GMm_i}{2|\vec x_i|}
    -\sum_{j\neq i}^{n} \frac{Gm_im_j}{2|\vec x_i - \vec x_j|}
    \right), \\
\end{align}

where \(\vec x_i\) and \(\vec p_i\) are the Cartesian position and momentum vectors, respectively.

\section{Assumptions and notation}
\label{sec:orge773c0d}
\begin{itemize}
\item Two comparable mass orbiters, \(i=1,2\)
\begin{itemize}
\item Internal/external duality around q=1
\end{itemize}
\item Coplanar, \(i_i=0\)
\item Eccentricities are small, \(e_i\ll 1\)
\begin{itemize}
\item First order expansion in \(e_i\)
\end{itemize}
\item Orbiter masses are small, \(m_i\ll M\)
\item Near a first order MMR (1MMR), \(j = 1, 2, \ldots\)
\item independently parameterized \(\dot\varpi_{1,\rm ext}\) and \(\dot\varpi_{2,\rm ext}\)
\end{itemize}
\subsection{Notations}
\label{sec:org493b8b4}

\begin{center}
\begin{tabular}{ll}
\hline
Symbol & Definition\\
\hline
 & \\
\hline
\end{tabular}
\end{center}

\subsection{Definitions}
\label{sec:orga6b9b83}
\begin{defn}[Runge-Lenz vector]
the Cartesian vector with magnitude $e$ in the
direction of orbital perihelion
\end{defn}

\subsection{Dimensionless parameters}
\label{sec:orgca4e7d0}
We define:
\begin{align}
 q&\equiv \frac{m_1}{m_2}\quad\text{mass ratio with each other}\\ 
 \mu_i&\equiv \frac{m_i}{M}\quad\text{dimless mass}\\ 
 \mu_{\rm tot}&\equiv \mu_1+\mu_2\quad\text{total mass},\\ 
 %\mu_{\rm ext}&\equiv \frac{m_{\rm ext}}{M}\quad\text{total mass},\\ 
\end{align}

\noindent
where \(M\) is the primary mass.

\section{MMR Hamiltonian Model}
\label{sec:org22371e7}
Index MMR locations with \(j\), such that first order MMRs (hereafter ``1MMRS'') occur whenever
\(n_1/n_2\simeq(j+1)/j\).
Near these points, the following angles in the literal expansion of the perturbing
function (see \cite{murray_solar_2000} for details) have commensurate ``fast'' frequencies
\begin{align}
\theta_1 &= (j+1)\lambda_2-j\lambda_1-\varpi_1\\
\theta_2 &= (j+1)\lambda_2-j\lambda_1-\varpi_2,
\end{align}

\noindent
because the terms \((j+1)\lambda_2-j\lambda_1\approx 0\) near the MMR locations (by definition).

We choose dimensionless time units so that \(GM=1\) and
the Hamiltonian is given by
\begin{equation}
  \mathcal{H} = -\frac{m_1}{2a_1} -\frac{m_2}{2a_2} - \mu_2 m_1 f_1(\alpha) e_1 \cos\theta_1 + \mu_1 m_2 f_2(\alpha) e_2\cos\theta_2
\end{equation}

\noindent
where the mass ratios \(\mu_1\) and \(\mu_2\) now play the part as the perturbation parameter.

In the restricted planar 3-body problem, one planet is a test particle and has mass equal to zero.
For an internal (external) test particle,
the appropriate limit for the Hamiltonian above is \(m_1\to0\) (\(m_2\to0\)).
The Hamiltonian for an inner test particle is therefore
\begin{equation}
  \mathcal{H} = -\frac{1}{2a_1} - \mu_p( f_1(\alpha) e_1 \cos\theta_1 + f_2(\alpha) e_p\cos\theta_p)
\end{equation}

\noindent


\section{Test particle results for \(\dot\varpi_{\rm eff}>0\)}
\label{sec:org94f9ac7}
From the definitions of the angles \(\theta_1\) and \(\theta_2\),
any external precession frequency will "split" the location
of \(\theta_1\) and \(\theta_2\). The effects of this
shift are summarized in 

\begin{center}
\includegraphics[width=1\textwidth]{/home/jtlaune/multi-planet-architecture/projects/omeff/images/resShiftDiag.png}
\captionof{figure}{Figure caption}
\end{center}

\subsection{Internal}
\label{sec:orgd9fc98c}

\section{References}
\label{sec:org332b607}
\bibliography{references}
\bibliographystyle{unsrt}

\clearpage
\onecolumn
\appendix

\section{Poincair\'e's conjugate pair}
\label{sec:orga556740}
We utilize the following dimensionless coordinate-momentum conjugate
pairs (aka Poincair\'e coordinates):
\begin{align}
  \lambda_i \longleftrightarrow\Lambda_i &= \mu_i\sqrt{\alpha_i} \\
  -\varpi_i \longleftrightarrow\Gamma_i &= \mu_i\sqrt{\alpha_i}(1-\sqrt{1-e_i^2}) \approx \frac12\mu_i\sqrt{\alpha_i}e_i^2,
\end{align}

\noindent
where \(\varpi_i\) is the longitude of perihelion and \(\lambda_i\) the mean longitude
of orbiter \(m_i\).

\section{Geometric energy and AM}
\label{sec:orge26c6e9}
In the following, we characterize dissipation by its effects on each
planets' angular momentum (AM) and energy.  A planet's energy,
\(\mathcal E\), is determined by its semimajor axis (sma), \(a\):
\begin{align}
   \mathcal E = -\frac{1}{2a},
\end{align}
\noindent

\noindent
where we have chosen units such that \(GM=1\).
Angular momentum is given by
\begin{align}
h = \mathcal E \sqrt{1-e^2}.
\end{align}

\subsection{Dissipative effects}
\label{sec:orgdd6cbc2}
The dissipative effects are modeled
by two constant timescales for each planet, 
\begin{align}
  \frac{\dot a_i}{a_i} = -\frac{1}{2\pi\tau_{ai}} - \frac{pe_i^2}{2\pi\tau_{ei}} \\
  \frac{\dot e_i}{e_i} = -\frac{1}{2\pi\tau_{ei}} ,
\end{align}

where \(\tau_{ai}\) is the exponential e-damping of sma in years.  The
quantity \(\tau_{ei}\) is the same for eccentricity.

\section{Effects of quadrupole potential}
\label{sec:org4f45204}
A quadrupole potential may arise as a result of secular perturbations
from nearby planets on circular orbits or a \(J_2\) moment in the
primary's gravitational field. Due to the difference in sma
between any two orbiters, a quadrupole potential induces
differential apsidal precession on the orbiters.

\subsection{Derivation of differential precession rate \(\omega_{\rm eff}\)}
\label{sec:orgb88bf5a}
Consider a massive planet on a circular orbit which perturbs an MMR
which lies internal to its orbit.  Let the planet's mass and sma are
given by \(\mu_{\rm ext}\) and \(a_{\rm ext}\).  For each planet \(m_i\) in
the resonance, the interaction Hamiltonian with the external
planet is given by
\begin{equation}
  H_{i,\rm ext} = -\frac14 \Gamma_i \mu_{\rm ext}
  \left(\frac{a_i}{a_{\rm ext}}\right) b_{3/2}^{(1)}\left(\frac{a_i}{a_{\rm ext}}\right),
\end{equation}

\noindent
for \(j=1,2\) and we have utilized the approximation \(\Gamma_i \approx \frac12 \Lambda_i e_i^2\).

As a result, each planet experiences a precession in its mean longitude \(\lambda_i\) and
\(\gamma_i\equiv -\varpi_i\). In particular, the \(\dot\varpi_i\) precession frequency
is
\begin{equation}
\dot\varpi_{i, \rm ext} = \frac14 \mu_{\rm ext} 
    \left(\frac{a_i}{a_{\rm ext}}\right) b_{3/2}^{(1)}\left(\frac{a_i}{a_{\rm ext}}\right),
\end{equation}

\subsubsection{Murray and Dermott}
\label{sec:org026bc7a}
\begin{itemize}
\item secular perturbations \href{./images/screenshot-02.png}{7.8}
\item coefficients: \href{./images/screenshot-03.png}{7.9-7.12}
\item bar(alpha12) = [(alpha12 if j=1 external pert),  (1 if j=2 internal pert)]
\end{itemize}

\subsection{Constant \(\omega_{\rm eff}\)}
\label{sec:orgff130bb}
Set \(q=0.5\), inward migration. For simplicity, set \(\dot\varpi_{2,\rm
ext} = 0\) and \(\dot\varpi_{1,\rm ext}=\dot\varpi_{\rm eff}\) to be an
arbitrary precession frequency on \(m_1\).

\begin{center}
\includegraphics[width=0.6\textwidth]{./projects/omeff/varyOmeff/q0.50/h-0.03-Tw0-1000-mutot-1.0e-03/000-omeff0-0.000e+00-0.000e+00.png}
\captionof{figure}{Here is the unperturbed system, with \(\mu_{tot}=1e-3\), \$q=0.5, inward migration, and slow dissipative timescales (T\textsubscript{w,0}=10000 years) compared to those in Apsidal Alignment paper.}
\end{center}

\begin{center}
\includegraphics[width=0.6\textwidth]{./projects/omeff/varyOmeff/q0.50/h-0.03-Tw0-1000-mutot-1.0e-03/010-omeff0-3.981e-06-0.000e+00.png}
\captionof{figure}{Here is a perturbed system, with \(\dot\varpi_{1,\rm ext}=\dot\varpi_{\rm eff}\approx 4\times10^{-6}\) and  \(\dot\varpi_{2,\rm ext}=0\).}
\end{center}

\subsubsection{Results summary in final eccentricity and \(\Delta\varpi\)}
\label{sec:org37571b7}
\begin{center}
\includegraphics[width=0.6\textwidth]{./projects/omeff/varyOmeff/final-Dvarpi-states.png}
\captionof{figure}{Here we have plotted the final apsidal angle as a function of \(\dot\varpi_{1,\rm ext}\)}
\end{center}

\begin{center}
\includegraphics[width=0.6\textwidth]{./projects/omeff/varyOmeff/final-ecc-states.png}
\captionof{figure}{Here we have plotted the final eccentricities as a function of \(\dot\varpi_{1,\rm ext}\)}
\end{center}

\subsubsection{Code validation}
\label{sec:orgb52cad1}
\begin{center}
\includegraphics[width=0.6\textwidth]{./mpa/tests/omEff/test-omEff.png}
\captionof{figure}{In the above figure, we have set the total mass to be 1e-7 so that resonant and secular effects are negligible compared to the effects of external precession on \(\gamma_1\).}
\end{center}


\subsection{Sign of \(\omega_{\rm eff}\) and resonance splitting}
\label{sec:org0ea4cc3}

\section{Formal constructions}
\label{sec:org76b8484}
The \textbf{Kepler problem} is a special case of the \textbf{2-body problem}.
Its solutions are\ldots{}

We may characterize dissipation by its action on the \ldots{}
\(\tau_{mi}(t)\) and \(\tau_{ei}(t)\) by the instantaneous derivatives
\begin{align}
   \frac{\dot e_i}{e_i} &= - \frac{1}{\tau_e(t)} - \xi(t, \mathbf X_i)\frac{1}{\tau_m(t)} \\
   \frac{\dot a_i}{a_i} &= -\frac{1}{\tau_m(t)} - \zeta(t, \mathbf X_i)\frac{1}{\tau_e(t)},
\end{align}
\noindent

\noindent
where the dot notation corresponds to the time derivative of the
orbital elements. The functions \(\xi(t)\) and \(\zeta(t)\) are the
coupling between the eccentricity damping, \(\tau_e(t)\), and the
semimajor axis (sma) damping, \(\tau_m(t)\).

\section{Hamiltonian Mechanics}
\label{sec:org2412dcb}
\section{"Natural scaling" in the solar system}
\label{sec:org5dc48c2}
The following units
\begin{align}
\frac{[GM_\odot][{\rm au}]}{[2\pi{\rm yr}]^2} = 1
\end{align}

\noindent so that time \(\tau(t) \equiv 2\pi t\) is the
dimensionless arc length parameterization of a circular orbit
with sma=1 au and \(t\) is measured in years.
\section{Disturbing function for N body problem}
\label{sec:org3413614}
\end{document}