% Created 2022-02-16 Wed 20:43
% Intended LaTeX compiler: pdflatex
\documentclass[11pt]{article}
\usepackage[utf8]{inputenc}
\usepackage[T1]{fontenc}
\usepackage{graphicx}
\usepackage{longtable}
\usepackage{wrapfig}
\usepackage{rotating}
\usepackage[normalem]{ulem}
\usepackage{amsmath}
\usepackage{amssymb}
\usepackage{capt-of}
\usepackage{hyperref}
\usepackage{amsthm}
\newtheorem{defn}{Definition}
\author{JT Laune}
\date{\textit{<2022-02-08 Tue>}}
\title{MMR Summary}
\hypersetup{
 pdfauthor={JT Laune},
 pdftitle={MMR Summary},
 pdfkeywords={},
 pdfsubject={},
 pdfcreator={Emacs 27.2 (Org mode 9.5.2)}, 
 pdflang={English}}
\begin{document}

\maketitle

\section{Assumptions and notation}
\label{sec:org60efc5e}
\begin{itemize}
\item Two comparable mass orbiters, \(i=1,2\)
\begin{itemize}
\item Internal/external duality around q=1
\end{itemize}
\item Coplanar, \(i_i=0\)
\item Eccentricities are small, \(e_i\ll 1\)
\begin{itemize}
\item First order expansion in \(e_i\)
\end{itemize}
\item Orbiter masses are small, \(m_i\ll M\)
\item Near a first order MMR (1MMR), \(j = 1, 2, \ldots\)
\item independently parameterized \(\dot\varpi_{1,\rm ext}\) and \(\dot\varpi_{2,\rm ext}\)
\end{itemize}

\section{Scaling}
\label{sec:org463d5ca}
We choose the following units
\begin{align}
\frac{[GM_\odot][{\rm au}]}{[2\pi{\rm yr}]^2} = 1
\end{align}

\noindent so that time \(\tau(t) \equiv 2\pi t\) is the
dimensionless arc length parameterization of a circular orbit
with sma=1 au and \(t\) is measured in years.

\section{Perturbation theory in the N-body problem}
\label{sec:org039a230}
\section{Disturbing function for N body problem \href{./images/screenshot-02.png}{7.8}}
\label{sec:org965460f}
\begin{itemize}
\item coefficients: \href{./images/screenshot-03.png}{7.9-7.12}
\item bar(alpha12) = [(alpha12 if j=1 external pert),  (1 if j=2 internal pert)]
\end{itemize}
\section{MMR Hamiltonian Model}
\label{sec:orgd7eca4a}
Index MMR locations with \(j\), such that first order MMRs (hereafter ``1MMRS'') occur whenever
\(n_1/n_2\simeq(j+1)/j\).
Near these points, the relative angle between these 

Near this point, the following geometric angles in the literal expansion of the perturbing
function (see \cite{murray_solar_2000} for details) have commensurate ``fast'' frequencies
\begin{align}
\theta_1 &= (j+1)\lambda_2-j\lambda_1-\varpi_1\\
\theta_2 &= (j+1)\lambda_2-j\lambda_1-\varpi_2
\end{align}


We may use the Poincair\'e elements to write
the dimensionless Hamiltonian as:
\begin{align}
\label{eq:H_1}
  \mathcal{H}
  = -\frac{q^3}{2(1+q)^3 \Lambda_1^2}
    - \frac{1}{2(1+q)^3\Lambda_2^2}
   - \frac{\tilde\mu}{(1+q)^2 \Lambda_2^2}\left[
    f_1\sqrt{\frac{2\Gamma_1}{\Lambda_1}}\cos\theta_1
    +f_2\sqrt{\frac{2\Gamma_2}{\Lambda_2}}\cos\theta_2
    \right],
\end{align}

\noindent where we have defined
\(\tilde\mu=\mu_1\mu_2/(\mu_1+\mu_2)\) to be the reduced mass ratio.
and the \(\theta_i\) are conjugate to \(\Gamma_i\).  For the limiting
cases of \(q\to \infty\) (\(m_2=0\)) or \(q\to 0\) (\(m_1=0\)), the ratio
\(\mathcal{H}/\Lambda_i\) reduces to the standard test particle
Hamiltonian found in \cite{murray_solar_2000} if the limits are taken
carefully.

\section{Poincair\'e's conjugate pair}
\label{sec:org71b92c1}
We utilize the following dimensionless coordinate-momentum conjugate
pairs (aka Poincair\'e coordinates):
\begin{align}
  \lambda_i \longleftrightarrow\Lambda_i &= \mu_i\sqrt{\alpha_i} \\
  -\varpi_i \longleftrightarrow\Gamma_i &= \mu_i\sqrt{\alpha_i}(1-\sqrt{1-e_i^2}) \approx \frac12\mu_i\sqrt{\alpha_i}e_i^2,
\end{align}

\noindent
where \(\varpi_i\) is the longitude of perihelion and \(\lambda_i\) the mean longitude
of orbiter \(m_i\).

\section{Dimensionless parameters}
\label{sec:orgfae2a6a}
We define:
\begin{align}
 q&\equiv \frac{m_1}{m_2}\quad\text{mass ratio with each other}\\ 
 \mu_i&\equiv \frac{m_i}{M}\quad\text{dimless mass}\\ 
 \mu_{\rm tot}&\equiv \mu_1+\mu_2\quad\text{total mass},\\ 
 %\mu_{\rm ext}&\equiv \frac{m_{\rm ext}}{M}\quad\text{total mass},\\ 
\end{align}

\noindent
where \(M\) is the primary mass.

\section{Geometric energy and AM}
\label{sec:org8dbdc89}
In the following, we characterize dissipation by its effects on each
planets' angular momentum (AM) and energy.  A planet's energy,
\(\mathcal E\), is determined by its semimajor axis (sma), \(a\):
\begin{align}
   \mathcal E = -\frac{1}{2a},
\end{align}
\noindent

\noindent
where we have chosen units such that \(GM=1\).
Angular momentum is given by
\begin{align}
h = \mathcal E \sqrt{1-e^2}.
\end{align}

\subsection{Dissipative effects}
\label{sec:org263ff1c}
The dissipative effects are modeled
by two constant timescales for each planet, 
\begin{align}
  \frac{\dot a_i}{a_i} = -\frac{1}{2\pi\tau_{ai}} - \frac{pe_i^2}{2\pi\tau_{ei}} \\
  \frac{\dot e_i}{e_i} = -\frac{1}{2\pi\tau_{ei}} ,
\end{align}

where \(\tau_{ai}\) is the exponential e-damping of sma in years.  The
quantity \(\tau_{ei}\) is the same for eccentricity.

\section{External companion}
\label{sec:orgde87c73}
\subsection{Constant \(\omega_{\rm eff}\)}
\label{sec:orgfcebcce}
Set \(q=0.5\), inward migration. For simplicity, set \(\dot\varpi_{2,\rm
ext} = 0\) and \(\dot\varpi_{1,\rm ext}=\dot\varpi_{\rm eff}\) to be an
arbitrary precession frequency on \(m_1\).

\begin{center}
\includegraphics[width=0.6\textwidth]{./projects/omeff/varyOmeff/q0.50/h-0.03-Tw0-1000-mutot-1.0e-03/000-omeff0-0.000e+00-0.000e+00.png}
\captionof{figure}{Here is the unperturbed system, with \(\mu_{tot}=1e-3\), \$q=0.5, inward migration, and slow dissipative timescales (T\textsubscript{w,0}=10000 years) compared to those in Apsidal Alignment paper.}
\end{center}

\begin{center}
\includegraphics[width=0.6\textwidth]{./projects/omeff/varyOmeff/q0.50/h-0.03-Tw0-1000-mutot-1.0e-03/010-omeff0-3.981e-06-0.000e+00.png}
\captionof{figure}{Here is a perturbed system, with \(\dot\varpi_{1,\rm ext}=\dot\varpi_{\rm eff}\approx 4\times10^{-6}\) and  \(\dot\varpi_{2,\rm ext}=0\).}
\end{center}

\subsubsection{Results summary in final eccentricity and \(\Delta\varpi\)}
\label{sec:orgec09f85}
\begin{center}
\includegraphics[width=0.6\textwidth]{./projects/omeff/varyOmeff/final-Dvarpi-states.png}
\captionof{figure}{Here we have plotted the final apsidal angle as a function of \(\dot\varpi_{1,\rm ext}\)}
\end{center}

\begin{center}
\includegraphics[width=0.6\textwidth]{./projects/omeff/varyOmeff/final-ecc-states.png}
\captionof{figure}{Here we have plotted the final eccentricities as a function of \(\dot\varpi_{1,\rm ext}\)}
\end{center}

\subsubsection{Code validation}
\label{sec:org8c517f8}
\begin{center}
\includegraphics[width=0.6\textwidth]{./mpa/tests/omEff/test-omEff.png}
\captionof{figure}{In the above figure, we have set the total mass to be 1e-7 so that resonant and secular effects are negligible compared to the effects of external precession on \(\gamma_1\).}
\end{center}


\subsection{Secular $\varpi$-precession rates}
\label{sec:org74e4749}
Consider a massive planet on a circular orbit which perturbs an MMR
which lies internal to its orbit.  Let the planet's mass and sma are
given by \(\mu_{\rm ext}\) and \(a_{\rm ext}\).  For each planet \(m_i\) in
the resonance, the interaction Hamiltonian with the external
planet is given by
\begin{equation}
  H_{i,\rm ext} = -\frac14 \frac{\Gamma_i}{\Lambda_i} \mu_{\rm ext}
  \left(\frac{a_i}{a_{\rm ext}}\right) b_{3/2}^{(1)}\left(\frac{a_i}{a_{\rm ext}}\right),
\end{equation}

\noindent
for \(j=1,2\) and we have utilized the approximation \(\Gamma_i \approx \frac12 \Lambda_i e_i^2\).

As a result, each planet experiences a precession in its mean longitude \(\lambda_i\) and
\(\gamma_i\equiv -\varpi_i\). In particular, the \(\dot\varpi_i\) precession frequency
is
\begin{equation}
\dot\varpi_{i, \rm ext} = \frac14 \frac{1}{\Lambda_j} \mu_{\rm ext} 
    \left(\frac{a_i}{a_{\rm ext}}\right) b_{3/2}^{(1)}\left(\frac{a_i}{a_{\rm ext}}\right),
\end{equation}

\subsection{Sign of \(\omega_{\rm eff}\)}
\label{sec:org57c7f63}
One can show
\begin{equation}
  \dot\varpi_{1,\rm ext}-\dot\varpi_{2,\rm ext}
\propto \left(\frac{f_3(a_2/a_{\rm ext})}{f_3(a_1/a_{\rm ext})} - q\right)
\end{equation}

where \(f_3\) is a combination of Laplace coefficients and their derivatives.
\begin{center}
\includegraphics[width=.9\linewidth]{./images/maxq-aext-5.0.png}
\captionof{figure}{Here we have plotted $f_3(a_2/a_{\rm ext})/f_3(a_1/a_{\rm ext})$. The horizontal lines show \(\alpha\) values where MMRs occur.}
\end{center}

\section{References}
\label{sec:org9eddb06}
\bibliography{references}
\bibliographystyle{unsrt}
\end{document}