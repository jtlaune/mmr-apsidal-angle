% Created 2022-03-03 Thu 12:48
% Intended LaTeX compiler: pdflatex
\documentclass[11pt]{article}
\usepackage[utf8]{inputenc}
\usepackage[T1]{fontenc}
\usepackage{graphicx}
\usepackage{longtable}
\usepackage{wrapfig}
\usepackage{rotating}
\usepackage[normalem]{ulem}
\usepackage{amsmath}
\usepackage{amssymb}
\usepackage{capt-of}
\usepackage{hyperref}
\usepackage{amsthm}
\usepackage[margin=1.5in]{geometry}
\newtheorem{defn}{Definition}
\author{JT Laune}
\date{\textit{<2022-02-08 Tue>}}
\title{Coplanar MMR Notes}
\hypersetup{
 pdfauthor={JT Laune},
 pdftitle={Coplanar MMR Notes},
 pdfkeywords={},
 pdfsubject={},
 pdfcreator={Emacs 27.2 (Org mode 9.5.2)}, 
 pdflang={English}}
\begin{document}

\maketitle

\section{Problem}
\label{sec:orgbad3fca}
Consider \(n\) point \textbf{orbiters} with finite masses
\(m_1,m_2,m_3,\ldots\ll M\), where \(M\) is the mass of the primary about
which they orbit. In the absence of dissipation, its Hamiltonian is:
\begin{align}
  \mathcal H
  &\simeq \sum^{n}_{i=1}\left(\frac12 \frac{\vec p_i\cdot\vec p_i}{m_i}
    -\frac{GMm_i}{2|\vec x_i|}
    -\sum_{j\neq i}^{n} \frac{Gm_im_j}{2|\vec x_i - \vec x_j|}
    \right),
\end{align}

\noindent where \(\vec x_i\) and \(\vec p_i\) are the Cartesian position
and momentum vectors, respectively.  For now, we'll be considering the
case of 2 orbiters \(m_1\) and \(m_2\) around a primary of mass \(M\).

We now transform to orbital elements and add the following potential
to the Hamiltonian:
\begin{align}
  \mathcal H_{\rm ext}
  &= -\dot\varpi_{1,\rm ext}\Gamma_1 - \dot\varpi_{1,\rm ext}\Gamma_2,
\end{align}

\noindent where the \(\Gamma_i=\frac12m_i\sqrt{a_i}(1-\sqrt{1-e_i^2})\)
are the canonically conjugate momenta for the canonical coordinate
\(-\varpi_i\).  This induces constant differential rotation in the
Runge-Lenz vectors, \(\vec{e}_1\) and \(\vec{e}_2\).  We treat
\(\dot\varpi_{i,\rm ext}\) as free, constant parameters.

\textbf{How does differential apsidal precession affect the dynamics of first order MMRs?}

\subsection{Parameters}
\label{sec:orgd6a6aa0}
We define:
\begin{align}
 q&\equiv \frac{m_1}{m_2}\quad\text{mass ratio with each other}\\ 
 \mu_i&\equiv \frac{m_i}{M}\quad\text{dimless mass}\\ 
 \mu_{\rm tot}&\equiv \mu_1+\mu_2\quad\text{total mass},\\ 
 %\mu_{\rm ext}&\equiv \frac{m_{\rm ext}}{M}\quad\text{total mass},\\ 
 \dot\varpi_{i, \rm ext} &\equiv \text{externally induced precession of $\varpi_i$} \\
 \dot\varpi_{\rm eff} &\equiv \dot\varpi_{1,\rm ext}-\dot\varpi_{2,\rm ext} \quad\text{differential precession} \\
\end{align}

\noindent
where \(M\) is the primary mass.

\subsection{Assumptions and notation}
\label{sec:org0f3d6be}
\begin{itemize}
\item Coplanar
\item Orbiter masses are small \(\mu_i\ll 1\)
\item Two situations:
\begin{itemize}
\item Two comparable mass orbiters, internal/external denoted by
\(i=1,2\), mass ratio \(q\equiv m_1/m_2\).
\item Test particle limit, \(q\to0,\infty\). Denote test particle with no
subscript and perturbing mass with \(p\) subscript. Test particle
can be inside or outside the perturber.
\end{itemize}
\item Near a first order MMR (1MMR), defined by \(P_2/P_1\simeq (j+1)/j\),
with \(j = 1, 2, \ldots\)
\begin{itemize}
\item Eccentricities are small, \(e_i\ll 1\)
\begin{itemize}
\item First order expansion of perturbing potential in \(e_i\)
\end{itemize}
\end{itemize}
\item independently parameterized \(\dot\varpi_{1,\rm ext}\) and
\(\dot\varpi_{2,\rm ext}\) induced by an arbitrary potential
\end{itemize}

\section{MMR Hamiltonian Model}
\label{sec:org6468359}
Index MMR locations with \(j\), such that first order MMRs (hereafter ``1MMRS'') occur whenever
\(n_1/n_2\simeq(j+1)/j\).
Near these points, the following angles in the literal expansion of the perturbing
function (see \cite{murray_solar_2000} for details) have commensurate ``fast'' frequencies
\begin{align}
\theta_1 &= (j+1)\lambda_2-j\lambda_1-\varpi_1\\
\theta_2 &= (j+1)\lambda_2-j\lambda_1-\varpi_2,
\end{align}

\noindent
because the terms \((j+1)\lambda_2-j\lambda_1\approx 0\) near the MMR locations (by definition).

We choose dimensionless time units so that \(GM=1\) and
the Hamiltonian is given by
\begin{equation}
  \mathcal{H} = -\frac{m_1}{2a_1} -\frac{m_2}{2a_2} - \mu_2 m_1 f_1(\alpha) e_1 \cos\theta_1 + \mu_1 m_2 f_2(\alpha) e_2\cos\theta_2
\end{equation}

\noindent
where the mass ratios \(\mu_1\) and \(\mu_2\) now play the part as the perturbation parameter.

In the restricted planar 3-body problem, one planet is a test particle and has mass equal to zero.
For an internal (external) test particle,
the appropriate limit for the Hamiltonian above is \(m_1\to0\) (\(m_2\to0\)).
The Hamiltonian for an inner test particle is therefore
\begin{equation}
  \mathcal{H} = -\frac{1}{2a_1} - \mu_p( f_1(\alpha) e_1 \cos\theta_1 + f_2(\alpha) e_p\cos\theta_p)
\end{equation}

\noindent

\section{Test particle results for \(\dot\varpi_{\rm eff}\)}
\label{sec:org4ed446b}
From the definitions of the angles \(\theta_1\) and \(\theta_2\),
any external precession frequency will "split" the location
of \(\theta_1\) and \(\theta_2\). The effects of this
shift are summarized in Figure \ref{fig:resShiftDiag}.


\begin{center}
\includegraphics[width=1\textwidth]{/home/jtlaune/multi-planet-architecture/projects/omeff/images/resShiftDiag.png}
\captionof{figure}{\label{fig:resShiftDiag}Nominal MMR locations vs how they shift due to external precession.}
\end{center}

In the next section, we show the results of time dependent integrations
of the Hamiltonian.

\subsection{Equilibrium eccentricities}
\label{sec:org02d6f87}
Set \(\mu_p=1e-4\), \(a_p=1\), and let the test particle migrate starting
wide of the \(3:2\) MMR. The perturber is on a stationary orbit, unlike
Maryam's paper. Differential rotation couples to the perturber
eccentricity to possibly disrupt capture. Timescales \(T_e = 10^3\) and
\(T_m = 10^5\) yrs.

\begin{center}
\includegraphics[width=0.6\textwidth]{/home/jtlaune/multi-planet-architecture/projects/omeff/external-tp-ep0.03.png}
\captionof{figure}{\label{fig:}Final averaged eccentricity from the time-dependent integrations of the Hamiltonian. The vertical line denotes the resonance width, \(\delta\sim\mu_p^{2/3}\).  The horizontal line denotes \(e_p\).  We leave \(\varpi_p=0\) fixed and assume we are in the rotating frame of the planet. The differential precession is parameterized by \(\dot\varpi_{\rm eff} \equiv \dot\varpi_{1, \rm eff}\) (i.e., we only apply precession to the test particle). For frequencies \(\dot\varpi_{\rm\eff}\gtrsim\delta\), the particle's average eccentricities in equilibrium are altered by external precession.}
\end{center}

\subsubsection{Additional examples}
\label{sec:org564df2c}
\begin{center}
\includegraphics[width=0.6\textwidth]{/home/jtlaune/multi-planet-architecture/projects/omeff/external-tp-ep0.00.png}
\end{center}
\begin{center}
\includegraphics[width=0.6\textwidth]{/home/jtlaune/multi-planet-architecture/projects/omeff/external-tp-ep0.05.png}
\end{center}
\begin{center}
\includegraphics[width=0.6\textwidth]{/home/jtlaune/multi-planet-architecture/projects/omeff/external-tp-ep0.07.png}
\end{center}
\begin{center}
\includegraphics[width=0.6\textwidth]{/home/jtlaune/multi-planet-architecture/projects/omeff/external-tp-ep0.10.png}
\end{center}

\subsection{Librating angles}
\label{sec:org4a41a43}
\subsubsection{Internal \(\dot\varpi_{\rm eff}\in[10^{-4},10^{-1}]\)}
\label{sec:org92ac65e}
Set \(\mu_p=1e-4\), \(a_0=1\), and let the test particle migrate outwards.
The capture process leads to several different combinations of
librating resonant angles, summarized in Tables \ref{tab:int-pos} and
\ref{tab:int-neg}.  The simulations extended down to \(\dot\varpi_{\rm
eff}=10^{-8}\), but the results were not interesting.  We set
\(\dot\varpi_{\rm eff} = \dot\varpi_{1,\rm ext}\) and set
\(\dot\varpi_{2, \rm ext}=0\). The idea here is that we can always
transfer to the constantly-rotating frame of \(\mu_p\).


\begin{table}[htbp]
\caption{\label{tab:int-pos}Librating angles for an internal test particle with \(\varpi_{\rm eff}>0\). First two rows demonstrate weak coupling to \(e_p\) parameter. Last two rows demonstrate bifurcation in resonance behavior due to \(e_p-\om_{\rm eff}\) coupling.  Blank spots are just our migration model failing for the internal case (not a problem).  None means the resonant capture was disrupted and none of the three resonance angles, \(\theta_1\), \(\theta_2\), and \(\hat\theta\), are librating.}
\centering
\begin{tabular}{| l | c | c | c | c |}
\hline
\dot\(\varpi\)\textsubscript{\rm eff} & 10\textsuperscript{-1} & 10\textsuperscript{-2} & 10\textsuperscript{-3} & 10\textsuperscript{-4}\\
\hline
e\textsubscript{p}=0 & \(\theta\)\textsubscript{1}, \(\hat\theta\) & \(\theta\)\textsubscript{1}, \(\hat\theta\) & \(\theta\)\textsubscript{1}, \(\hat\theta\) & \(\theta\)\textsubscript{1}, \(\hat\theta\)\\
\hline
e\textsubscript{p}=0.001 & \(\theta\)\textsubscript{1}, \(\hat\theta\) & \(\theta\)\textsubscript{1}, \(\hat\theta\) & \(\theta\)\textsubscript{1}, \(\hat\theta\) & \(\theta\)\textsubscript{1}, \(\hat\theta\)\\
\hline
e\textsubscript{p}=0.03 & \(\theta\)\textsubscript{1} & \(\theta\)\textsubscript{1}, \(\theta\)\textsubscript{2}, \(\hat\theta\) & None & None\\
\hline
e\textsubscript{p}=0.1 & \(\theta\)\textsubscript{1}, \(\theta\)\textsubscript{2}, \(\hat\theta\) & \(\theta\)\textsubscript{1}, \(\theta\)\textsubscript{2}, \(\hat\theta\) & None & \\
\hline
\end{tabular}
\end{table}

\begin{table}[htbp]
\caption{\label{tab:int-neg}Same as Table \ref{tab:int-pos} but with \(\varpi_{\rm omeff}<0\).}
\centering
\begin{tabular}{| l | c | c | c | c |}
\hline
\dot\(\varpi\)\textsubscript{\rm eff} & -10\textsuperscript{-1} & -10\textsuperscript{-2} & -10\textsuperscript{-3} & -10\textsuperscript{-4}\\
\hline
e\textsubscript{p}=0 & \(\theta\)\textsubscript{1}, \(\hat\theta\) & \(\theta\)\textsubscript{1}, \(\hat\theta\) & \(\theta\)\textsubscript{1}, \(\hat\theta\) & \(\theta\)\textsubscript{1}, \(\hat\theta\)\\
\hline
e\textsubscript{p}=0.001 & \(\theta\)\textsubscript{1}, \(\hat\theta\) & \(\theta\)\textsubscript{1}, \(\hat\theta\) & \(\theta\)\textsubscript{1}, \(\hat\theta\) & \(\theta\)\textsubscript{1}, \(\hat\theta\)\\
\hline
e\textsubscript{p}=0.03 & \(\theta\)\textsubscript{1} & \(\theta\)\textsubscript{1} & None & \\
\hline
e\textsubscript{p}=0.1 & \(\theta\)\textsubscript{1} & \(\theta\)\textsubscript{1} & None & \\
\hline
\end{tabular}
\end{table}

\subsubsection{External \(\dot\varpi_{\rm eff}\in[10^{-4},10^{-1}]\)}
\label{sec:orgb4196a8}
\begin{center}
\begin{tabular}{lllll}
\hline
\dot\(\varpi\)\textsubscript{\rm eff} &  &  &  & \\
\hline
e\textsubscript{p}=0 &  &  &  & \\
\hline
e\textsubscript{p}=0.001 &  &  &  & \\
\hline
e\textsubscript{p}=0.03 &  &  &  & \\
\hline
e\textsubscript{p}=0.1 &  &  &  & \\
\hline
\end{tabular}
\end{center}
\subsection{Examples for \(e_p=0.03\)}
\label{sec:orgfbd97aa}
\subsubsection{Cutoff at 150kyr}
\label{sec:org882160c}
The systems are stable for \(e_p=0.03\) even after shutting off
the dissipative forces halfway through. Appears to be a stable
equilibrium.

\begin{center}
\includegraphics[width=0.7\textwidth]{./tpOmEff/cutoff-ep0.03/0000-omeff-1.00e-04.png}
\label{fig:resShiftDiag}
\end{center}
\begin{center}
\includegraphics[width=0.7\textwidth]{./tpOmEff/cutoff-ep0.03/0004-omeff-1.39e-03.png}
\label{fig:resShiftDiag}
\end{center}
\begin{center}
\includegraphics[width=0.7\textwidth]{./tpOmEff/cutoff-ep0.03/0005-omeff-2.68e-03.png}
\label{fig:resShiftDiag}
\end{center}
\begin{center}
\includegraphics[width=0.7\textwidth]{./tpOmEff/cutoff-ep0.03/0007-omeff-1.00e-02.png}
\label{fig:resShiftDiag}
\end{center}
\section{References}
\label{sec:org2ef6820}
\bibliography{references}
\bibliographystyle{unsrt}

\clearpage
\onecolumn
\appendix

\section{Poincair\'e's conjugate pair}
\label{sec:org748b5c4}
We utilize the following dimensionless coordinate-momentum conjugate
pairs (aka Poincair\'e coordinates):
\begin{align}
  \lambda_i \longleftrightarrow\Lambda_i &= \mu_i\sqrt{\alpha_i} \\
  -\varpi_i \longleftrightarrow\Gamma_i &= \mu_i\sqrt{\alpha_i}(1-\sqrt{1-e_i^2}) \approx \frac12\mu_i\sqrt{\alpha_i}e_i^2,
\end{align}

\noindent
where \(\varpi_i\) is the longitude of perihelion and \(\lambda_i\) the mean longitude
of orbiter \(m_i\).

\section{Geometric energy and AM}
\label{sec:orge5528e8}
In the following, we characterize dissipation by its effects on each
planets' angular momentum (AM) and energy.  A planet's energy,
\(\mathcal E\), is determined by its semimajor axis (sma), \(a\):
\begin{align}
   \mathcal E = -\frac{1}{2a},
\end{align}
\noindent

\noindent
where we have chosen units such that \(GM=1\).
Angular momentum is given by
\begin{align}
h = \mathcal E \sqrt{1-e^2}.
\end{align}

\subsection{Dissipative effects}
\label{sec:org66b76d8}
The dissipative effects are modeled
by two constant timescales for each planet, 
\begin{align}
  \frac{\dot a_i}{a_i} = -\frac{1}{2\pi\tau_{ai}} - \frac{pe_i^2}{2\pi\tau_{ei}} \\
  \frac{\dot e_i}{e_i} = -\frac{1}{2\pi\tau_{ei}} ,
\end{align}

where \(\tau_{ai}\) is the exponential e-damping of sma in years.  The
quantity \(\tau_{ei}\) is the same for eccentricity.

\section{Effects of quadrupole potential}
\label{sec:org661975e}
A quadrupole potential may arise as a result of secular perturbations
from nearby planets on circular orbits or a \(J_2\) moment in the
primary's gravitational field. Due to the difference in sma
between any two orbiters, a quadrupole potential induces
differential apsidal precession on the orbiters.

\subsection{Derivation of differential precession rate \(\omega_{\rm eff}\)}
\label{sec:org88fbeb8}
Consider a massive planet on a circular orbit which perturbs an MMR
which lies internal to its orbit.  Let the planet's mass and sma are
given by \(\mu_{\rm ext}\) and \(a_{\rm ext}\).  For each planet \(m_i\) in
the resonance, the interaction Hamiltonian with the external
planet is given by
\begin{equation}
  H_{i,\rm ext} = -\frac14 \Gamma_i \mu_{\rm ext}
  \left(\frac{a_i}{a_{\rm ext}}\right) b_{3/2}^{(1)}\left(\frac{a_i}{a_{\rm ext}}\right),
\end{equation}

\noindent
for \(j=1,2\) and we have utilized the approximation \(\Gamma_i \approx \frac12 \Lambda_i e_i^2\).

As a result, each planet experiences a precession in its mean longitude \(\lambda_i\) and
\(\gamma_i\equiv -\varpi_i\). In particular, the \(\dot\varpi_i\) precession frequency
is
\begin{equation}
\dot\varpi_{i, \rm ext} = \frac14 \mu_{\rm ext} 
    \left(\frac{a_i}{a_{\rm ext}}\right) b_{3/2}^{(1)}\left(\frac{a_i}{a_{\rm ext}}\right),
\end{equation}

\subsubsection{Murray and Dermott}
\label{sec:orgf0637cd}
\begin{itemize}
\item secular perturbations \href{./images/screenshot-02.png}{7.8}
\item coefficients: \href{./images/screenshot-03.png}{7.9-7.12}
\item bar(alpha12) = [(alpha12 if j=1 external pert),  (1 if j=2 internal pert)]
\end{itemize}

\section{Formal constructions}
\label{sec:org02e812f}
The \textbf{Kepler problem} is a special case of the \textbf{2-body problem}.
Its solutions are\ldots{}

We may characterize dissipation by its action on the \ldots{}
\(\tau_{mi}(t)\) and \(\tau_{ei}(t)\) by the instantaneous derivatives
\begin{align}
   \frac{\dot e_i}{e_i} &= - \frac{1}{\tau_e(t)} - \xi(t, \mathbf X_i)\frac{1}{\tau_m(t)} \\
   \frac{\dot a_i}{a_i} &= -\frac{1}{\tau_m(t)} - \zeta(t, \mathbf X_i)\frac{1}{\tau_e(t)},
\end{align}
\noindent

\noindent
where the dot notation corresponds to the time derivative of the
orbital elements. The functions \(\xi(t)\) and \(\zeta(t)\) are the
coupling between the eccentricity damping, \(\tau_e(t)\), and the
semimajor axis (sma) damping, \(\tau_m(t)\).

\section{Hamiltonian Mechanics}
\label{sec:orga72c4e2}
\section{"Natural scaling" in the solar system}
\label{sec:org467b0cf}
The following units
\begin{align}
\frac{[GM_\odot][{\rm au}]}{[2\pi{\rm yr}]^2} = 1
\end{align}

\noindent so that time \(\tau(t) \equiv 2\pi t\) is the
dimensionless arc length parameterization of a circular orbit
with sma=1 au and \(t\) is measured in years.
\section{Disturbing function for N body problem}
\label{sec:orgc5e72c2}
\end{document}