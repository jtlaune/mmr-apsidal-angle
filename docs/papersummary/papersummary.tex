\documentclass{article}
%section standard commands/packages
\twocolumn
\usepackage[margin=1.0in]{geometry}
\usepackage{pdfpages}
\usepackage{booktabs}
\usepackage{enumitem}
\usepackage{graphicx}
\usepackage{amsmath}
\usepackage{amssymb}
\usepackage{tensor}
\usepackage{amsthm}
\usepackage{ wasysym }
\usepackage{amsfonts}
\usepackage{mathtools}
\usepackage{xcolor}
\usepackage{cancel}
\newcommand{\note}[1]{{\color{red} #1 }}
\newcommand{\extra}[1]{{}}
\newcommand{\F}{\mathcal{F}}
\newcommand{\edisk}{e_{\rm disk}}
\newcommand{\omext}{\omega_{\rm ext}}
\newcommand{\ep}{e_{\rm p}}
\newcommand{\Te}{T_{\rm e}}
\newcommand{\Tm}{T_{\rm m}}
\newcommand{\ebar}{\bar{e}}
\newcommand{\gammabar}{\bar{\gamma}}
\newcommand{\T}{\mathcal{T}}
\renewcommand{\L}{\mathcal{L}}
\renewcommand{\d}{\partial}
\renewcommand{\v}[1]{\boldsymbol{ #1 }}
\renewcommand{\t}[1]{\tilde{ #1 }}
\newcommand{\tg}{\t{g}}
\newcommand{\vh}[1]{\hat{\boldsymbol{ #1 }}}
\newcommand{\bbR}{\mathbb{R}}
\newcommand{\pp}[2]{\frac{\partial #1}{\partial #2}}
\newcommand{\dd}[2]{\frac{d #1}{d #2}}
\usepackage{tikz}
\newcommand*\circled[1]{\tikz[baseline=(char.base)]{
            \node[shape=circle,draw,inner sep=2pt] (char) {#1};}}
\DeclarePairedDelimiter{\abs}{|}{|}
\DeclarePairedDelimiter{\p}{(}{)}
\DeclarePairedDelimiter{\we}{\langle}{\rangle}
\newtheorem{prob}{Problem}
\usepackage{tikz,colortbl}
\usetikzlibrary{calc}
\usepackage{zref-savepos}
\newcounter{NoTableEntry}
\renewcommand*{\theNoTableEntry}{NTE-\the\value{NoTableEntry}}
%section table shenanigans from stack exchange
\newcommand*{\notableentry}{%
  \multicolumn{1}{@{}c@{}|}{%
    \stepcounter{NoTableEntry}%
    \vadjust pre{\zsavepos{\theNoTableEntry t}}% top
    \vadjust{\zsavepos{\theNoTableEntry b}}% bottom
    \zsavepos{\theNoTableEntry l}% left
    \hspace{0pt plus 1filll}%
    \zsavepos{\theNoTableEntry r}% right
    \tikz[overlay]{%
      \draw[black]
        let
          \n{llx}={\zposx{\theNoTableEntry l}sp-\zposx{\theNoTableEntry r}sp},
          \n{urx}={0},
          \n{lly}={\zposy{\theNoTableEntry b}sp-\zposy{\theNoTableEntry r}sp},
          \n{ury}={\zposy{\theNoTableEntry t}sp-\zposy{\theNoTableEntry r}sp}
        in
        (\n{llx}, \n{lly}) -- (\n{urx}, \n{ury})
        (\n{llx}, \n{ury}) -- (\n{urx}, \n{lly})
      ;
    }% 
  }%
}

\usepackage{array}
\usepackage{makecell}

\renewcommand\theadalign{bc}
\renewcommand\theadfont{\bfseries}
\renewcommand\theadgape{\Gape[4pt]}
\renewcommand\cellgape{\Gape[4pt]}

\title{Eccentric MMR with Perturber}
\author{JT Laune}
\begin{document}
\maketitle
\section{Relevant Background}
%subsection intro
In this section, we consider a $j:j+1$ MMR between a test particle and
a planet of mass $\mu_p$.  For the rest of this report, the subscript
``$p$'' denotes orbital parameters of the massive planet, no subscript
denotes the test particle, and a subscript ``$\rm{ext}$'' denotes an
external perturber.  The Hamiltonian is given by
\begin{align}\label{eq:H}
  H
  &= - \frac{1}{2\Lambda^2} \pm \mu_p\left(A
    \left(\frac{2\Gamma}{\Lambda}\right)^{1/2}\cos\theta + B
    \ep\cos\theta_p\right) \nonumber\\
  &- \mu_p \left(C\left(\frac{2\Gamma}{\Lambda}
      + \ep^2\right) + D\ep
    \sqrt{\frac{2\Gamma}{\Lambda}}\cos\gamma\right)
\end{align}
The ($\pm$) term is ($+$) for an internal resonance, and ($-$) for an
external resonance.  The period ratio $\alpha$ is defined to be
$a/a_p$ for an internal resonance, and $a_p/a$ for an outer resonance.
$A,B,C$ and $D$ are functions of $\alpha$, and their definitions are
given in \ref{app:coeff} in terms of Laplace coefficients. They are
different expressions, values, and signs depending on whether or not
the resonance is internal or external.

\subsection{Disk forces}
The forces the disk exerts on the test particle are given
in a parametrized fashion by $\Te$, the eccentricity damping timescale,
and $\Tm$, the migration timescale:z
\begin{equation}
  \label{eq:dota}
  \frac{\dot{a}}{a} = \pm\frac{1}{\Tm} - \frac{2e^2}{\Te}
\end{equation}
The ($\pm$) becomes ($+$) for outward migration, and ($-$) for inward migration. 
The eccentricity damping force is given by
\[ \frac{\dot{e}}{e} = -\frac{1}{\Te}. \]
We note that $\Te>0$ always.

\subsection{Equilibrium eccentricity}
In the case of a massive planet on a circular orbit ($\ep=0$), a test
particle in mean motion resonance will reach an equilibrium
eccentricity depending on the ratio $\sqrt{\Te/\Tm}$.

For an internal resonance, the equilibrium eccentricity is given by
\[ \edisk = \sqrt{\frac{\Te}{2j \Tm}} \]

For an external resonance, the equilibrium eccentricity is given by
\[ \edisk = \sqrt{\frac{\Te}{2(j+1) \Tm}} \]

\note{INSERT relating $e_{\rm disk}$ to disk aspect ratio}.

\subsection{Stability}
For both internal and external resonances, the planet must be massive enough
to capture the test particle:
\[\mu_{\rm cap}^{4/3}\gg \frac{1}{T_m}.\]

In the case of an external test particle, the resonance is always stable
and librations go to zero.

For an internal test particle, stability depends upon the disk parameters
and planet mass $\mu_p$:
\begin{align}
  \label{eq:stability}
\textbf{Stable, no libration}:&\quad \mu_p > \frac{3j^2}{\alpha_0 A} e_{\rm eq}^3 \nonumber\\
\textbf{Stable, finite libration}:&\quad \frac{3j^2}{\alpha_0 A} e_{\rm
 eq}^3 > \mu_p > \frac{3j}{8\alpha_0 A} e_{\rm eq}^3\nonumber\\
\textbf{Unstable, escape}:&\quad \frac{3j}{8\alpha_0 A} e_{\rm eq}^3 > \mu_p.
\end{align}

\subsection{Eccentricity of $\mu_p$}
Through a series of canonical transformations, we may combine the $\ep$
term of the Hamiltonian with the $e$ term.
The transormed Hamiltonian for $\ep > 0$ is
\[ H = \eta R - R^2 - \sqrt{R}\cos\bar{\theta}. \]
The momentum $R\propto \ebar^2$, where $\ebar$ is
\[ \ebar^2 = e^2 + \frac{2B}{A}e\ep\cos\gamma + \frac{B^2}{A^2}\ep^2. \]
The ``shifted'' $\overline{\gamma}$ is given by
\[ \overline{\gamma} = \tan^{-1}\p*{\frac{e\sin\gamma}{e\cos\gamma + Be_p/A}} \]
The resonance angle is then given by
\[ \theta = (j+1)\lambda_p - j\lambda + \overline{\gamma} \]
for an internal resonance, and
\[ \theta = (j+1)\lambda - j\lambda_p + \overline{\gamma} \]
for an external resonance.
We will only be utilizing $\ebar$ and $\gammabar$ in this report,
but the full details of this are given in \ref{app:reducedH} for completeness.
\subsection{External Perturber}\label{sec:extpert}
A massive external companion induces relative apsidal precession within the MMR pair.
We use a parametrized approach, by transforming to the rotating frame of $\gamma_p$ and
then setting $\dot\gamma = -\omext$ with the following Hamiltonian:
  \begin{align*}
    H_{\gamma,\rm sec} =& -\Gamma(\omega_{1,\rm ext} - \omega_{p,\rm ext}) \equiv -\Gamma\omext.  \\
  \end{align*}
  The units of $\omext$ are in $[n_p]$.

  An external companion will also induce precession in $\lambda$,
  which we may include with the following Hamiltonian
  \begin{align*}
H_{\lambda,\rm sec} =&
  -\frac{GM}{2a_p} \mu_p b_{1/2}^{(0)}\p*{\frac{a}{a_p}}
- \frac{GM}{2a_{\rm ext}} \mu_{\rm ext} b_{1/2}^{(0)}\p*{\frac{a}{a_{\rm ext}}}\\
                      &- \frac{GM}{2a_{\rm ext}} \mu_{\rm ext} b_{1/2}^{(0)}\p*{\frac{a_p}{a_{\rm ext}}}
  \end{align*}

  We treat $a_{\rm ext}$ as an independent variable, and choose it to be $a_{\rm ext} = 3$
  throughout this work. Then, we solve for $\mu_{\rm ext}$ with the following formulas:
  \[ \omega_{1,\rm ext} = \mu_{\rm ext} \p*{\frac{a_p}{a}}^{3/2} \frac{a^2}{4a_{\rm ext}^2}
    b_{3/2}^{(1)}\p*{\frac{a}{a_{\rm ext}}} \]
  \[ \omega_{p,\rm ext} = \mu_{\rm ext} \frac{a_p^2}{4a_{\rm ext}^2}
    b_{3/2}^{(1)}\p*{\frac{a_p}{a_{\rm ext}}}. \]

\subsection{Resonance width}
The resonance widths in $a$- and $e$-space, in units of $\delta n$, are given by
  \[ \frac{\delta a}{a} \sim \frac{\delta n}{n} \sim \mu_p^{2/3}\]
  \[ \frac{\delta e}{e}\sim \mu_p^{1/3}\]
\note{INSERT derivation or citation}
\subsection{(WIP) Resonance splitting}
\note{INSERT discussion of resonance splitting, etc}
\section{Summary of Results}
See Table~\ref{tab:runs} for a summary of the integrations carried out in this work.
See Table~\ref{tab:summary} for a compact summary of the regimes explored in this work,
as well as our notable results.
\begin{table*}[h]
  \centering
  \begin{tabular}{| l | c | c | c | c | c |}
    \hline
   Regime &$\edisk$ &$e_p$ &$\omext$ &\# runs &Figures \\ 
    \hline
    \hline
    \multicolumn{6}{|c|}{\textbf{External}} \\
    \hline
    $\gamma$-alignment& [0.01,0.1]&[0.01,0.1] &0 &25 &\ref{fig:examples},\ref{fig:align-external}\\
    \hline
    \makecell{external\\ precession}& [0.01,0.1]&[0.01,0.1] &[0.1,10]$\delta n$ &250 &\ref{fig:gcomponents},\ref{fig:gammadot-behaviors},\ref{fig:newres}\\
    \hline
    \hline
    \multicolumn{6}{|c|}{\textbf{Internal}} \\
    \hline
    $\gamma$-alignment& [0.01,0.1]&[0.01,0.1] &0 &15 &\ref{fig:align-internal}\\
    \hline
  \end{tabular}
  \caption{Summary of the integrations carried out in this work. N-body simulations are not included, as they are a WIP. The eccentricities are 5 values distributed equally in log space, and the
  external precession rates are 10 values distributed the same.}
  \label{tab:runs}
\end{table*}

\begin{table*}[h]
  \centering
  \begin{tabular}{| l || c || c || c | c | c |}
    \hline
    & \textbf{Internal} & \multicolumn{4}{c|}{\textbf{External}} \\
    \hline\hline
    &$\omext = 0$ &$\omext = 0$ &$\omext <\delta n$ &$\omext \approx \delta n$ &$\omext > \delta n$ \\ 
    \hline\hline
    $e_p = 0$ & \notableentry& \notableentry&\makecell{unchanged\\from $\omext=0$}&\makecell{unchanged\\from $\omext=0$} &\makecell{unchanged\\from $\omext=0$}\\
    \hline
    $\edisk < \ep$ & \notableentry& $\gamma$-aligned&\makecell{unchanged\\from $\omext=0$}& \makecell{$\gamma$-anti-aligned\\$e_1$ excited}& \makecell{$\gamma$-anti-aligned\\$e_1$ suppressed}\\
    \hline
    $\edisk \approx \ep$ & $\gamma$-aligned& $\gamma$-circulating&\makecell{unchanged\\from $\omext=0$}&chaotic& \makecell{$\gamma$-anti-aligned\\$e_1$ suppressed} \\
    \hline
    $\edisk > \ep$ & $\gamma$-circulating& $\gamma$-circulating& \makecell{unchanged\\from $\omext=0$}&\makecell{$\gamma$-anti-aligned\\$e_1$ excited}& unclear trend \\
    \hline
  \end{tabular}
  \caption{Summary of the results from this study for
    $\gamma$-alignment and eccentricity suppression/excitation. In the
    case of \textbf{``unclear trend''}, the behavior varies between
    [$\gamma$-circulating] and [$\gamma$-anti-aligned, $e_1$ excited].
    \textbf{``Chaotic''} means that, for closely spaced starting
    locations $a_0$ and holding all other conditions equal, the
    behaviors of $\gamma_1$ and $e_1$ varied between
    [$\gamma$-circulating] and [$\gamma$-anti-aligned, $e_1$ excited].
    The results presented here for $\omext = 0$ are summarizing the
    plots in Figures~\ref{fig:align-external} and
    \ref{fig:align-internal}. For $\omext >0$, they are summarizing
    the plots in Figure~\ref{fig:gammadot-behaviors}. We note that
    these are a general summary of the results, and sometimes there
    are exceptions (e.g., for $e_p = \edisk = 0.01$ and
    $\omext\approx\delta n$, the runs are not [$\gamma$-anti-aligned,
    $e_1$ excited], see the leftmost plot in
    Figure~\ref{fig:gammadot-behaviors}).}
  \label{tab:summary}
\end{table*}

\section{Unperturbed ($\omext=0$) resonant behavior}
\subsection{Equilibrium eccentricity}
For $e_{\rm disk} > e_p >0$, the test particle librates around
$e_{\rm disk}$, as expected (see the right of
Figure~\ref{fig:examples}).
Whenever $\edisk<e_p$, however, the librations of $e_1$ are typically large
and centered away from $\edisk$.
\subsection{Apsidal alignment}
%subsubsection intro
For the $\edisk\lesssim \ep$ regime, when the disk forces dominate over the eccentricity of $\mu_p$,
$\dot{\gamma}\approx 0$ and $\abs{\gamma-\gamma_p}\approx 0$, resulting in
apsidal alignment. In the opposite regime, $\dot{\gamma}\neq 0$ and
$\abs{\gamma - \gamma_p}$ circulates.
For $\edisk \sim \ep$ we must turn to numerical
simulations to find the boundary between $\gamma$-aligned and $\gamma$-circulating systems.

\subsubsection{External}

\begin{figure*}[htb]
  \centering
 \includegraphics[width=0.4\textwidth]{../examples/aligned-eq1.78e-02-ep1.00e-01-om0.00e+00.png}
 \includegraphics[width=0.4\textwidth]{../examples/circulating-eq1.78e-02-ep1.00e-02-om0.00e+00.png}
  \caption{ }
  \label{fig:examples}
\end{figure*}

In Figure~\ref{fig:examples}, we plot our integrations for
$\edisk = 0.018$ and $e_p = 0.1$ (left, $\gamma$-aligned) and
$e_p = 0.01$ (right, $\gamma$-circulating).  
Figure \ref{fig:align-external} summarizes the apsidal alignment
results for the case of an external resonance.  We have a similar
trend of apsidal alignment for $\ep \gtrsim e_{\rm eq}$ with $\gamma$
circulating for the opposite regime. The main difference from the
internal case is that for the simulations with $\ep = e_{\rm eq}$,
$\gamma$ is no longer aligned with $\gamma_P$.
\begin{figure}[htb]
  \centering
  \includegraphics[width=0.45\textwidth]{figures/align-external-grid-1e-3.png}
  \caption{ }
  \label{fig:align-external}
\end{figure}

The general behavior of $\gamma$ may be understood from its equation of motion
in an external resonance (see \ref{app:eom}):
\[\dot{\gamma} = \mu_p\left(\frac{-A\cos\theta}{\sqrt{2\Gamma\Lambda}}
    - \frac{2C}{\Lambda} - D\ep\frac{\cos\gamma}{\sqrt{2\Gamma\Lambda}} \right)\]
where
\[ \theta = (j+1)\lambda - j\lambda_p + \gamma\]
is the old resonance angle.
We have
\[ -A\cos\theta{\sqrt{2\Gamma\Lambda}}\propto \frac1e \]
\[- \frac{2C}{\Lambda}\propto \mathcal{O}(1)\]
\[- D\ep\left(\frac{\cos\gamma}{\sqrt{2\Gamma\Lambda}} \right)\sim \mathcal{O}(1)\]
and so the first term typically dominates the motion if $\theta \neq \pi/2$

The shifted resonance angle, however, is
\[ \overline{\theta} = (j+1)\lambda - j\lambda_p + \overline{\gamma}\]
where
\[ \overline{\gamma} = \tan^{-1}\left(\frac{e\sin\gamma}{e\cos\gamma + B\ep/A}\right) \]

When $\ep\lesssim e_{\rm eq}$, $\gamma\approx\overline{\gamma}$ and $\theta\to 0$,
meaning the first term will dominate $\dot{\gamma}$.
However, when $\ep\gtrsim e_{\rm eq}$, $\theta$ circulates and the first term
averages to 0, leading to the second two terms governing the equation of motion.
For these two terms, if $\ep \gtrsim e_{\rm eq}$, an equilibrium
exists where $\dot{\gamma}\to 0$
for $\gamma\to 0$.

The behavior of
$\dot{\gamma}$ and its individual components along with the
corresponding behavior of $\gamma,\theta,\overline{\theta}$ is plotted
in Figure \ref{fig:gcomponents}.  From the left column to the right,
we hold $\edisk = 0.03$ and increase $\ep$ uniformly in log-space
between 0.01 and 0.1. We can see that between the third and fourth
columns, at the transition between apsidal circulation and alignment,
$\theta$ begins to circulate, and vice versa. Note that
$\overline{\theta}\to\pi$ always, as it should in resonance.

\begin{figure}[htb]
  \centering
  \includegraphics[width=0.45\textwidth]{figures/align-internal-grid-relmup.png}
  \caption{ }
  \label{fig:align-internal}
\end{figure}

\begin{figure*}[htb]
  \centering
  \includegraphics[width=0.19\textwidth]{{figures/gammacomps0omeff/eq3.16e-02/1.00e-02-0.00e+00}.png}
  \includegraphics[width=0.19\textwidth]{{figures/gammacomps0omeff/eq3.16e-02/1.78e-02-0.00e+00}.png}
  \includegraphics[width=0.19\textwidth]{{figures/gammacomps0omeff/eq3.16e-02/3.16e-02-0.00e+00}.png}
  \includegraphics[width=0.19\textwidth]{{figures/gammacomps0omeff/eq3.16e-02/5.62e-02-0.00e+00}.png}
  \includegraphics[width=0.19\textwidth]{{figures/gammacomps0omeff/eq3.16e-02/1.00e-01-0.00e+00}.png}
  \caption{ }
  \label{fig:gcomponents}
\end{figure*}


\subsubsection{Internal}
The situation for an internal resonance is similar:
if $\edisk\sim \ep$, then
$\abs{\gamma-\gamma_p}$ librates around 0.  If $\edisk>\ep$, then
$\abs{\gamma-\gamma_p}$ circulates.  If $\edisk< \ep$ (bottom right triangle of
Figure \ref{fig:align-internal}), the test particle is not captured
into resonance (i.e., $\bar{\theta}$ circulates) or equation
(\ref{eq:dota}) breaks down and the particle migrates in the opposite
direction. This is a limitation of our migration model and is not
necessarily physical. Our simulation results have been compiled into Figure
\ref{fig:align-internal}. For these simulations, $\mu_p$ was chosen
to be $\mu_p = 1.5 \times\mu_{\rm stable}$, where $\mu_{\rm stable}$
is given in the top line of equations (\ref{eq:stability}).

\subsubsection{Non-chaotic transition}
In Figure \ref{fig:align-external}, the boundary between the apsidally
aligned and anti-aligned regions is sharp up to a precision
of $2\times 10^{-3}$ in eccentricity space.

To test this, we performed simulations with initial conditions
matching the simulation in the center of the grid ($\ep = 0.032$,
$\edisk = 0.032$).  From there, we tested simulations moving down on
the grid (i.e., holding $\edisk$ constant and decreasing $\ep$) and
moving right on the grid  (i.e., holding $\ep$ constant and $\edisk$).
These preliminary results have determined the transition happens in
the following intervals:
\begin{enumerate}
\item For $\ep = 0.032$ constant, transition occurs within the interval
  $\edisk \in (0.054, 0.056)$.
\item For $\edisk = 0.032$ constant, transition occurs within the interval
  $\ep \in (0.017, 0.018)$.
\end{enumerate}
\subsubsection{(WIP) Comparable mass results}
\subsubsection{(WIP) N-body results} 


\section{Perturbed resonant behavior}
Note that we have not performed simulations
of an internal MMR perturbed by a giant companion.
Thus far, we have observed no disruption of MMR capture due to an
induced precession $\omext$.

\subsection{Circular case $e_p = 0$}
For disk conditions ranging between $$\edisk\in [0.01,0.1],$$ and
external precession frequencies
$$\omext \in [0.1,10]\times\delta n,$$ we have
found no change in resonance for $\mu_p = 10^{-4}$
whenever $e_p = 0$.

\subsection{Eccentric case $e_p > 0$}
%subsubsection
For $\omext < \delta n$, the resonance behavior is largely unchanged.

\begin{figure}[htb]
  \centering
  \includegraphics[width=0.3\textwidth]{../examples/ext-pert/newres-eq3.16e-02-ep1.00e-01-om1.00e-03.png}
  \caption{ }
  \label{fig:newres}
\end{figure}

However, for $\omext \sim \delta n$, a new behavior is observed:
$\gamma\to \pi$ and the resonance becomes anti-aligned. In Figure~\ref{fig:newres}, we have
plotted an integration for $\edisk=0.032$, $e_p=0.1$ to demonstrate
this behavior.  In addition to this, the average eccentricity $e_1$ is
excited to higher values.

For $\omext > \delta n$, depending on the relationship between
$\edisk$ and $\ep$, $e_1$ may be suppressed to small values, while
$\gamma\to \pi$ and the apsidal anti-alignment remains. For
$\ep>\edisk$, the values of $e_1$ decrease below $\edisk$ as $\omext$
increases. For $\ep \lesssim \edisk$, occasionally $e_1$ ``breaks''
out of the trend to return to $\sim \edisk$.

This can be understood from the equation of motion for $\gamma$ (see \ref{app:eom}),
where we have added the external precession frequency $\omext$ (see \ref{sec:extpert}).
In case of $\gamma$ circulating:
\[ -\mu_p\p*{\frac{A\cos\theta}{\Lambda e} + \frac{2C}{\Lambda} +
    \frac{De_p\cos\gamma}{\Lambda e} + \frac{\omega_{\rm eff}}{\mu_p}}
  > 0 \]
\[\Rightarrow \frac{1}{e\Lambda}\p*{A+De_p\abs{\left<\cos\gamma\right>}} > \frac{2C}{\Lambda} + \frac{\omega_{\rm eff}}{\mu_p} \]

In case of $\gamma$ locked:
\[ -\mu_p\p*{\frac{A\cos\theta}{\Lambda e} + \frac{2C}{\Lambda} +
    \frac{De_p\cos\gamma}{\Lambda e} + \frac{\omega_{\rm eff}}{\mu_p}}
  \approx 0 \]
\[\Rightarrow \frac{1}{e\Lambda}\p*{A+De_p\abs{\left<\cos\gamma\right>}} \approx \frac{2C}{\Lambda} + \frac{\omega_{\rm eff}}{\mu_p} \]
These results are summarized by the series of plots in Figure \ref{fig:gammadot-behaviors}.

\begin{figure*}[htb]
  \centering
 \includegraphics[width=0.3\textwidth]{../ext-perturber/varyomeff/gammadots-eq1.00e-02/total-gbehaviors.png}
 \includegraphics[width=0.3\textwidth]{../ext-perturber/varyomeff/gammadots-eq3.16e-02/total-gbehaviors.png}
 \includegraphics[width=0.3\textwidth]{../ext-perturber/varyomeff/gammadots-eq1.00e-01/total-gbehaviors.png}
  \caption{ }
  \label{fig:gammadot-behaviors}
\end{figure*}
\subsubsection{(WIP) Phase diagrams}
See Figure~\ref{fig:phasediags} for preliminary results.
\begin{figure*}[htb]
  \centering
 \includegraphics[width=0.45\textwidth]{{../examples/ext-pert/newres-phasediag-eq3.16e-02-ep1.00e-01-om1.00e-03}.png}
 \includegraphics[width=0.45\textwidth]{{../examples/ext-pert/newres-phasediag-eq3.16e-02-ep1.00e-01-om1.00e-03}.png}
  \caption{ }
  \label{fig:phasediags}
\end{figure*}

\subsubsection{(WIP) Comparable mass results}
WIP, code written, not working/tested yet.
\subsubsection{(WIP) N-body results} 
See Figure~\ref{fig:nbodynewres} for preliminary results.
\begin{figure*}[htb]
  \centering
 \includegraphics[width=0.45\textwidth]{{../nbody/testsuite/collect/precess-eq1.00e-02-ep1.00e-01-om1.00e-03}.png}
 \includegraphics[width=0.45\textwidth]{{../ext-perturber/varyomeff/eq1.00e-02/ep1.00e-01/1.00e-02-1.00e-03}.png}
  \caption{ }
  \label{fig:nbodynewres}
\end{figure*}

\clearpage
\onecolumn
\appendix
\section{Appendix}
\subsection{Resonant coefficients}\label{app:coeff}
The coefficients for an internal resonance are:
  \begin{align*}
    A &=  \frac12[2(j+1)+\alpha D]b_{1/2}^{(j+1)}(\alpha) \approx 2.0 \\
    B &= -\frac12[-1+2(j+1)+\alpha D]b_{1/2}^{(j)}(\alpha) \approx -2.5\\
    C &= \frac18[2\alpha D + \alpha^2 D^2]b_{1/2}^{(0)}(\alpha) \approx 1.15\\
    D &= \frac14[2-2\alpha D - \alpha^2 D^2]b_{1/2}^{(1)}(\alpha)\approx -2.0,
  \end{align*}
  where $b_{k}^{(j)}$ are the usual Laplace coefficients (e.g. Murray \& Dermott 2000).
For an external resonance:
\begin{align*}
    A &= \frac12\alpha[-1+2(j+1)+\alpha D]b_{1/2}^{(j)}(\alpha) \approx 1.9\\
    B &=  -\frac12\alpha[2(j+1)+\alpha D]b_{1/2}^{(j+1)}(\alpha) \approx -1.5 \\
    C &= \frac18\alpha[2\alpha D + \alpha^2 D^2]b_{1/2}^{(0)}(\alpha) \approx 0.9\\
    D &= \frac14\alpha[2-2\alpha D - \alpha^2 D^2]b_{1/2}^{(1)}(\alpha)\approx -1.5. 
  \end{align*}
\subsection{Equations of Motion}\label{app:eom}
The full equations of motion, derived from the Hamiltonian in equation \ref{eq:H},
for an internal resonance, are given by:
\begin{align*}
  H
  &= - \frac{1}{2\Lambda^2} + \mu_p\left(A
    \left(\frac{2\Gamma}{\Lambda}\right)^{1/2}\cos\theta + B
    e_p\cos\theta_p\right) \\
  &- \mu_p \left(C\left(\frac{2\Gamma}{\Lambda}
      + e_p^2\right) + De_p
    \sqrt{\frac{2\Gamma}{\Lambda}}\cos\gamma\right)
\end{align*}

\begin{align*}
\dot{\lambda}
  =& \frac{1}{\Lambda^3} + \mu_p\left(-A\sqrt{\frac{\Gamma}{2\Lambda^3}}\cos\theta
    + C\frac{2\Gamma}{\Lambda^2} + De_p\sqrt{\frac{\Gamma}{2\Lambda^3}}\cos\gamma\right) \\
\dot{\Lambda} =& -\mu_p \left(Aj\sqrt{\frac{2\Gamma}{\Lambda}}\sin\theta
      +Bj e_p\sin\theta_p\right) + \frac{\Lambda}{2}\left(\frac{1}{T_m}-\frac{4\Gamma}{\Lambda T_e}\right) \\
\dot{\gamma} =& \mu_p\left(\frac{A\cos\theta}{\sqrt{2\Gamma\Lambda}}
    - \frac{2C}{\Lambda} - De_p\frac{\cos\gamma}{\sqrt{2\Gamma\Lambda}} \right) \\
  \dot{\Gamma} = &\mu_p\left(A\sqrt{\frac{2\Gamma}{\Lambda}}\sin\theta
    - De_p\sqrt{\frac{2\Gamma}{\Lambda}}\sin \gamma\right)  \\
&-\frac{\Gamma}{\Lambda}\frac{\Lambda}{2}\left(\frac{1}{T_m}-\frac{4\Gamma}{\Lambda T_e}\right)
  -\frac{2\Gamma}{T_e}
\end{align*}
For an external resonance, are given by:
\begin{align*}
  H
  &= - \frac{1}{2\Lambda^2} - \mu_p\left(A
    \left(\frac{2\Gamma}{\Lambda}\right)^{1/2}\cos\theta + B
    e_p\cos\theta_p\right) \\
  &- \mu_p \left(C\left(\frac{2\Gamma}{\Lambda}
      + e_p^2\right) + De_p
    \sqrt{\frac{2\Gamma}{\Lambda}}\cos\gamma\right)
\end{align*}

\begin{align*}
\dot{\lambda}
  =& \frac{1}{\Lambda^3} + \mu_p\left(A\sqrt{\frac{\Gamma}{2\Lambda^3}}\cos\theta
    + C\frac{2\Gamma}{\Lambda^2} + De_p\sqrt{\frac{\Gamma}{2\Lambda^3}}\cos\gamma\right)\\
\dot{\Lambda} =& \mu_p \left(Aj\sqrt{\frac{2\Gamma}{\Lambda}}\sin\theta
      Bj e_p\sin\theta_p\right) + \frac{\Lambda}{2}\left(-\frac{1}{T_m}-\frac{4\Gamma}{\Lambda T_e}\right) \\
\dot{\gamma} =& \mu_p\left(\frac{-A\cos\theta}{\sqrt{2\Gamma\Lambda}}
    - \frac{2C}{\Lambda} - De_p\frac{\cos\gamma}{\sqrt{2\Gamma\Lambda}} \right)\\
  \dot{\Gamma} =& \mu_p\left(-A\sqrt{\frac{2\Gamma}{\Lambda}}\sin\theta
    - De_p\sqrt{\frac{2\Gamma}{\Lambda}}\sin \gamma\right) \\
&-\frac{\Gamma}{\Lambda}\frac{\Lambda}{2}\left(-\frac{1}{T_m}-\frac{4\Gamma}{\Lambda T_e}\right)
  -\frac{2\Gamma}{T_e}
\end{align*}


\subsection{Reduced Hamiltonian}\label{app:reducedH}
The reduced Hamiltonian for an internal resonance with $\ep>0$ is given by:
  \[ \overline{\Gamma} = \Gamma + \frac{\sqrt{\Lambda_0}B}{A} e_p\sqrt{2\Gamma}\cos\gamma + \frac{\Lambda_0B^2}{A^2}e_p^2 \]
\[ \overline{\gamma} = \tan^{-1}\p*{\frac{e\sin\gamma}{e\cos\gamma + Be_p/A}} \]
\[ \overline{e}^2 = e^2 + \frac{2B}{A} e_p e \cos\gamma + \frac{B^2}{A^2}e_p^2\]
  \[\overline{\theta} = (j+1)\lambda_p - j\lambda + \overline{\gamma} \]
\[ \overline{H} = - \frac{(GM)^2}{2\Lambda^2} + \frac{Gm_p}{a_p}A\sqrt{\frac{2\overline{\Gamma}}{\Lambda}}\cos\overline{\theta}, \]
and in the external case:
  \[ \overline{\Gamma} = \Gamma + \frac{\sqrt{\Lambda_0}B}{A} e_p\sqrt{2\Gamma}\cos\gamma + \frac{\Lambda_0B^2}{A^2}e_p^2 \]
\[ \overline{\gamma} = \tan^{-1}\p*{\frac{e\sin\gamma}{e\cos\gamma + Be_p/A}} \]
\[ \overline{e}^2 = e^2 + \frac{2B}{A} e_p e \cos\gamma + \frac{B^2}{A^2}e_p^2\]
  \[\overline{\theta} = (j+1)\lambda - j\lambda_p + \overline{\gamma} \]
\[ \overline{H} = - \frac{(GM)^2}{2\Lambda^2} - \frac{Gm_p}{a_p}\alpha A\sqrt{\frac{2\overline{\Gamma}}{\Lambda}}\cos\overline{\theta} \]
%section Extra
\extra{
For the $\edisk\lesssim \ep$ regime in the $(\ep,\edisk)$ parameter space,
$\dot{\gamma}\to 0$ and $\abs{\gamma-\gamma_p}\to 0$, resulting in
apsidal alignment. In the opposite regime, $\dot{\gamma}\neq 0$ and
$\abs{\gamma - \gamma_p}$ circulates.
For $\edisk \sim \ep$, the dynamics are complicated, and we turn to numerical
simulations.

  \begin{equation}
  \label{eq:gammadot}
\dot{\gamma} = \mu_p\left(\pm\frac{A\cos\theta}{\sqrt{2\Gamma\Lambda}}
    - \frac{2C}{\Lambda} -
    D\ep\frac{\cos\gamma}{\sqrt{2\Gamma\Lambda}}\right).
\end{equation}
Since $\theta$ circulates, the first term typically averages out to zero,
and the $(\pm)$ does not matter. If $\dot{\gamma}\approx 0$,
then
\[ \ep \approx 2\abs*{\frac{C}{D}}\edisk\approx \edisk \]
for both the internal and external resonances.
Hence, if $\ep\lesssim \edisk$, the middle term in equation (\ref{eq:gammadot})
dominates and $\dot\gamma$ may go to 0, leading to apsidal alignment.
If $\ep\gtrsim \edisk$, the last term in equation (\ref{eq:gammadot})
dominates, and $\dot\gamma \neq 0$ destroying any apsidal alignment.}
\end{document}