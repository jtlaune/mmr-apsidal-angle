\documentclass[12pt]{article}
\usepackage{geometry}
\usepackage{enumitem}
\usepackage{graphicx}
\usepackage{amsmath}
\usepackage{mathtools}
\usepackage{xcolor}
\usepackage{cancel}
\usepackage{titlesec}
\newcommand{\note}[1]{{\color{red} #1 }}
\newcommand{\fignote}[1]{{\color{blue} #1 }}
\renewcommand{\v}[1]{\boldsymbol{ #1 }}
\renewcommand{\bar}[1]{\overline{ #1 }}
\newcommand{\vh}[1]{\hat{\boldsymbol{ #1 }}}
\newcommand{\dd}[2]{\frac{\partial #1}{\partial #2}}
\newcommand{\cE}{\mathcal{E}}
\newcommand{\ebar}{\overline{e}}
\newcommand{\gammabar}{\overline{\gamma}}
\newcommand{\thetabar}{\overline{\theta}}
\newcommand{\edisk}{e_d}
\DeclarePairedDelimiter{\abs}{|}{|}
\DeclarePairedDelimiter{\p}{(}{)}
\DeclarePairedDelimiter{\we}{\langle}{\rangle}

\title{Apsidal alignment in exoplanet systems}

\begin{document}
\section{Introduction}
\begin{itemize}
\item Migration and MMR capture in disks
\item Observations of Kepler-88, K2-19 show apsidal alignment
\item What conditions are necessary for alignment?
\item \textbf{Goal:} Constrain disk aspect ratio with observations of aligned systems
\end{itemize}
\section{Restricted problem}
%subsection
\begin{itemize}
\item Background formalism, Wisdom's Hamiltonian
\item Summarize shifted resonance and resonant variables ($\gammabar$,
  $\ebar$, $\thetabar$)
\end{itemize}
\subsection{Disk forces}
\begin{itemize}
\item Dissipative forces from protoplanetary disk conditions
\item Connect parameters $T_m$ and $T_e$ to disk aspect ratio and density
\item Equilibrium eccentricity $\edisk$ for $e_p=0$
\end{itemize}
\subsection{External MMR}
\begin{itemize}
\item Heuristic explanation with equation of state
\item \fignote{Plot of gamma components}
\end{itemize}
\includegraphics[width=0.6\textwidth]{../../images/external-examples.png}\\
\includegraphics[width=0.9\textwidth]{../../images/galign-grid.png}\\
\includegraphics[width=0.6\textwidth]{../../images/external-examples-phasediags.png}
\subsection{Internal MMR}
\begin{itemize}
\item Similar to external
\item Migration model fails for $e_p<\edisk$
\end{itemize}
\section{Two massive planets}
\begin{itemize}
\item \note{Work in progress}
\end{itemize}
\section{Observations}
\begin{itemize}
\item K2-19 b,c have $\delta\varpi\approx 0$, $e_b,e_c\approx 0.2$,
  but $\theta_b$ and $\theta_c$ both circulate
\end{itemize}
\section{Conclusion}
\end{document}