% Created 2022-03-01 Tue 02:10
% Intended LaTeX compiler: pdflatex
\documentclass[usenatbib,twocolumn]{mnras}
\usepackage[utf8]{inputenc}
\usepackage[T1]{fontenc}
\usepackage{graphicx}
\usepackage{longtable}
\usepackage{wrapfig}
\usepackage{rotating}
\usepackage[normalem]{ulem}
\usepackage{amsmath}
\usepackage{amssymb}
\usepackage{capt-of}
\usepackage{hyperref}
\usepackage{caption}
\usepackage{tabularx}
\usepackage{subcaption}
\usepackage{pdfpages}
\usepackage{float}
\usepackage{booktabs}
\usepackage{enumitem}
\usepackage{graphicx}
\usepackage{tensor}
\usepackage{ wasysym }
\usepackage{mathtools}
\usepackage{xcolor}
\usepackage{cancel}
%\newcommand{\note}[1]{{\color{red} \large #1 }}
\newcommand{\note}[1]{}
\renewcommand{\O}{\mathcal{O}}
\renewcommand{\d}{\partial}
\renewcommand{\v}[1]{\boldsymbol{ #1 }}
\renewcommand{\t}[1]{\tilde{ #1 }}
\newcommand{\tg}{\t{g}}
\newcommand{\vh}[1]{\hat{\boldsymbol{ #1 }}}
\newcommand{\pp}[2]{\frac{\partial #1}{\partial #2}}
\newcommand{\dd}[2]{\frac{d #1}{d #2}}
\DeclarePairedDelimiter{\abs}{|}{|}
\DeclarePairedDelimiter{\norm}{||}{||}
\DeclarePairedDelimiter{\p}{(}{)}
\DeclarePairedDelimiter{\we}{\langle}{\rangle}
\title[MMR Architecture]{Architecture of Planetary Systems in Mean Motion Resonance}
\author[Laune et al.]{
JT Laune,$^{1}$
Laetitia Rodet,$^{1}$
and Dong Lai$^{1}$
\\
$^{1}$Department of Astronomy and Space Sciences, Cornell University\\}
\date{}
\title{Apsidal Architecture of Planetary Systems in Mean Motion Resonance}
\hypersetup{
 pdfauthor={JT Laune},
 pdftitle={Apsidal Architecture of Planetary Systems in Mean Motion Resonance},
 pdfkeywords={},
 pdfsubject={},
 pdfcreator={Emacs 27.2 (Org mode 9.5.2)}, 
 pdflang={English}}
\begin{document}

\maketitle
\begin{abstract} Long before the first exoplanet detection,
theoretical works predicted that young planets migrate across their
natal disk and sweep up other planets into mean motion resonances,
which operate whenever the planets' period ratios are related by an
integer fraction $j:j+k$.  In the literature, the typical treatment of
resonance capture parametrizes the planet's migration timescale and
damps all eccentricities to zero.  Post-capture, the pair evolves into
a configuration in which both planets' resonance angles librate and
their perihelia are anti-aligned.  However, there are some
observations which the standard picture cannot explain. The system
K2-19 hosts two planets close to, but just wide of, the 3:2 MMR which
are moderately eccentric and apsidally aligned.  In this paper, we
investigate the MMR Hamiltonian under the influence of external disk
forces which facilitate migration but drive the eccentricites of the
system to a finite, nonzero saturation value.  We find that, for
certain saturation eccentricities, apsidal alignment is a natural
outcome if the external driving force overpowers the resonance.  In
this configuration, both planets' resonance angles circulate, but
there is a combined resonance angle which still librates and traps the
planets near the nominal resonance location.  Conversely, we find that
eccentricity forcing may also disrupt resonance capture or destroy an
existing resonance.
\end{abstract}

\section{Introduction}
\label{sec:orgc42ad6c}
Even before the first detection of an exoplanetary system, planets
were predicted to migrate from their initial position due to
interactions with their natal protoplanetary disk
\citep[PPD, ][]{lin79_tidal_torques_accret_discs_binar,goldreich_excitation_1979,goldreich_disk-satellite_1980-1,moutamid14_coupl_between_corot_lindb_reson}.
The associated angular momentum and energy exchange acts to reshape
the orbital architecture during the first million years of a system's
life, in particular the planets' semimajor axes and eccentricities.
This process is collectively known as planetary migration
\citep{nelson_planetary_2018}.  Migration can move a planet toward or
away from its star and is heavily dependent on the disk profile,
particularly its aspect ratio and density profile.  During migration,
the disk typically circularizes the planets' orbits while it shrinks
their semimajor axes.  More recent theoretical works, however, have
proposed several mechanisms by which the PPD can drive the planets
eccentricities to a finite saturation value rather than damping them
to zero.

The authors in \cite{goldreich03_eccen_evolut_planet_gaseous_disks}
demonstrated how Lindblad and corotation resonances compete to either
drive or damp a giant planet's eccentricity.  More recently,
\citet{teyssandier17_secul_evolut_eccen_protop_discs} and
\citet{ragusa17_eccen_evolut_durin_planet_disc_inter} found that, in
long hydrodynamic planet-disk simulations, a large gap-opening
planet's eccentricity exhibits contrasting growth and decay as the
system evolves.  In
\citet{romero21_eccen_drivin_pebbl_accret_low_mass_planet}, the authors
found that luminous, super-earth mass protoplanets can be driven to
eccentricities beyond the disk's aspect ratio through a thermal
back-reaction with the disk gas density.

As a planet migrates through its natal disk, its relative position
amongst the other (proto-)planets constantly changes.  When two
planets' periods are related by an integer fraction \(j\):\(j+k\)
(e.g. 1:3, 2:3), mutual gravitational interactions can lock their
periods at that fraction in what is known as a mean motion resonance
(MMR).  If planets are migrating at differential rates in the same
direction, they will cross these period ratios, where they may be
captured into resonance.  This process is collectively known as
resonant capture.  MMRs are characterized by the libration of two
angles specific to the values of \(j\) and \(k\).  If \(k=1\),
i.e. \(j\):\(j+1\) for some \(j\), the gravitational interactions are first
order in eccentricity and inclination.  The resonance angles for first
order MMRs are given by
\begin{align}
\label{circangles1}
 \theta_1 &= (j+1)\lambda_2 - j\lambda_1 - \varpi_1 \\
\label{circangles2}
 \theta_2 &= (j+1)\lambda_2 - j\lambda_1 - \varpi_2.
\end{align}

\noindent The mean longitudes of the inner and outer planets
are denoted by \(\lambda_1\) and \(\lambda_2\), respectively, and
\(\varpi_1\) and \(\varpi_2\) denote their longitudes of perihelion.  If
the angles \(\theta_1\) and \(\theta_2\) librate around \(180^\circ\) and
\(0^\circ\), respectively, the system is said to be in resonance.  Since
these angles are related by \(\theta_2-\theta_1=\varpi_1-\varpi_2\),
this implies that a resonant pair of planets will have
\(\Delta\varpi\equiv \varpi_1-\varpi_2\approx 180^\circ\).  As a planet
migrates, its resonance locations sweep across the disk and capture
planets into resonant pairs and chains.  The MMR equilibrium is
governed by various conditions which are well studied in the
literature
\citep{henrard_second_1983,deck_migration_2015,goldreich_overstable_2014,xu_migration_2018}.
The resonant pair's path across the disk, characterized by the
direction, distance traveled, and migration and eccentricity damping
timescales, determines the final orbital architecture for the system
\citep{cresswell_evolution_2006,cresswell_three-dimensional_2008}.

The classical MMR capture scenario envisions two coplanar, nearly
circular (\(e\ll 1\)) planets on orbits with periods near the nominal
resonance location, \((j+1):j\). For small enough eccentricity and slow
enough migration, capture occurs as the planets cross the resonance
during convergent migration.  After capture, the system evolves into
an equilibrium with finite eccentricities which are determined by the
particular migration timescales.  This leads to pairs and then chains
of planets in resonance.  However, the effects of moderate
eccentricities on the equilibrium point of the capture problem have
not been thoroughly examined yet.

In the past 40 years, exoplanet surveys have given us a wealth of
systems to test our understanding of disk migration and resonant
dynamics.  Thousands of exoplanet systems have been discovered, and,
although many resonant systems have been found, the proportion of such
systems is low compared to theoretical expectations based on disk
migration \citep{fabrycky_architecture_2014}.  In addition to the paucity
of planets in MMRs, there also seems to be an overabundance of planets
just outside of mean motion resonances, with period ratios slightly
larger than 2 or 3/2
\citep{fabrycky_architecture_2014,choksi_sub-neptune_2020}.  A system's
orbital architecture can reveal details about its history.
Commensurate period ratios may indicate past resonant interactions
within the planetary system, but the present state of the resonance
can only be determined by analyzing the behavior of the angles given by
equations \eqref{circangles1} and \eqref{circangles2}.  

NASA's \emph{Kepler} mission observed several planet pairs near a
first-order resonance with constraints on \(\Delta\varpi\), such as
Kepler-88 \citep{weiss_discovery_2020} and Kepler-24
\citep{antoniadou_exploiting_2020}. Both of these systems are apsidally
anti-aligned (\(\Delta\varpi\approx180^\circ\)).  Kepler-9b and
Kepler-9c are near to the 2:1 resonace and apsidally anti-aligned, but
their angles \(\theta_1\) and \(\theta_2\) likely circulate.
Meanwhile, a
puzzling resonant architecture has been discovered recently
by NASA's \emph{K2} mission in the
three-planet system K2-19.

The planets K2-19b and c are near a 3:2 period ratio (\(P_b=7.9\) d,
\(P_c=11.9\) d), and planet K2-19d lies on an orbit interior to the
other two at \(P_d=2.5\) d
\citep{howell14_k2_mission,armstrong15_one_closes_exopl_pairs_to,sinukoff16_eleven_multip_system_fromk_masses}.
The system's photometry data can be reproduced by setting the
innermost planet's eccentricity to zero while planets b and c orbit
the primary with moderate eccentricity, \(e_b\approx e_c \approx 0.2\).
K2-19 is a solar-type star
(\(M=0.88M_\odot\)), planet b has \(M_{b}=10.8 M_{\oplus}\), and c has
\(M_{c}<10M_{\oplus}\).  Further observations revealed K2-19b and
c to be apsidally \emph{aligned}, with \(\Delta\varpi\equiv
\varpi_c-\varpi_b \approx 2\pm 2^\circ\) \citep{petigura_k2-19b_2020}.
Hence, the K2-19 system poses a problem for the conventional
understanding of planet migration and resonance capture.

Investigating how K2-19 could have formed with \(\Delta\varpi=0^\circ\)
through resonance capture and mutual migration will offer us an
insight into its dynamical history as well as a better understanding
of the genesis of extrasolar orbital configurations in general.  In
this paper, we review the analytically simple migration model commonly
used in the literature.  In Section \ref{sec:org16ae390}, we present the classical picture of resonant capture and
explore the parameter space for the coupling between the resonant
eccentricities and the protoplanetary disk. We fail to find any disk
conditions which robustly lead to apsidal alignment.  In Section \ref{sec:org585941d}, we explore apsidal alignment for a test particle in
the vicinity of an eccentric planet's MMR. These results guide our
analysis in Section \ref{sec:org34f950f}, where we design a toy model which
relies on eccentricity driving to a finite, non-zero value and argue
that it can reproduce the apsidal alignment between K2-19b and c.
Such an external force which drives the eccentricities beyond their
equilibrium ratio \(e_2/e_1 = m_1/m_2\) can
break the MMR system out of the \(\theta_1\) and \(\theta_2\) resonances.
However, through a series of canonical rotations, we demonstrate that
there is a third resonance angle, \(\hat\theta\), which may
librate.  If so, this resonance still traps the system near the nominal
period ratio \(P_2/P_1=(j+1)/j\) and, in this configuration, naturally
leads to apsidal alignment.

\section{Standard Picture of Resonance Capture}
\label{sec:org16ae390}
\begin{table*}[htb]
\caption{\label{tab:ICs}Initial conditions for the finite mass ratio integrations in this paper. The quantity \(\epsilon\) is the tolerance used in the numerical integrator.}
\centering
\begin{tabular}{| l | c | c | c |}
\hline
 & Section 2 & Section 3 & Section 4\\
\hline
\(\mu_{\rm tot}\) & \(10^{-4}\) & \(10^{-4}\) & \(10^{-4}\)\\
\hline
\(P_2/P_1 (t=0)\) & 1.6 & 1.6 & 1.5\\
\hline
\(e_1\), \(e_2 (t=0)\) & \(0.001\) & \(0.001\) & \(0.001\)\\
\hline
\(\theta_1\), \(\theta_2 (t=0)\) & \(\mathcal{U}[0^\circ,360^\circ]\) & \(\mathcal{U}[0^\circ,360^\circ]\) & \(\mathcal{U}[0^\circ,360^\circ]\)\\
\hline
\(e_{1d}\), \(e_{2d}\) & \(0\) & \(0\) & \(0, 0.1, 0.2\)\\
\hline
\(h\) & 0.03 & 0.03 & 0.03\\
\hline
\(T_{e,0}\) [yr] & 1000 & 1000 & 1000\\
\hline
\(q=m_1/m_2\) & \([0.5,1)\cup(1,2]\) & \(0,\infty\) & \([0.5,1)\cup(1,2]\)\\
\hline
\(T_{e,1}/T_{e,2}\) & \(1/q\), free parameter & \(\infty,0\) & \(1/q\)\\
\hline
\(T_{e,i}/T_{m,i}\) & \(3.46h^2\) & \(3.46h^2\) & \(3.46h^2\)\\
\hline
\(T\) & \(150\times\max(T_{e,1},T_{e,2})\) & \(150\times\max(T_{e,1},T_{e,2})\) & \(60\times\max(T_{e,1},T_{e,2})\)\\
\hline
\(\epsilon\) & \(10^{-9}\) & \(10^{-9}\) & \(10^{-9}\)\\
\hline
\end{tabular}
\end{table*}
Planets embedded within a protoplanetary disk interact gravitationally
with the gas and lose angular momentum, leading to inward migration
towards the central star.  Disk torques vary with planet mass as well
as across semimajor axis; large outer planets lose angular momentum
quickly and sweep up inner planets into MMRs
\citep{tanaka_three-dimensional_2004,xu_migration_2018}.  In some
cases, the planet can gain angular momentum and migrate away from the
primary.

In this paper, we will ignore the detailed physics of planet-disk
interactions and instead implement a proxy for dissipative forces
parametrized by the eccentricity damping and migration timescales,
denoted by \(T_{e,i}\) and \(T_{m,i}\) for \(i=1,2\).  We will denote all
quantities relevant to the inner planet with the subscript \(1\), and the
outer with \(2\).  The equations of motion for disk effects are:
\begin{align}\label{eq:disforce}
  \frac{\dot{e}_i}{e_i} &= -\frac{1}{T_{e,i}} \\
\label{eq:disforce1}
  \frac{\dot{a}_i}{a_i} &= -\frac{1}{T_{m,i}} -\frac{2e_i^2}{T_{e,i}}.
\end{align}

\noindent This approximate migration model has been proposed
by \citet{goldreich_disk-satellite_1980-1} and is used in most studies
of MMR capture
\citep[e.g.][]{goldreich_overstable_2014,xu_migration_2018} In our
notation, \(T_{m,i}>0\) \((<0)\) denotes inward (outward) migration.

We will consider two planets with masses \(m_1\) and \(m_2\) around a star
of mass \(M\). We define \(\mu_i=m_i/M\) to be their mass fractions
and set \(M=1M_\odot\) throughout. For typical, thin disk profiles, we
have
\citep{tanaka_three-dimensional_2004,cresswell_three-dimensional_2008,xu_migration_2018}
\begin{align}
  \label{eq:Teratio}
  \frac{T_{e,1}}{T_{e,2}}&= \frac1q\\
  \label{eq:Tmratio}
  T_{e,i}&=3.46 h^2 \abs{T_{m,i}},
\end{align}

\noindent where \(q=m_1/m_2\) is the mass ratio and \(h\) is the
aspect ratio of the disk.
The numerical prefactor in equation \eqref{eq:Tmratio} is a parameter fit 
by hydrodynamic simulation data \citep[for details see][]{cresswell_three-dimensional_2008}.
To scale the dissipation times in the
integrations, we choose a parameter \(T_{e,0}\) and set
\begin{align}
  T_{e,1}&=T_{e,0}/\sqrt{q}\\
  T_{e,2}&= T_{e,0}\sqrt{q}.
\end{align}

\noindent We must have \(1/T_{m,1} - 1/T_{m,2} < 0\) for
convergent (i.e. \(\abs{a_1-a_2}\) is shrinking) inward migration, and
vice versa for outward migration.  Hence, for \(q>1\), we set \(T_{m,i}<
0\); for \(q<1\), we set \(T_{m,i}>0\).  For all of the runs in this paper,
we choose \(h=0.03\) and \(T_{e,0}=1000~\rm{years}\).
\subsection{Resonant Hamiltonian}
\label{sec:orgbd7056f}
\begin{figure*}
  \centering
  \includegraphics[width=0.7\textwidth]{{./standard-example-h-0.03-Tw0-1000}.png}
  \caption{Standard MMR capture process for $h=0.03$ and
    $q=2$. The inner planet starts at $a_1=1$ au and the outer
    planet starts wide of resonance at $P_2/P_1=1.6$.  Both
    planets start with very small eccentricities,
    $e_1=e_2=0.001$. The planets are captured into resonance
    near $t=20,000$ yrs, indicated by the libration of
    $\theta_1\to180^\circ$ and $\theta_2\to 0^\circ$ and the
    period ratio approaching 1.5.  While in resonance, the $e_i$
    values are driven to equilibrium, with $e_1\approx 0.008$
    and $e_2\approx 0.016$, and the periapses are anti-aligned.}
  \label{fig:standardex}
\end{figure*}
When two planets have commensurate period ratios, \(j\):\(j+k\) where
\(j,k\) are integers, their gravitational interactions may lock them
into a mean motion resonance.  As young planets migrate within
their disk, if the migration is convergent, they cross MMR period
ratios and may be captured. In our paper, we will be considering only
first order MMRs, which occur when \(n_2/n_1 = (j+1)/j\), where \(n_1,n_2\)
are the planets' mean motions.

The Hamiltonian of a system with two planets orbiting
a primary of mass \(M\) is
\begin{align}
  H = -\frac{GMm_1}{r_1} - \frac{GMm_2}{r_2} - \frac{Gm_1m_2}{\abs{\mathbf{r}_1-\mathbf{r}_2}},
\end{align}

\noindent where \(m_1\), \(m_2\) are the masses of the planets
and \(\mathbf{r}_1\), \(\mathbf{r}_2\) their position vectors with respect
to \(M\).  The third term is known as the ``disturbing function'', and can
be expanded into cosine terms with angles and amplitudes
involving the orbital elements of the two planets.
For details of this derivation, see Chapter 6 of
\citet{murray_solar_2000}.
If the planets are
coplanar, have \(m_1\), \(m_2\ll M\), and have periods near a first order
MMR, the Hamiltonian can be aproximated to order \(\mathcal{O}(e^2)\) as
\citep{murray_solar_2000}:
\begin{align}
  H_{\rm kep} = & -\frac{G M m_{1}}{2 a_{1}}-\frac{G M m_{2}}{2 a_{2}}\\
  H_{\rm res} = & -\frac{G m_{1} m_{2}}{a_{2}}
                  \left[
                  f_{1} e_{1} \cos \theta_{1} 
                  +f_{2} e_{2} \cos \theta_{2}\right]\\
  H_{\rm sec} = &-\frac{G m_{1} m_{2}}{a_{2}}\left[f_{3} (e_1^2 + e_2^2)
                  +f_4e_1e_2\cos(\varpi_2-\varpi_1)
                  \right] \\
\label{hamiltonian}
  H = &~ H_{\rm kep} + H_{\rm res}+ H_{\rm sec},
\end{align}

\noindent
where \(\theta_1\) and \(\theta_2\) are given in equations
\eqref{circangles1} and \eqref{circangles2}.
Here, the \(f_i\) are functions of the semimajor
axis ratio \(\alpha=a_1/a_2\) that can be found in Appendix B of
\citet{murray_solar_2000} as
\begin{align}
\label{coefficients}
  f_1 &= \frac12[2(j+1)+\alpha D]b_{1/2}^{(j+1)}(\alpha); f_1(\alpha_{2:3})\approx 2.0 \\
  f_2 &= -\frac12[-1+2(j+1)+\alpha D]b_{1/2}^{(j)}(\alpha);f_2(\alpha_{2:3}) \approx -2.5\\
  f_3 &= \frac18[2\alpha D + \alpha^2 D^2]b_{1/2}^{(0)}(\alpha); f_3(\alpha_{2:3})\approx 1.15\\
  f_4 &= \frac14[2-2\alpha D - \alpha^2 D^2]b_{1/2}^{(1)}(\alpha); f_4(\alpha_{2:3})\approx -2.0,
\end{align}

\noindent where \(D\) denotes the derivative operator with
respect to \(\alpha\) and the \(b_{l}^m\) are Laplace coefficients.  In
our numerical integrations, we evaluate them at the instantaneous
semimajor axis ratio.  The coefficients \(f_1\) and \(f_2\) depend weakly
on \(\alpha\) around the resonant value, so we may ignore them in our
analytical treatment. \(H_{\rm kep}\) is the standard Keplerian
Hamiltonian; \(H_{\rm res}\) and \(H_{\rm sec}\) are the resonant and secular
Hamiltonians, respectively.

The Hamiltonian system defined by equation \eqref{hamiltonian} admits
seven independent coupled ordinary differential equations, which we
may integrate together with the effects of dissipation (equations \eqref{eq:disforce}
and \eqref{eq:disforce1}) to simulate MMR
capture.  The equations are obtained by scaling and substituting the
following Poincair\'e momenta into equation
\eqref{hamiltonian}
\begin{align}
  \Lambda_i &= m_i\sqrt{GMa_i}\\
  \Gamma_i &= \Lambda_i(1-\sqrt{1-e_i^2}),
\end{align}

\noindent where \(\Lambda_i\) is conjugate to \(\lambda_i\) and
\(\Gamma_i\) to \(-\varpi_i\).  Then we apply Hamilton's equations to
generate the equations of motion and combine \(\lambda_1\) and
\(\lambda_2\) into the angles \(\theta_1\) and \(\theta_2\).  We utilize the
\(\mathtt{scipy}\) Python package with the Runge-Kutta method of order
5(4) and a relative and absolute tolerance \(\epsilon=10^{-9}\).  The
resonance angles are initialized over a uniform distribution between
\(0^\circ\) and \(360^\circ\).  At \(t=0\), we set \(a_1=1~\mathrm{au}\),
\(P_2/P_1=1.6\), and \(e_1=e_2=0.001\).  Table \ref{tab:ICs} summarizes our
simulation parameters for this section and the next two.

An example of MMR capture is given in Figure \ref{fig:standardex}.  The
period ratio \(P_2/P_1\) initially starts wide of the nominal resonance
value.  After around \(2~\rm{kyr}\) of convergent migration, the planets
are caught into MMR, indicated by the stabilization of \(\theta_1\) to
\(180^\circ\) and \(\theta_2\) to \(0^\circ\).  The planets' eccentricities
level off at their equilibrium values near \(e_1\approx 0.008\) and
\(e_2\approx0.016\), and the planets become apsidally anti-aligned with
\(\varpi_1-\varpi_2\approx 180^\circ\).  In this paper, we will use the
term ``resonance'' to mean the libration of an angle such as \(\theta_1\),
\(\theta_2\), and later on \(\hat\theta\).  We will also use the angle
itself to refer to the resonance, i.e. the planets \(m_1\) and \(m_2\) in
Figure \ref{fig:standardex} are caught into both \(\theta_1\) and
\(\theta_2\), respectively, since those angles are librating.

\subsection{Equilibrium}
\label{sec:org6125b8c}
\begin{figure}
  \centering
  \begin{subfigure}[t]{0.225\textwidth}
  \includegraphics[width=1\textwidth]{{standard-eeqs-Tm2--454146-Tw0-1000}.png}
  \caption{ }
  \label{fig:standardeqecc}
  \end{subfigure}
  \begin{subfigure}[t]{0.225\textwidth}
  \includegraphics[width=1\textwidth]{{standard-pomega-Tm2--454146-Tw0-1000}.png}
  \caption{ }
  \label{fig:standardDpom}
  \end{subfigure}
  \caption{\emph{(a)} The analytical solutions to equations
    (\ref{dote1}) -- (\ref{dotdpom}) and (\ref{doteta}) are plotted as
    dashed lines for various values of $q$.  The initial conditions
    are the same as in Figure \ref{fig:standardex}.  We hold $h=0.03$
    and $T_{e,0}=1000$ yrs constant, but allow the migration
    timescales to vary with $q$, as in equations (\ref{eq:Teratio}) and
    (\ref{eq:Tmratio}).  The points indicate results from integrating
    the time-dependent equations of motion and time-averaging the
    eccentricities over the last 10\% of the integration.  The error
    bars correspond to the standard deviation of the eccentricities,
    but most fall within the marker for eccentricity.  \emph{(b)} Same
    as \emph{(a)}, but for $\Delta\varpi$.}
\label{fig:standard}
\end{figure}
The MMR capture in Figure \ref{fig:standardex} leads to an equilibrium
state in period ratio, resonant angles, eccentricities, and
\(\Delta\varpi\).  The Hamiltonian in equation
\eqref{hamiltonian}, including the dissipative terms, admits the
following three equilibrium equations for
\(e_1\), \(e_2\),and \(\Delta\varpi\):
\begin{equation}
\label{dote1}
  \dot e_1 = \frac{\mu_2}{\alpha_2} [f_1\sin(\theta_1) - De_2 \sin(\varpi_1-\varpi_2)] - \frac{e_1}{T_{e,1}}=0
\end{equation}

\begin{equation}
\label{dote2}
  \dot e_2 = \frac{q\mu_2}{\alpha_2} [f_2\sin(\theta_2) - De_1 \sin(\varpi_2-\varpi_1)]- \frac{e_2}{T_{e,2}}=0
\end{equation}

\begin{align}
\label{dotdpom}
  \frac{d}{dt}\Delta\varpi = \dot\varpi_1-\dot\varpi_2
  &= \frac{\mu_2}{\alpha_2} \left[ \frac{f_1\cos\theta_1}{\alpha_1^{1/2} e_1}
     - \frac{qf_2\cos\theta_2}{\alpha_2^{1/2}e_2}\right.\nonumber \\
  &\quad+ \left.\frac{2C}{\alpha_1^{1/2}} + \frac{De_2}{\alpha_1^{1/2} e_1}
    - \frac{2qC}{\alpha_2^{1/2}} - \frac{qDe_1}{ \alpha_2^{1/2}e_2}\right]=0.
\end{align}

\noindent To first order in eccentricity, the first two
equations determine the equilibrium values of \(\theta_1\) and
\(\theta_2\), while the last implies that \(e_2/e_1 \sim q\).

Moreover, absent any dissipative or secular forces, the following
quantities are strictly conserved \citep{xu_migration_2018}:
\begin{align}
  J &= \Lambda_1\sqrt{1-e_1^2} + \Lambda_2\sqrt{1-e_2^2}\\
  G &= \frac{j+1}{j} \Lambda_1 + \Lambda_2,
\end{align}

\noindent where \(\Lambda_1 = (m_1/m_{\rm
tot})\sqrt{a_1/a_0}\) and \(\Lambda_2=(m_2/m_{\rm tot})\sqrt{a_2/a_0}\).
Here we have scaled the overall Hamiltonian by the quantity
\(GMm_{\rm tot}/a_0\) with \(m_{\rm tot}=m_1+m_2\) and \(M\) the mass of
the primary. The quantity \(J\) is the angular
momentum of the system, and \(G\) is an integral of motion for the the
Hamiltonian in equation \eqref{hamiltonian}.
Next, we define \(\eta\) to be the following function of \(\alpha\) and \(J/G\),
\begin{align}
  \eta(\alpha, e_1, e_2) &\equiv - 2(q/\alpha_0+1)\p*{\frac{J}{G}-\left.\frac{J}{G}\right|_{0}},
\end{align}

\noindent
where \(\alpha_0 = (j/(j+1))^{3/2}\) and \(\left.(J/G)\right|_{0}\) is
evaluated at \(e_i=0\) and \(\alpha=\alpha_0\).
Thus, we have \(\eta(\alpha_0, 0, 0)=0\) and the corresponding Taylor expansion yields
\begin{align}
  \eta \approx -\frac{q(\alpha-\alpha_0)}{j\sqrt{\alpha_0}(q/\alpha_0+1)} + q\sqrt{\alpha_0}e_1^2 + e_2^2
\end{align}

\noindent
In resonance, \(\eta\) is conserved as written to good approximation.
Hence, the only nonzero terms in its derivative, \(\dot{\eta}\),
can be from dissipative effects.
The evolution over time of the conserved quantity \(\eta\) is then given by
\begin{align}
\label{doteta}
  \dot\eta = \frac{q\alpha_0^{1/2}}{j(q\alpha_0^{-1}+1)}&\left[ \frac{1}{T_{m,2}} - \frac{1}{T_{m,1}}
      + \frac{2e_1^2}{T_{e,1}}- \frac{2e_2^2}{T_{e,2}} \right] \nonumber\\
    &- q\alpha_0^{1/2}\frac{2e_1^2}{T_{e,1}} - \frac{2e_2^2}{T_{e,2}}=0.
\end{align}

\noindent We note that \(\dot\eta\) depends only on the
\emph{effective} migration rate, \(1/T_m \equiv 1/T_{m,2} - 1/T_{m,1}\).  This
equation, combined with equation \eqref{dotdpom}, determines the
equilibrium eccentricities to first order. The requirement \(d\Delta\varpi/dt=0\) sets their ratio,
and equation \eqref{doteta} sets their magnitude based on \(1/T_m\), \(T_{e,1}\),
and \(T_{e,2}\).

By solving the four equations \eqref{dote1} -- \eqref{dotdpom} and
\eqref{doteta}, we may calculate the exact equilibrium values for the system.
In the standard picture while neglecting secular terms (i.e., for small
\(e_i\)), equations \eqref{dote1} and \eqref{dote2} show
\(\sin(\theta_i)\approx 0^\circ\).  Equation \eqref{dotdpom} gives us
\(\abs{\cos\theta_i} \approx 1\) and \(\cos\theta_1 = -\cos\theta_2\), but
the solution \((\theta_1,\theta_2)=(0^\circ,180^\circ)\) is unstable.  Hence,
\(\theta_1\approx180^\circ\) and \(\theta_2\approx 0^\circ\) in equilibrium.  Since
\(\theta_1-\theta_2 = \varpi_2-\varpi_1\), we therefore see that
convergent migration produces anti-aligned periapses.  We confirm this
in the time-dependent integration in Figure \ref{fig:standardex}.
The equilibrium \(e_i\)'s and \(\Delta\varpi\)'s for comparable mass
planets \((q\in[0.5,2])\) are given in Figures \ref{fig:standardeqecc} and
\ref{fig:standardDpom}.
We calculated the analytical solutions to equations \eqref{dote1} -- \eqref{dotdpom}
and \eqref{doteta}
with the $\mathtt{scipy}$ root finding library.
and plotted them as dashed lines.
We compare the analytical results to a numerical integration of the 
time-dependent differential equations from Hamiltonian
\eqref{hamiltonian} and plot the average \(e_1\), \(e_2\), and
\(\Delta\varpi\) over the last 10\% of the timespan.  These results are
calculated with outward migration for \(q>1\) and inward migration for
\(q<1\).

As we can see in Figures \ref{fig:standardeqecc} and
\ref{fig:standardDpom}, the final averaged eccentricities for \(m_1\) and
\(m_2\) go approximately as \(e_2/e_1 \sim q\). As expected, the
\(\Delta\varpi\) average values are all very close to \(180^\circ\).  The
numerical (markers) and analytical (lines) results largely agree.
Hence, in the standard picture of resonant capture, comparable mass
planets will always end up caught in both \(\theta_1\) and \(\theta_2\)
and consequently apsidally anti-aligned.

In the next two sections, we will explore the effects of eccentricity
damping ratio \(T_{e,1}/T_{e,2}\) on the apsidal angle in MMR
equilibrium.  By forcing the ratio away from \(T_{e,1}/T_{e,2}\sim
1/q\), we hope to shift the capture equilibrium, with the ultimate goal
of finding disk conditions which naturally give rise to apsidal
alignment in exoplanetary systems.

\subsection{Eccentricity damping timescales}
\label{sec:org824a897}
\begin{figure}
  \centering
  \includegraphics[width=0.3\textwidth]{{./varyTe-eeqs-h-0.03-Tw0-1000}.png}
  \caption{ In each row, we have plotted the final time-averaged
    eccentricities for varying $T_{e,1}/T_{e,2}$ while keeping $q$
    fixed to values 0.5, 1, and 2. We have kept $T_{e,0}$, $h$, and
    the initial conditions fixed to the same values as in Figure
    \ref{fig:standardex}.  The dashed lines indicate the analytical
    solutions to equations (\ref{dote1}) -- (\ref{dotdpom}) and
    (\ref{doteta}), as in \ref{fig:standardeqecc}.  The errorbars are
    calculated in the same way as Figure \ref{fig:standard}, but they fall
    within the plot markers.  For $T_{e,1}/T_{e,2}>1$, migration is
    inwards, and vice versa for $T_{e,1}/T_{e,2}<1$, since we keep the
    ratio $T_{e,i}/T_{m,i}\propto h^2$. For $q=0.5$, the inward
    migrating branch agrees well with the equilibrium
    equations. However, on the outward migrating branch the analytic
    solutions typically overestimate the final eccentricities. The
    results are analagous for $q=2$, and both branches roughly agree
    for $q=1$. }
  \label{fig:eqecc}
\end{figure}

\begin{figure}
  \centering
  \includegraphics[width=0.3\textwidth]{{./varyTe-pomega-h-0.03-Tw0-1000}.png}
  \caption{Same as Figure \ref{fig:eqecc} but for $\Delta\varpi$. In all
    cases, despite varying the eccentricity damping ratio, the
    apsides are tightly anti-aligned. The analytical solutions for
    the inward migrating branches (i.e., $T_{e,1}/T_{e,2}<1)$)
    typically predict $\Delta\varpi$ slightly larger than
    $180^\circ$, and slightly smaller for the outward migrating
    branch. The variation in $\Delta\varpi$ is larger for smaller
    values of the damping ratio in all cases.}
  \label{fig:eqDpom}
\end{figure}
Up until now, we have strictly been considering the standard picture
of planet migration -- with \(T_{e,1}/T_{e,2} = 1/q\) and
\(T_{e,i}=3.46h^2T_{e,i}\) -- which always gives rise to apsidal
anti-alignment for reasonable disk conditions (\(h\lesssim 0.1\),
\(T_{e,i}\sim h^2 T_{m,i}\)). This simple parametrized model fails to
capture the complicated hydrodynamics of real astrophysical disks, and
so we could therefore expect a difference in the ratio
\(T_{e,1}/T_{e,2}\) over an order of magnitude.  We would like to
determine the effects of the eccentricity damping ratio on the
equilibrium values of \(e_i\) and whether such a change could lead to
apsidal alignment.

We explore this possibility in Figures \ref{fig:eqecc} and
\ref{fig:eqDpom}. The ratio \(T_{e,1}/T_{e,2}\) varies freely between
\(0.2\) and \(10\), regardless of the mass ratio.  The migration
timescales are still set to \(\abs{T_{m,i}}=T_{e,i}/3.46 h^2\).  For
\(T_{e,1}<T_{e,2}\), then, we set \(T_{m,i}>0\), corresponding to outward
migration, and vice versa for \(T_{e,1}>T_{e,2}\).

For comparable mass planets with \(q=0.5\), \(1\), and \(2\), varying the
ratio \(T_{e,1}/T_{e,2}\) around \(1/q\) modifies the final equilibrium
eccentricities by a roughly similar factor, as seen in Figure
\ref{fig:eqecc}.  The dashed lines plot the analytic results from
solving equations \eqref{dote1} -- \eqref{doteta}; these findings largely
reproduce the numerical results if the system migrates in the same
direction as in the standard picture,
For Te1/Te2 far from 1/q,
the analytical results systematically overestimate the equilibrium
eccentricity for Te1/Te2 < 1, and underestimate it for Te1/Te2 > 1.
The eccentricity ratio \(e_1/e_2\) is
unchanged, yet the magnitudes \(e_1\) and \(e_2\) are larger for more
extreme values of \(T_{e,1}/T_{e,2}\).  The corresponding values for
\(\Delta\varpi\) are shown in Figure \ref{fig:eqDpom}. In all cases, the
analytic equilibrium equations predict \(\Delta\varpi\approx
180^\circ\), and the numerical integrations agree.

The peaked shape of the dashed lines in Figures \ref{fig:standardeqecc} and
\ref{fig:eqecc} can be explained as follows.  As
\(T_{m,1}/T_{m,2}=T_{e,1}/T_{e,2}\) approaches one, the effective
migration timescale \(T_m\) approaches infinity.  Equation \eqref{doteta}
therefore implies that the planets' eccentricities will approach zero.
We note that the equilibrium solutions given in Figures
\ref{fig:standard} - \ref{fig:eqDpom} are not continuous across the line
\(T_{e,1}/T_{e_2} = 1\) (i.e. \(q=1\) in Figure \ref{fig:standard}), which
is where we reverse the migration direction to ensure it is
convergent.

Variations in the eccentricity damping ratio cannot
account for apsidal alignment; \(e_1\) and \(e_2\) always adjust to
satisfy equation \eqref{dotdpom}.  In order to simplify the problem, we
turn to the case of a test particle orbiting near an MMR with a finite
mass planet. The planet's orbit is unchanging, and so we may isolate
the effect of eccentricity on the apsidal angle.

\section{Test Particle Results}
\label{sec:org585941d}
\begin{figure*} \centering
  \includegraphics[width=0.7\textwidth]{{tp-h0.030-ext-ep0.000-circ}.png}
  \caption{Here we have plotted the capture outcome for a test particle
  into an exterior 3:2 MMR with a planet of mass $\mu_p=10^{-4}$,
  which is on a circular orbit ($e_p=0$). We have set $h=0.03$,
  $T_{e}=1000$~yrs, and let the system evolve into equilibrium.  The
  capture occurs similarly to the comparable mass case in Figure
  \ref{fig:standardex}, but $\theta_1$ never librates because
  $m_2=0$. The green dashed line in the bottom left panel indicates
  the equilibrium eccentricity for the system as calculated from
  equation (\ref{eq:eeqint}). The test particle's $\varpi$ circulates.}
  \label{fig:tp-ep0}
\end{figure*}

%\begin{figure} \centering
%\includegraphics[width=0.45\textwidth]{{tp-eq-eccs}.png}
%\caption{Here we have plotted the eccentricity of a test particle in
%  interior (left) and exterior (right) MMRs for various values of $h$.
%  The colors of the lines correspond to the value of $h$ chosen. All
%  other parameters are identical to Figure \ref{fig:tp-ep0}.}
%\label{fig:tp-eqeccs}
%\end{figure}

\begin{figure*} \centering
  \includegraphics[width=0.7\textwidth]{{tp-h0.030-ext-ep0.040-circ}.png}
  \caption{Here we have plotted the capture process for a test particle
    into a 3:2 MMR for a planet with $\mu_p=10^{-4}$ on an orbit with
    eccentricity $e_p=0.04$. All other parameters are identical to
    Figure \ref{fig:tp-ep0}. As we can see, the particle is captured
    into resonance, but $\theta_2$ librates with large
    amplitude. Similarly, the eccentricity librates with large
    amplitude. The test particle's $\varpi$ still circulates.}
  \label{fig:tp-circ}
\end{figure*}

\begin{figure*}
  \centering
  \includegraphics[width=0.7\textwidth]{{tp-h0.030-ext-ep0.054-aligned}.png}
  \caption{Here we have plotted the capture process for a test
    particle into a 3:2 MMR for a planet with $\mu_p=10^{-4}$ on an
    orbit with eccentricity $e_p=0.054$. All other parameters are
    identical to Figure \ref{fig:tp-ep0}. This time, period ratio is
    still trapped very close to the nominal resonance location,
    $P_2/P_1 = 1.5$, but the resonance angle $\theta_2$ circulates
    throughout the simulation.  As in Figure \ref{fig:tp-circ}, the
    test particle eccentricity still librates with large amplitude,
    but now the planets apsidal angle is aligned with
    $\varpi_p\equiv 0^\circ$.}
  \label{fig:tp-align}
\end{figure*}

\begin{figure} \centering
  \includegraphics[width=0.4\textwidth]{{tp-grid-ext}.png}
  \caption{Here we have summarized the behavior of $\varpi$ for a
    range of values for $h$ and $e_p$ in an external 3:2 MMR with a
    planet of mass $\mu_p=10^{-4}$. Generally, for $e_p$ large enough,
    the system will become apsidally aligned, as in Figure
    \ref{fig:tp-align}.  The dashed line indicates our rough
    analytical approximation for the boundary between the
    $\varpi$-aligned and $\varpi$-circulating regions.}
        \label{fig:tp-grid-ext}
\end{figure}
The dynamics of first order MMRs with comparable mass planets are
complicated by the presence of two critical arguments in the
Hamiltonian, \(\theta_1\) and \(\theta_2\).  By assuming that one of the
planets is a test particle, i.e. \(m_1=0\), we may ignore the variation
of the other planet, \(m_2\).  This greatly simplifies the problem.  The
respective case for an external test particle is setting \(m_2=0\) while
keeping the orbit of planet \(m_1\) constant.  To emphasize the fact
that we are formally transitioning to a different problem, we adopt
the following notation: The subscript \(p\) will denote the constant
orbital elements and coordinate-momentum pairs of the massive planet,
while no subscript will indicate these quantities for the test
particles.

We assume that no dissipative forces act on the massive perturber
while implementing a migration model identical to the one used in
Section \ref{sec:org16ae390} for the test particle.
Equivalently, the dissipative timescales acting on the massive planet
are assumed to approach infinity.  To remain consistent with notation,
we label the sma- and \(e\text{-damping}\) for the test particle as
\(T_{m}\) and \(T_{e}\), respectively. This problem is well-studied in the
literature, and so in lieu of a detailed derivation we briefly
summarize the problem with references to the key results.

\subsection{Equations of motion}
\label{sec:org3ef8a30}
We will work with the dimensionless Hamiltonian with units set by the
perturber's orbital elements.  Including secular effects, the
Hamiltonian for a test particle in the vicinity of an interior MMR is
\citep[c.f.][]{xu_migration_2018,goldreich_overstable_2014,wisdom_canonical_1986,deck_migration_2015,henrard86_reduc_trans_apocen_librat}
\begin{align*}
  \label{eq:tpint}
  H
  &= - \frac{1}{2(a/a_p)} - \mu_p\left(f_1
    e\cos\theta +f_2 e_p\cos\theta_p\right) \\
  &- \mu_p \left(f_2\left(e^2 + e_p^2\right)
    + f_4e_p e\cos\varpi\right)
\end{align*}

\noindent
where
\begin{align}
  \theta &= (j+1)n_p t - j\lambda - \varpi \\
  \theta_p &= (j+1)n_p t - j\lambda - \varpi_p.
\end{align}

\noindent
For an external MMR, 
\begin{align*}
  \label{eq:tpext}
  H
  &= - \frac{1}{2(a/a_p)} + \mu_p\left(f_1
    e\cos\theta +f_2
    e_p\cos\theta_p\right) \\
  &- \mu_p \left(f_2\left(e^2
      + e_p^2\right) + f_4e_p
    e\cos\varpi\right)
\end{align*}

\noindent
and
\begin{align}
  \theta &= (j+1)\lambda - jn_pt - \varpi \\
  \theta_p &= (j+1)\lambda - jn_pt - \varpi_p.
\end{align}

In these two Hamiltonians, \(\theta_p\) is now an explicit function of
time, since \(\lambda_p=n_p t\) and \(\varpi_p=0\) is constant.  For
simplicity, we first consider the case for which the massive planet is
on a circular orbit, i.e. \(e_p=0\).
In this case, \(\varpi_p\) is undefined, and \(\gamma\) will
always circulate in equilibrium.
This problem is known as the circular restricted three
body problem (CR3BP), while \(e_p>0\) is the eccentric restricted three
body problem (ER3BP).  In Figure \ref{fig:tp-ep0}, we've plotted an
example of a test particle captured into an external resonance with
\(h=0.03\), \(\mu_p = 10^{-4}\), and \(T_{e} = 1000\).  The test particle is
captured into the \(\theta\) resonance, while its longitude of
perihelion, \(\varpi\), and the other resonance angle, \(\theta_p\),
circulates.

\subsection{Equilibrium eccentricity and stability}
\label{sec:orga313c5c}
When a test particle is caught into a stable resonance, its
eccentricity grows and saturates at a finite value which depends on
the ratio \(T_{e}/T_{m}\).  For a detailed derivation, see e.g.
\citep{goldreich_overstable_2014} and \citep{xu_migration_2018}.
By setting \(\dot\eta=0\) (equation \ref{doteta}), equilibrium
in \(e\) arises once the forces of migration into the resonance, \(T_m\), balances
the forces of eccentricity damping, \(T_e\). For an internal
test particle, the equilibrium eccentricity is
\begin{equation}
\label{eq:eeqint}
  e_{\rm eq} = \sqrt{\frac{T_e}{2(j+1)T_m}} = h\sqrt{\frac{1.78}{j+1}}.
\end{equation}

\noindent
For an external test particle,
\begin{equation}
\label{eq:eeqext}
  e_{\rm eq} = \sqrt{\frac{T_e}{2jT_m}} = h\sqrt{\frac{1.78}{j}}.
\end{equation}

For a test particle, capture occurs if migration is slow enough so that
\citep{goldreich_overstable_2014}
\begin{equation}
\mu_p^{4/3} \geq \frac{2.5}{j^{5/3}n T_m}.
\end{equation}

\noindent
If captured into an external resonance, the test particle equilibrium
is stable. However, for an interior resonance, if the equilibrium
eccentricity is larger than
\begin{equation}
  \frac{3j^2}{8\alpha_0 \abs{f_1}}e_{\rm eq}^3 > \mu_p
\end{equation}

\noindent
then the capture is only temporary and the particle escapes after a
long enough time. On the other hand, if the eccentricity is smaller
than
\begin{equation}
  \frac{3j}{\alpha_0 \abs{f_1}}e_{\rm eq}^3 < \mu_p
\end{equation}

\noindent
then the particle librates around its equilibrium eccentricity with
finite amplitude. Otherwise, the particles' librations in eccentricity
damp to zero around \(e_{\rm eq}\).

%In Figure \ref{fig:tp-eqeccs}, we have plotted the evolution of
%eccentricities in the CR3BP for various values of $h$ and the
%corresponding equilibrium eccentricities, $e_{\rm eq}$.  Each run
%levels off at the equilibrium value after a period of eccentricity
%growth following the initial capture into resonance.

\subsection{Eccentric massive planet}
\label{sec:org4f06f96}
Now, after considering the
CR3BP, we may move on to the ER3BP, where \(e_p>0\).  For moderate
eccentricities, \(0<e_p\lesssim 0.1\), the qualitative features of the
CR3BP are preserved as long as \(e_p \lesssim e_{\rm eq}\). In Figure
\ref{fig:tp-circ}, we have plotted the capture outcome with the same
parameters as Figure \ref{fig:tp-ep0} but setting \(e_p = 0.04\). As we
can see, the particle is still captured into resonance and \(\theta_2\)
begins librating around 180\(^\circ\). The eccentricity \(e\) librates
around its equilibrium value with large amplitude. The test particles
longitude of perihelion, \(\varpi\), still circulates.

On the other hand, if \(e_p \gtrsim e_{\rm eq}\), the test particle's
migration halts near the nominal resonance location of \(P_2/P_1=1.5\)
while both \(\theta_1\) and \(\theta_2\) continue to circulate. The particle's
eccentricity librates with slightly larger amplitude than
in the previous case. Finally, the system becomes apsidally aligned.

In Figure \ref{fig:tp-grid-ext}, we summarize the behavior of
\(\Delta\varpi\) for an internal and external test particle while
varying the values \(e_p\) and \(h\) (and subsequently \(e_{\rm
eq}\)). Generally, for large enough \(e_{\rm eq}\gtrsim e_p\), the system
becomes apsidally aligned.

These test particle results motivate the eccentricity driving ansatz
we employ in the next section for comparable mass planets. We
demonstrate that there is a resonance \(\hat\theta\) which operates
independently of the \(\theta_i\) resonances and traps the planets into
resonance despite the circulation of both \(\theta_i\). For a detailed
analysis of how \(\hat\theta\) reduces from the comparable mass
(\(0<q<\infty\)) case to the test particle case (\(q\to0\) or
\(q\to\infty\)), see Section \ref{sec:orgf23a5a9}. In the same section,
we will also describe how we calculate the dashed line in Figure
\ref{fig:tp-grid-ext}, which depicts our analytical estimate for the
boundary between the $\varpi$-aligned and $\varpi$-circulating
regions.

\section{Apsidal Angle}
\label{sec:org34f950f}
As we have seen, capture into the \(\theta_1\) and \(\theta_2\) resonance
always leads to \(\Delta\varpi\approx 180^\circ\) due to their
equilibrium values being close to \(180^\circ\) and \(0^\circ\),
respectively.  The apsidally aligned K2-19 system therefore poses a
problem for our standard model.  In order to match this observation,
either \(\theta_1\), \(\theta_2\), or both angles must circulate.

In the previous section, we saw that apsidal alignment arises whenever
the massive planet has an eccentricity larger than the equilibrium
eccentricity of the test particle in resonance. If this is the case,
the CR3BP resonance angle circulates and \(e\) librates with large
amplitude, but the particle is still trapped near the period resonance
location, \(P_2/P_1=1.5\).  Guided by this result, in this section, we
formulate a toy model for driving the eccentricities of comparable
mass planets to an arbitrary value and explore the resulting apsidal
angle. Then, we reduce the Hamiltonian of the system to a single
degree of freedom through a series of canonical transformations.  This
formulation allows us to identify the key dynamical processes that
lead to alignment. Finally, we take the test particle limit of the
Hamiltonian to connect $\varpi$-alignment in the comparable
mass to the results in the previous section.

\subsection{Eccentricity driving forces}
\label{sec:orga1fd784}
\begin{figure*}
  \centering
  \includegraphics[width=0.7\textwidth]{{inres-driving-example-h-0.03-Tw0-1000-q2.0}.png}
  \caption{Here we have set $e_{1,d}=0.2$ and $e_{2,d}=0$ with
    $h=0.03$ and $q=2$, so that we are driving the eccentricity of the
    larger inner planet. All other initial conditions are held the
    same as in Figure \ref{fig:standardex}, except for the initial period
    ratio, which we set to the nominal resonance location,
    $P_2/P_1=1.5$, so that the system is very quickly caught into
    $\theta_1$ and $\theta_2$.  After about 10~kyr, the system escapes
    the circular resonances, indicated by the circulation of
    $\theta_1$ and $\theta_2$. At this point, the planets becomes
    apsidally aligned and $\Delta\varpi$ librates around $0^\circ$
    with large amplitudes.}
  \label{fig:drivingex}
\end{figure*}

\begin{figure*}
  \centering
  \includegraphics[width=0.7\textwidth]{{inres-driving-perpendicular-example-h-0.03-Tw0-1000-q2.0}.png}
  \caption{The initial conditions and system parameters are identical
    to Figure \ref{fig:drivingex}, but we are instead driving the
    eccentricity of smaller outer planet, with $e_{1,d}=0$ and
    $e_{2,d}=0.2$. Similar to that integration, after about 10~kyr,
    the system escapes the $\theta_1$ resonance, indicated by its
    shift in libration center to $270^\circ$ rather than $180^\circ$.
    We see that $\theta_2$ still librates around $0^\circ$, and so the
    apsidal angle librates tightly around $90^\circ$.}
  \label{fig:perpex}
\end{figure*}
Theoretical works suggest that interactions with the PPD can actually
\emph{increase} a planet's eccentricity rather than damp it
\citep{ragusa17_eccen_evolut_durin_planet_disc_inter,goldreich03_eccen_evolut_planet_gaseous_disks,teyssandier17_secul_evolut_eccen_protop_discs}.
In some systems, super-earth-sized luminous protoplanets can saturate
at eccentricities larger than the disks aspect ratio
\citep{romero21_eccen_drivin_pebbl_accret_low_mass_planet}.  The planets
K2-19b and c are moderately eccentric, with \(e_{b}\approx e_c\approx
0.2\) \citep{petigura_k2-19b_2020}.  In \citet{petit_resonance_2020}, the
authors suggest that the apsidal alignment in the system could be
caused by eccentricity driving to a common value.  To mimic the
effects of an externally induced driving eccentricity, we modify the
eccentricity damping for \(m_i\) in equation \eqref{eq:disforce} to be
\begin{equation}
  \frac{\dot e_i}{e_i} = -\frac{(e_i-e_{i,d})}{T_{e,i}},
\end{equation}

\noindent so that planet \(m_i\) is driven to \(e_{i,d}\) with a
timescale of \(T_{e,i}\).  Utilizing this eccentricity driving force we
may examine the \(j:j+1\) MMR system's response to forced disequilibrium
with respect to \(\theta_1\) and \(\theta_2\).

In Figure \ref{fig:drivingex}, we demonstrate the feasibility of this
approach, where we integrate the time-dependent equations while
driving the inner planet's eccentricity to \(e_{1,d}=0.2\).  The outer
planet has \(e_{2,d}=0.0\) and the mass ratio is \(q=2\).  We initalize
the system with \(e_{1}=e_2=0.001\) at the nominal resonance location,
\(P_{2}/P_1 = 1.5\). The planets are caught in the \(\theta_1\) and
\(\theta_2\) resonances for 10,000 years, after which the planets escape
and the angles circulate. At this point, both planets' ecentricities
are excited to about \(e_i\approx 0.2\) and the planets become apsidally
aligned as \(\Delta\varpi\) librates around \(0^\circ\) with a large
amplitude.  Despite the circulation of both resonance angles, the
period ratio remains locked very close to the nominal resonance
location (\(P_2/P_1= 1.5\)). The system is caught in a different type of
resonance which we will study in the following subsection.

On the other hand, for \(e_{2,d}=0.2\) and \(e_{1,d}=0\), the sytem
displays different resonant behavior.  We have plotted the details of
this system in Figure \ref{fig:perpex}. The angle \(\theta_2\) librates
with large amplitude around its resonant value of \(0^\circ\), whereas
\(\theta_1\) librates around \(270^\circ\) rather than \(180^\circ\).  As a
result, \(\Delta\varpi\) approaches \(90^\circ\) and so the planets'
perihelia are now perpendicular to each other.

\subsection{Reducing the Hamiltonian}
\label{sec:org2071b06}
\begin{figure}
  \centering
  \includegraphics[width=0.45\textwidth]{{./S2-conserved-inres-q2.0}.png}
  \caption{In the left column, we have plotted $\sigma_1$ and
    $\sigma_2$ from equations (\ref{eq:sS1}) and (\ref{eq:sS2})
    for the three different combinations of $e_{1d}$ and $e_{2d}$
    in Figures \ref{fig:standardex}, \ref{fig:drivingex}, and
    \ref{fig:perpex}. Each row corresponds to one of the three
    different modes of resonance identified in this paper,
    $\Delta\varpi=180^\circ$, $\Delta\varpi=0^\circ$, and
    $\Delta\varpi=90^\circ$, respectively. In the right column, we
    plot the evolution of the conserved resonance quantity $S_2$
    (equation (\ref{eq:S2eq})).  For $\Delta\varpi=180^\circ$, all three
    of $\sigma_1$, $\sigma_2$, and $S_2$ are conserved near
    zero. For the other two cases, $S_2$ transitions to larger values
    near $\sim 0.5$ (middle) and $\sim 0.1$ (bottom) as the
    eccentricity driving reach an equilibrium.  Both $\sigma_1$
    and $\sigma_2$ are excited to factors of a few in the
    apsidally aligned case, while only $\sigma_1$ is excited for
    $\Delta\varpi=90^\circ$.  For $\Delta\varpi=0^\circ$ and
    $\Delta\varpi=90^\circ$, the eccentricities oscillate in such a
    way to conserve $S_2$ according to equation (\ref{eq:S2eq}).}
  \label{fig:S2cons}
\end{figure}

\begin{figure}
  \centering
  \includegraphics[width=0.45\textwidth]{{./relative-geometry}.png}
  \caption{Here we have plotted the eccentricity vectors
    $\mathbf{e}_1$, $\mathbf{e}_2$, and $\mathbf{\hat e}$ in the
    reference frame of $\mathbf{e}_1$. Similar to Figure
    \ref{fig:S2cons}, each row is a different combination of $e_{1d}$
    and $e_{2d}$ corresponding to the three different modes of
    resonance. The left column shows the initial conditions of the
    resonance, while the right column shows the evolution at late
    times. The top row ($\Delta\varpi=180^\circ$) exhibits little
    qualitative change besides the libration amplitudes shrinking to
    zero.  The second row ($\Delta\varpi=0^\circ$) shows
    $\mathbf{e}_2$ circulating around $\mathbf{e}_1$ strictly
    contained to the second and third quadrants. The last row
    ($\Delta\varpi=90^\circ$) shows $\mathbf{e}_1$ and $\mathbf{e}_2$
    transitioning into a perpendicular arrangement. Meanwhile, the
    $\mathbf{\hat e}$ vector circulates outside of both $\mathbf{e}_1$
    and $\mathbf{e}_2$ in the second row, while remaining aligned with
    $\mathbf{e}_2$ in the last row.}
  \label{fig:relgeom}
\end{figure}
A detailed analysis of the MMR Hamiltonian \eqref{hamiltonian}
illustrates the underlying dynamics behind the capture processes in
Figure \ref{fig:drivingex} which lead to apsidal alignment.  It can be
shown that \(\theta_1\) and \(\theta_2\) are actually subresonances of a
resonance \(\hat\theta\) which arises after transforming the system's
Hamiltonian so that it has only a single degree of freedom. 

If we assume that the secular behavior of the semimajor axis ratio
\(\alpha\) is stationary or varying adiabatically, we may transform the
resonant Hamiltonian \(H_{\rm Kep} + H_{\rm res}\) in equation
\eqref{hamiltonian} into the form
\begin{equation}
  \label{hhat}
  \hat H(\hat R,\hat\theta) = -3(\delta+1) \hat R + \hat R^2 - 2\sqrt{2\hat R} \cos(\hat\theta)
\end{equation}

\noindent through a series of rotations in phase space.  For
the details of these transformations, see Appendix \ref{sec:org34caf95}.
The Hamiltonian in equation \eqref{hhat} is well studied
in the literature.
\citep[for an extensive review in a planetary context, see][]{moutamid14_coupl_between_corot_lindb_reson}
The parameter \(\delta\) quantifies
the system's depth into resonance.  We do not include \(H_{\rm sec}\) in
this analysis because it is second order in eccentricities.

For the case where \(q>0\) is finite, both planets will be on eccentric
osculating orbits.  Hence, we expect the action \(\hat R\) in equation
\eqref{hhat} to be a function of both \(e_1\) and \(e_2\).  Define \(\v{\hat
e} = \abs{f_1}\v e_1 - \abs{f_2}\v e_2\) and \(\hat e = \abs{\v{\hat
e}}\), where \(\v e_i\) is the Runge-Lenz vector, i.e. the vector with
magnitude \(e_i\) in the direction of perihelion.  The action \(\hat R\)
takes the form \(\hat R \propto \tilde \mu \hat e^2\), where
\(\tilde\mu=\tilde m/M\) and \(\tilde m= m_1m_2/(m_{\rm tot}M)\) is the
reduced mass ratio of the planets.  The coordinate angle, \(\hat\theta\),
is given by
\begin{align}
\label{hattheta}
  \tan\hat{\theta}_1 = \frac{W_1}{w_1} = \frac{f_1 e_1\sin(\theta_1)
  + f_2e_2\sin(\theta_2)}{f_1e_1\cos(\theta_1) + f_2e_2\cos(\theta_2)}.
\end{align}

\noindent
Appendix \ref{sec:org34caf95}
describes the detailed behavior of \(\hat H\), \(\hat \theta\),
and \(\hat R\) in the CR3BP limits.
This angle is the same one that \citet{petit_resonance_2020}
found to librate in K2-19b and c.

\subsection{Three modes of resonance}
\label{sec:orge0bfaec}
Under the assumption that the semimajor axes are stationary (or slowly
varying), the Hamiltonian in equation \eqref{hamiltonian} has two
degrees of freedom (dof) as written: the angles \(\theta_1\) and
\(\theta_2\).  However, by transforming this Hamiltonian into the single
dof form of equation \eqref{hhat} the resonance admits the following
conserved quantity (see equation \eqref{eq:appS2deriv}):
\begin{align}
  \label{eq:s2}
  S_2 = q\sqrt{\alpha}f_2^2e_1^2
-2\abs{f_1f_2}e_1e_2\cos(\varpi_1-\varpi_2) + \frac{f_1^2}{q\sqrt\alpha}e_2^2.
\end{align}

\noindent
By enforcing \(dS_2/dt = 0\) together with the assumption \(d\alpha/dt
\approx 0\), we arrive at the following equilibrium condition:
\begin{align}
  \label{eq:S2eq}
  \frac{dS_2}{dt} \propto e_1^2\left(\frac{e_1-e_{1\rm d}}{T_{e,1}}\right)\abs*{\frac{f_2}{f_1}}
  \sigma_1
  + e_2^2\left(\frac{e_2-e_{2\rm d}}{T_{e,2}}\right)
  \sigma_2
  = 0.
\end{align}

\noindent
where
\begin{align}
  \label{eq:sS1}
  \sigma_1=&\left[
                  q^2\alpha\abs*{\frac{f_2}{f_1}}
                  + \frac{e_2}{e_1}q\sqrt{\alpha}\cos(\varpi_1-\varpi_2)
                  \right]\\
  \label{eq:sS2}
  \sigma_2=&\left[
                  \abs*{\frac{f_2}{f_1}} q\sqrt{\alpha}
                  \frac{e_1}{e_2}\cos(\varpi_1-\varpi_2) + 1
                  \right].
\end{align}

The systems in Figures \ref{fig:standardex}, \ref{fig:drivingex}, and
\ref{fig:perpex} are representative of three different modes of
resonance, ones where \(\Delta\varpi=180^\circ\),
\(\Delta\varpi=0^\circ\), and \(\Delta\varpi=90^\circ\),
respectively. These correspond to three different behaviors of the
quantities \(\sigma_1\) and \(\sigma_2\) while in resonance
under the influence of eccentricity forcing.
In Figure \ref{fig:S2cons}, we have plotted \(\sigma_1\) and
\(\sigma_2\) as well as \(\dot S_2/S_2\) for these systems.  The top
row is for the standard eccentricity damping case where
\(e_{1d}=e_{2d}=0\).  Once the system equilibrates, \(S_2\approx 10^{-4}\)
is well conserved (top left) and small. Both \(\sigma_1\) and
\(\sigma_2\) are also close to zero. From equations \eqref{eq:sS1}
and \eqref{eq:sS2}, we see this corresponds to \(e_2/e_1 \sim q\), as we
found in Section \ref{sec:org16ae390}.

The second row corresponds to the integration shown in Figure
\ref{fig:drivingex}, where \(e_{1d}=0.2\) and \(e_{2d}=0\).  For early
times, while the system is still caught in the \(\theta_1\) and
\(\theta_2\) resonances, \(\sigma_1\), \(\sigma_2\), and \(S_2\) are
small.  Once the \(\theta_i\) resonances are broken, and only
\(\hat\theta\) librates, they are excited to larger values.  The
quantities \(\sigma_1\) and \(\sigma_2\) undergo large periodic
oscillations away from zero, while \(S_2\) grows and then stabilizes at
its new equilibrium value, \(S_2\approx 0.5\). The planets'
eccentricities oscillate in such a way as to conserve \(S_2\).  The last
row corresponds to the integration in \ref{fig:perpex}, where
\(e_{2d}=0.2\) and \(e_{1d}=0\).  For this system, \(\sigma_2\) is
conserved close to 0, while \(\sigma_1\) grows to a magnitude
similar to its value in the apsidally aligned case.

In Figure \ref{fig:relgeom}, we plot the eccentricity
vectors \(\mathbf{e}_{1}\), \(\mathbf{e}_{2}\), and \(\mathbf{\hat e}\)
in the reference frame which rotates with \(\mathbf{e}_1\).
The three systems begin with the same configuration,
caught in the \(\theta_1\) and \(\theta_2\) resonances.
The vectors \(\mathbf{e}_1\) and \(\mathbf{e}_2\) are anti-aligned,
while \(\mathbf{\hat e}\) aligns with \(\mathbf{e}_1\).
At later stages, the system without eccentricity driving
remains in this configuration (top row).
At later times, the second and third rows exhibit the new resonance
behaviors described above.  For the apsidally aligned case,
\(\mathbf{e_2}\) circulates in a pattern strictly constrained to the
second and third quadrants, and \(\mathbf{\hat e}\) circulates around
the other two vectors.  In the last row, \(\mathbf{e}_1\) and
\(\mathbf{e}_2\) are perpendicular to each other and \(\mathbf{\hat e}\)
is aligned with \(\mathbf{e}_2\).

\subsection{Saturation eccentricities}
\label{sec:org0e3d92f}
\begin{figure} \centering
\includegraphics[width=0.4\textwidth]{{./Rhat-grids}.png}
    \caption{\emph{Left:} The resonance architecture for integrations
spanning a grid of driving eccentricities. Here we hold $h=0.03$,
$T_{e,0}=1000$~yrs, while varying the driving saturation
eccentricities $e_{1,d}$ and $e_{2,d}$.  Systems which are not
captured or become unstable and escape resonance are plotted as black
``x''-marks. Roughly, for $e_{1d}\gtrsim e_{2d}$, the system becomes
aligned.  For $e_{2d}>e_{1d}$, the system becomes perpendicular for
large values of $e_{2d}$. The other simulations remain trapped in both
of the $\theta_1$ and $\theta_2$ resonances.  \emph{Right:} The
numerically averaged final eccentricities of the systems shown in the
left panel. The apsidally aligned systems fall just below the line
$e_2=e_1$. Their error bars correspond to the standard deviation of
the eccentricities. The perpendicular system fall just above the line
$e_2=2e_1=qe_1$, while the anti-aligned systems fall just below
it. Both have small librations compared to the aligned case.}
    \label{fig:Rhat-grid}
  \end{figure}

\begin{figure*}
  \centering
  \includegraphics[width=0.7\textwidth]{{./addenda/q2.0/inres/inres-driveTe-h-0.03-mutot-1.0e-04-Tw0-1000-q2.0-e1d-0.000-e2d-0.100}.png}
  \caption{This integration corresponds to the $\mathtt{x}$ marker
    in Figure \ref{fig:Rhat-grid} at $e_{1d}=0$, $e_{2d}=0.1$. The
    system starts off in resonance, with all three $\theta_i$ and
    $\hat\theta$ librating. However just after $10^4$ years, the
    system breaks out of all three resonance. Nevertheless, the
    period ratio remains locked around $1.5$ with small
    librations. The eccentricities reach an equilibrium value with
    large librations, while the apsidal angle transitions from
    $\lesssim 180^\circ$ to $\gtrsim 180^\circ$.  }
  \label{fig:escapeex}
\end{figure*}
Now that we have identified these three resonant modes, we briefly
explore the \((e_{1d},e_{2d})\) parameter space of the saturation
eccentricities for moderate values between \(0\) and \(0.2\).  First, we
consider only the top row of Figure \ref{fig:Rhat-grid}, which
corresponds to \(q=2\), \(h=0.03\), and initial period ratio
\(P_2/P_1=1.5\).  The additional rows of Figure \ref{fig:Rhat-grid} show
summary plots analagous to the first row for various combinations of
\(q=0.5,2\) and initial period ratio \(P_2/P_1=1.5,1.55\).  We will
discuss how mass ratio and initial period ratio alter our results in
the next two sections.

In the left panel, we summarize the resonant behavior for each system
on a grid within the parameter space while holding \(q\), \(h\), and the
initial period ratio \(P_2/P_1\) constant. We have excluded runs which
become unstable and escape the resonance within the timescale of our
integrations. Roughly, for \(e_{1d}\gtrsim e_{2d}\), the system becomes
apsidally aligned, while for \(e_{2d} > e_{1d}\), some cases exhibit
\(\Delta\varpi=90^\circ\).  In the right panel of Figure
\ref{fig:Rhat-grid}, we have plotted the time-averaged final
eccentricities. The points share the same color-coding as in the right
panel.  The eccentricities for the aligned cases fall roughly along
the line \(e_1=e_2\), which reflects the fact that the angle
\(\hat\theta\) does not depend on the mass ratio \(q\) (e.g., equation
\eqref{hattheta}).  The perpendicular runs fall slightly above the line
\(e_2/e_1=q\), while the anti-aligned runs fall just under it.

The single \(\mathtt{x}\) marker in Figure \ref{fig:Rhat-grid} corresponds to a
run which is only temporarily caught into resonance.
We have plotted the detailed evolution of this system
in Figure \ref{fig:escapeex}. Despite all of
the resonance angles circulating, the period ratio librates with small
amplitude around the period ratio \(P_2/P_1=1.5\). The planets remain in
an anti-aligned configuration throughout. Before escape,
\(\Delta\varpi\lesssim180^\circ\), while after escape,
\(\Delta\varpi\gtrsim180^\circ\).

\subsection{Initial period ratio}
\label{sec:org47dd507}
\begin{figure}
  \centering
  \includegraphics[width=0.45\textwidth]{{./addenda/longrun-closeres-q2.0-e1d-0.000-e2d-0.200}.png}
  \caption{This integration has set $e_{1d}=0$ and $e_{2d}=0.2$ with
    the initial period ratio just wide of the nominal resonance
    location, $P_2/P_1=1.55$.  All other conditions are exactly the
    same as the top row of Figure \ref{fig:Rhat-grid}.
    The period ratio initially decreases as the planets migrate
    convergently. Around $t=10^4$ years, the period
    ratio increases, then again approaches $1.5$. After this, the system
    is again locked near $P_2/P_1=1.5$ while all resosnant angles
    circulate, similar to the integration in  Figure \ref{fig:escapeex}.}
  \label{fig:escapeex1}
\end{figure}

\begin{figure}
  \centering
  \includegraphics[width=0.45\textwidth]{{./addenda/longrun-closeres-q2.0-e1d-0.200-e2d-0.200}.png}
  \caption{This simulation has set $e_{1d}=0.2$ and $e_{2d}=0.1$ with
    an initial period ratio just wide of the nominal resonance,
    $P_{2}/P_1=1.55$.  Otherwise, all other conditions are the same as
    the top row of Figure \ref{fig:Rhat-grid}. The system convergently
    migrates outward for some time, after which it crosses the
    resonance location without being captured (i.e., the kink between
    $t=3000$ and 4000 years).  The period ratio then turns around and
    misses the resonance again, after which it continues to grow for
    the rest of the simulation.  In this simulation, as opposed to
    Figure \ref{fig:escapeex1}, the resonance angles never appear to
    librate.}
  \label{fig:escapeex2}
\end{figure}
The integrations in this section thus far have all started at the
nominal resonance location, \(P_2/P_1=1.5\), and a mass ratio \(q=2\).
While fully understanding the dynamics of capture and stability of the
\(\hat\theta\) resonance is beyond the scope of this paper, we can begin
to study the effects of initial conditions by slightly shifting the
starting location of the planets in the disk.

In the second row of Figure \ref{fig:Rhat-grid}, we have kept all of the
parameters from the first row constant, but shifted the starting
location of \(m_2\) so that \(P_2/P_1=1.55\). As we can see,
many more systems fail to capture and remain in resonance.

Figure \ref{fig:escapeex1} displays the results of a simulation
which fails to result in resonant capture. The planets
start at a period ratio of \(P_2/P_1=1.5\), and after a short period
of convergent migration, the outer planet is repelled
away from the resonance. After growing for a short time,
the period ratio turns around and, as it approaches
\(1.5\) again, it levels off into a resonance which
looks similar to Figure \ref{fig:escapeex}.

Figure \ref{fig:escapeex2} displays the results of a simulation in which
resonance capture does not occur.  Initially, both planets undergo
convergent outward migration, but after some time, \(a_1\) begins to
increase.  Then, the planets skip the resonance as the period ratio
passes through \(P_2/P_1=1.5\) from above. It continues decreasing,
turns around, and then skips the resonance from below.  Then, the
period ratio continues increasing for the rest of the integration.

\subsection{Mass ratio}
\label{sec:org3628228}
\begin{figure}
  \centering
  \includegraphics[width=0.45\textwidth]{{./addenda/longrun-inres-q0.5-e1d-0.200-e2d-0.100}.png}
  \caption{For this integration, we set $q=0.5$, the initial period
    ratio to be $P_2/P_1=1.55$, and the driving eccentricities to be
    $e_{1d}=0.2$ and $e_2d=0.1$. The planets are initially caught into
    all three resonances for most of the integration.  The period
    ratio is likewise locked near 1.5 with librations that grow
    slightly over time Between 50 and 60,000 years, the system exits
    all three resonances at the same time, after which $m_1$ launches
    outwards leading to eventual orbit-crossing.}
  \label{fig:escapeex3}
\end{figure}
Now we turn to the effect of mass ratio on the systems in this
paper. Rows three and four of Figure \ref{fig:Rhat-grid} depict the
results of running identical simulations to the first two rows but
with \(q=0.5\). Only one system out of the 18 simulations in the last
two rows becomes apsidally aligned, while for all cases with
\(e_{2,d}=0\), the planets are anti-aligned.  The top-left-most points
(i.e. \(e_{2,d}>e_{1,d}\)) all escape resonance whenever \(q<1\).  This is
due to the fact that the inner planet, \(m_1\), is now smaller than the
outer planet.  As we saw earlier in Section \ref{sec:org585941d}, the
inner test particle case is plagued by resonance instabilities.
Similar instabilities likely contribute to disrupting the apsidal
alignment process in the \(q<1\) case.

In Figure \ref{fig:escapeex3}, we display the results of a system which
is identical to Figure \ref{fig:drivingex} besides setting the mass
ratio to \(q=0.5\), reversing the migration direction, and modifying the
dissipation timescales appropriately.  The two planets convergently
migrate in resonance, but librations in the period ratio grow in
amplitude over time. Eventually, the planets escape resonance. The
inner planet \(m_1\) is kicked outwards and \(m_2\) continues migrating
inwards until the planets are orbit crossing.

\subsection{Test particle limit}
\label{sec:orgf23a5a9}
Whether or not the test particle becomes aligned with the planet can
be understood as a competition between the equilibrium value of \(e\), which
we denote by \(e_{\rm eq}\),
and the parameter \(e_p\). The comparable mass case can then be thought
of as a superposition of this mechanism since the perturbation
Hamiltonian in equation \eqref{eq:Hpert1} is linear with coupling
coefficient \(q\). The test particle
analysis simplifies the problem considerably because the vector
\(\mathbf{e}_p\) remains constant in time. 

The results of Section \ref{sec:org2071b06} can be applied to the
test particle scenario by taking the limit as \(q\) approaches \(0\) or
\(\infty\) while holding the orbital elements of the massive perturber
constant.  All expressions will use either the subscripts \(1\)
and \(2\) for \(0<q<\infty\) or \(p\) and no subscript for \(q=0,\infty\).
The resonant angle \(\hat\theta\) is given by
\begin{align}
  \tan\hat\theta = \frac{e\sin\theta}{e\cos\theta + \abs{f_2/f_1}e_p},
\end{align}

\noindent which is the the test particle limit of equation
\eqref{hattheta} assuming \(\varpi_p=0\).

The two equilibrium configurations we observe for a test particle, as
summarized in Figure \ref{fig:tp-grid-ext}, are as follows: The first is
circulation of the particle's perihelion while \(\theta\) librates (open
circles).  The second is alignment of the particle's perihelion with
the massive planet's perihelion while \(\theta\) circulates and only
\(\hat\theta\) librates (closed circles).  In
\citet{moutamid14_coupl_between_corot_lindb_reson}, \(\theta\) is the
apsidal co-rotation resonance (ACR) and \(\hat\theta\) is the Lindblad
eccentric resonance (LER).

The internal and external test particle limits are largely analagous,
and so we focus on the external limit, with \(q\) approaching infinity.
The massive inner planet has constant \(m_1=m_p\), \(a_1=a_p\), \(e_1=e_p\),
and \(\varpi_1=\varpi_p\). The test particle has \(m_2=m\to0\), \(a_2=a\),
\(e_2=e\), and \(\varpi_2=\varpi\).  We will now describe the phase space
disruption induced by \(e_p\), which splits the resonances \(\theta\) and
\(\hat\theta\). 

We denote the quantity \(S_2/\Lambda_p\) by \(S\), which gives
\begin{align}
S =  f_2^2e_p^2 - \frac{2\abs{f_1f_2}}{q\sqrt{\alpha}}e_pe\cos(\varpi_1-\varpi_2)
+ \frac{f_1^2}{q^2\alpha}e^2.
\end{align}

\noindent
If \(m\) is in \(\theta\), its eccentricity
in the limit behaves as \(e=e_p/q\to0\) (from equation \eqref{dotdpom}).
By sending \(e\to0\), we may compute
\begin{align}
S_{\in\theta}  \equiv \lim_{q\to\infty} S = f_2^2e_p^2, \\
\end{align}

\noindent where the subscript \(\in\theta\) denotes that the
test particle remains in the resonance \(\theta\).  On the other hand,
if the test particle is only trapped into \(\hat\theta\), \(e\) is no longer
constrained to the line \(e/e_p \approx 1/q\).  Then, to calculate the
corresponding test particle limit for \(S_{\notin\theta}\), we may treat
the coordinate \(\Lambda_p = q\sqrt{\alpha}\) as a constant (which is
equivalent to setting the units). Without loss of generality, we
choose \(\Lambda_p=1\). We may now compute
\begin{align}
  S_{\notin\theta} &= f_2^2e_p^2 -
                     \frac{2\abs{f_1f_2}}{\Lambda_p} e_pe\cos(\varpi_1-\varpi_2) +
                     \frac{f_1^2}{\Lambda_p^2} e^2 \\
                   &=f_2^2e_p^2 -
                     2\abs{f_1f_2} e_pe\cos(\varpi_1-\varpi_2) +
                     f_1^2 e^2
\end{align}

\noindent
Complete separation of the \(\theta\) and
\(\hat\theta\) resonances occurs whenever the quantity
\((S_{\notin\theta}-S_{\in\theta})/S_{\in\theta}\gg 1\).  The
\((e_p,e)\text{-phase}\) space should then be split between apsidal
alignment and circulation near the line defined by
\begin{align}
S_{\in\theta} &\simeq S_{\notin\theta}-S_{\in\theta} \label{eq:linedef}\\
\Rightarrow f_2e_p^2 &\simeq -2\abs{f_1f_2}e_pe\cos(\varpi_1-\varpi_2) + f_1^2e^2 \label{eq:linedefcoord}.
\end{align}

Up until now, we have been considering an analysis of the \(j:j+1\) MMR
Hamiltonian manifold without any external dissipative forces acting on
the system. However, we may still test equation \eqref{eq:linedef} on
our time-dependent, dissipative integrations for test particles in
Section \ref{sec:org585941d}.  As a first order approximation, we
parameterize the \((e_p,e)\text{-phase}\) phase by \(e\sim e_{\rm eq}
\propto h\) (where the last step is from equation \eqref{eq:eeqext}).
Utilizing equation \eqref{eq:linedef}, we calculate the boundary between
apsidal alignment and circulation and plot it as the black dashed line
in Figure \ref{fig:tp-grid-ext}.  To calculate the $x$-intercept, we
assumed \(\Delta\varpi=0\) in equation \eqref{eq:linedefcoord}.  Even though \(e\) oscillates for apsidally
aligned equilibria (e.g., in the bottom left of Figure
\ref{fig:tp-align}, \(e\) oscillates between \(0<e\lesssim 0.08\) in
equilibrium), our prescription describes the simulation results well.

\section{Conclusion}
\label{sec:org649407b}
The standard picture of MMR capture in the literature robustly leads
to capture into both the \(\theta_1\) and \(\theta_2\) resonances for
comparable masses.  The larger planet loses eccentricity to the
smaller planet and the system will always reach an equilibrium with
\(\theta_1\) and \(\theta_2\) librating over a long enough timescale.
This leads to apsidal anti-alignment due to the fact that
\(\Delta\varpi=\theta_1-\theta_2=180^\circ\) in resonance, as we found
in Section \ref{sec:org16ae390}.  However, there are
observed exoplanets near resonance which have
\(\Delta\varpi\approx0^\circ\), such as K2-19.  In this paper, we
investigated if resonant dynamics could align \(\varpi_1\) and
\(\varpi_2\).

This investigation began by analyzing the problem of a test particle
in the vicinity of an MMR with a planet of mass \(m_p\) (Section \ref{sec:org585941d}).  Here, we briefly reviewed the textbook treatment
of the CR3BP and its extension, the ER3BP.  This formulation of the
problem allowed us to adjust \(e_p\) independently to isolate its
effect.  We found that apsidal alignment occurs for large values of
\(e_p\) comparable to the aspect ratio. These results can be summarized
as a competition between the eccentricity \(e_p\) versus the magnitude
of the equilibrium eccentricity naturally induced by the disk's
damping forces.

Our test particle results informed our analysis of how apsidal
alignment may arise in the case of \(q\sim\mathcal{O}(1)\) and prompted
the introduction of an eccentricity driving force for the comparable
mass case in Section \ref{sec:org34f950f}.  While previous studies typically
assume that the eccentricity damps to zero, theoretically this is not
always the case, as several mechanisms have been proposed that drive
eccentricity in young exoplanetary systems rather than damp it
\citep{ragusa17_eccen_evolut_durin_planet_disc_inter,goldreich03_eccen_evolut_planet_gaseous_disks,teyssandier17_secul_evolut_eccen_protop_discs}.

For comparable mass systems (\(q\sim \mathcal O(1)\)) near a \(j:j+1\)
MMR, we observed apsidal alignment for strong enough damping forces
away from the phase space line \(e_2/e_1 \approx q\). However,
the process of apsidal alignment is likely chaotic, because
there are many combinations of damping forces, initial conditions,
and mass ratios which lead to disruption of capture.
In fact, we
have reached two conclusions throughout this paper:
\begin{enumerate}[label=\arabic*.]
  \item An external eccentricity driving force can produce
apsidal alignment if the system remains captured into resonance.
  \item Eccentricity driving forces can prevent the libration of $\theta_1$, $\theta_2$ and/or $\hat\theta$.
\end{enumerate}

\noindent

The outcome of planet migration and MMR capture is
clearly very sensitive to the system's parameters.  For the four
combinations of \(q=0.5,2\) and initial period ratios \(P_2/P_1=1.5,1.55\)
in Figure \ref{fig:Rhat-grid}, capture is not always likely. In fact,
for \(q=0.5\) and \(P_2/P_1=1.5\), none of the integrations lead to
apsidal alignment, and fewer simulations are captured than escape for
\(q=0.5\) and \(P_2/P_1=1.55\).  The stability and capture probability are
likely complicated functions of the initial conditions and other
system parameters, and so we leave a more detailed analysis to future
studies.

This paper represents only a preliminary investigation into the
effects of eccentricity driving in mean motion resonant systems. Our
results come with several caveats, the first of which is that our
model utilizes constant dissipative timescales. In reality, the
coupling between the disk and planet is a function of eccentricity,
location in the disk, and disk profile
\citep{cresswell_evolution_2006,cresswell_three-dimensional_2008}.  Full
hydrodynamic simulations are the most accurate method for calculating
planet-disk interactions but are too computationally expensive for the
scope of this paper.  Nevertheless, the apsidal alignment in K2-19b
and c, along with their moderate eccentricities and the fact that we
observe \(\hat\theta\) to librate \citep{petit_resonance_2020}, therefore
suggests that the planets could have interacted with an eccentricity
driving force in the past.

Both of these conclusions are also relevant to the distribution of
period ratios in the Kepler catalogue.  On the one hand, if two
planets are slightly wide of the nominal period ratio, they could
still be in resonance if they are observed to be eccentric \citep[e.g., ][]{petit_resonance_2020}  We also
found evidence that some systems remain locked slightly wide of
\(P_2/P_1=1.5\) for long timescales even after exiting the resonance.
These two configurations could comprise part of the overabundance of
observed Kepler systems with period ratios slightly larger than
\(j:j+1\). They would also simultaneously comprise part of the
underabundance at exactly \(P_2/P_1 = (j+1)/j\), simply because exactly
resonant systems can be pushed to slightly larger period ratios
whenever the disk drives the planets eccentricities.
%\twocolumn
\bibliography{references}
\bibliographystyle{mnras}
\clearpage
\onecolumn
\appendix
\section{Reducing the Hamiltonian to a single degree of freedom}
\label{sec:org34caf95}
\subsection{Scaling the Hamiltonian}
\label{sec:org432b75b}
The Hamiltonian for two comparable mass planets near a first order \(j:j+1\)
resonance is
\begin{align}
  H = -\frac{G M m_{1}}{2 a_{1}}-\frac{G M m_{2}}{2 a_{2}}
                 -\frac{G m_{1} m_{2}}{a_{2}}
                  \left[
                  f_{1} e_{1} \cos \theta_{1} 
                  +f_{2} e_{2} \cos \theta_{2}\right].
\end{align}

\noindent Define \(m_{\rm tot} = m_1+m_2\) and \(a_0\) to be the
scale length of the problem.  We will then scale the Hamiltonian by
\(H_0 = GMm_{\rm tot}/a_0\), the time by the frequency \(\omega_0 =
\sqrt{GM/a_0^3}\), and the canonical momenta by \(\Lambda_0 = m_{\rm
tot} \sqrt{GMa_0}\).  The dimensionless Hamiltonian \(\mathcal{H}\) is
then
\begin{align}
  \mathcal{H} \equiv \frac{H}{H_0}
  = -\frac{m_1/m_{\rm tot}}{2a_1/a_0}
    -\frac{m_2/m_{\rm tot}}{2a_2/a_0}
  -\frac{\tilde m}{M (a_2/a_0)}\left[
    f_1e_1\cos\theta_1+f_2e_2\cos\theta_2
    \right],
\end{align}

\noindent
where \(\tilde m = m_1m_2/m_{\rm tot}\) is the reduced mass.
The canonical momenta then become
\begin{align}
  \Lambda_1 &= \frac{m_1}{m_{\rm tot}}\sqrt{\frac{a_1}{a_0}} \\
  \Lambda_2 &= \frac{m_2}{m_{\rm tot}}\sqrt{\frac{a_2}{a_0}} \\
  \Gamma_1 &= \frac{m_1}{m_{\rm tot}}\sqrt{\frac{a_1}{a_0}}
             \left(1-\sqrt{1-e_2^2}\right) \\
  \Gamma_2 &= \frac{m_2}{m_{\rm tot}}\sqrt{\frac{a_2}{a_0}}
             \left(1-\sqrt{1-e_2^2}\right)
\end{align}

\noindent
Restoring \(\mathcal{H}\) with these momenta, we have
\begin{align}
\label{eq:H_1}
  \mathcal{H}
  = -\frac{q^3}{2(1+q)^3 \Lambda_1^2}
    - \frac{1}{2(1+q)^3\Lambda_2^2}
   - \frac{\tilde\mu}{(1+q)^2 \Lambda_2^2}\left[
    f_1\sqrt{\frac{2\Gamma_1}{\Lambda_1}}\cos\theta_1
    +f_2\sqrt{\frac{2\Gamma_2}{\Lambda_2}}\cos\theta_2
    \right],
\end{align}

\noindent where we have defined \(\tilde\mu=\tilde m/M\) to be
the reduced mass ratio.  and the \(\theta_i\) are conjugate to
\(\Gamma_i\).  For the limiting cases of \(q\to \infty\) (\(m_2=0\)) or
\(q\to 0\) (\(m_1=0\)), the ratio \(\mathcal{H}/\Lambda_i\) reduces to the
standard test particle Hamiltonian found in \citet{murray_solar_2000} if
the limits are taken along constant perturber mass, eccentricity, and semimajor axis
\citep[e.g.][]{moutamid14_coupl_between_corot_lindb_reson}
\subsection{Reducing rotation}
\label{sec:org59d6d2b}
Now, we would like to find the momenta conjugate to the fast
coordinates \(\lambda_i\) while keeping the slowly varying \(\theta_i\).
A canonical transformation preserves the form
\begin{align}
  \label{eq:dH} 
  d\mathcal{H}
  &= \Lambda_1 d\lambda_1+\Lambda_2d\lambda_2
    + \Gamma_1d\gamma_1+\Gamma_2d\gamma_2\nonumber\\
  &= \Gamma_1 d\theta_1 + \Gamma_2 d\theta_2
    +J_1 d\lambda_1+J_2d\lambda_2 .
\end{align}

\noindent
We can solve the set of equations in \eqref{eq:dH} for
\begin{align}
\label{eq:J1}
J_1 &= \Lambda_1 + j(\Gamma_1+\Gamma_2)\\
\label{eq:J2}
J_2 &= \Lambda_2 - (j+1)(\Gamma_1+\Gamma_2),
\end{align}

\noindent where \(\Gamma_i\) and \(J_i\) are conjugate to
\(\theta_i\) and \(\lambda_i\), respectively.
The coordinates \(\lambda_1\) and \(\lambda_2\)
no longer appear in the Hamiltonian,
which means \(J_1\) and \(J_2\) are constants of motion and
equation \eqref{eq:H_1} may be written
in the following form:
\begin{align}
\label{eq:H_2}
  \mathcal{H}
  = \mathcal{H}_0(\Gamma_1+\Gamma_2; J_1, J_2, q)
                  + \mathcal{H}_{\rm pert}(\theta_1, \theta_2, \Gamma_1,\Gamma_2; J_1, J_2, q),
\end{align}

\noindent
where
\begin{align}
  \label{eq:H01}
  \mathcal{H}_0(\Gamma_1+\Gamma_2; J_1, J_2, q)
  = -\frac{q^3}{2(1+q)^3(J_1-j(\Gamma_1+\Gamma_2))^2}
  -\frac{1}{2(1+q)^3(J_2+(j+1)(\Gamma_1+\Gamma_2))^2} 
\end{align}

\noindent
and
\begin{align}
  \label{eq:Hpert1}
  \mathcal{H}_{\rm pert}(\Gamma_1,\Gamma_2; J_1, J_2, q)
  = -\frac{\tilde\mu}{(1+q)^2(J_2+(j+1)(\Gamma_1+\Gamma_2))^2}
  \left[
    f_1\sqrt{\frac{2\Gamma_1}{J_1 - j(\Gamma_1+\Gamma_2)}}\cos\theta_1
  +f_2\sqrt{\frac{2\Gamma_2}{J_2 + (j+1)(\Gamma_1+\Gamma_2)}}\cos\theta_2
    \right].
\end{align}

\noindent We have \(\Gamma_i \ll \Lambda_i\) for small
eccentricities.  Under this assumption, we may drop terms smaller than
\(\mathcal{O}(\Gamma_i^2/\Lambda_i^4)\).  Equation \eqref{eq:H01} becomes
\begin{align}
  \label{eq:H02}
  \mathcal{H}_0
  = \mathcal C_0(q, J_1, J_2) -\frac{1}{(1+q)^3}\left[
     \frac{q^3}{2\Lambda_1^2} + \frac{1}{2\Lambda_2^2}
   + 2\left(
     \frac{jq^3}{\Lambda_1^3} - \frac{(j+1)}{\Lambda_2^3}
     \right)(\Gamma_1+\Gamma_2)
   -\frac32\left( 
     \frac{jq^3}{\Lambda_1^4} - \frac{(j+1)}{\Lambda_2^4}\right)
     (\Gamma_1+\Gamma_2)^2
     \right],
\end{align}

\noindent
where \(\mathcal C_0\) is a constant of resonance which depends on initial conditions.
Hence, we leave \(\mathcal C_0\) out of the following calculations.
Equation \eqref{eq:Hpert1} reduces to its original form
\begin{align}
\label{eq:H_3}
  \mathcal{H}_{\rm pert}
  =-\frac{\tilde\mu}{(1+q)^2\Lambda_2^2}
  \left[
  f_1\sqrt{\frac{2\Gamma_1}{\Lambda_1}}\cos\theta_1
  +f_2\sqrt{\frac{2\Gamma_2}{\Lambda_2}}\cos\theta_2
  \right].
\end{align}

\noindent Absent any dissipation, \(\Lambda_1\) and
\(\Lambda_2\) are approximately constant in resonance.  Hence, we may
drop the first two terms in parentheses in equation \eqref{eq:H02},
leaving only the terms which include factors of \((\Gamma_1+\Gamma_2)\):
\begin{align}
  \label{eq:H03}
  \mathcal{H}_0
  = -\frac{1}{(1+q)^3}\left[
   2\left(
     \frac{jq^3}{\Lambda_1^3} - \frac{(j+1)}{\Lambda_2^3}
     \right)(\Gamma_1+\Gamma_2)
   -\frac32\left( 
     \frac{jq^3}{\Lambda_1^4} - \frac{(j+1)}{\Lambda_2^4}\right)
     (\Gamma_1+\Gamma_2)^2
     \right].
\end{align}

Following \citep{henrard86_reduc_trans_apocen_librat}
\citep[or equivalently][]{wisdom_canonical_1986}, let \(\v X\) be the
cartesian formulation
\begin{align}
  \v X &= (x_1, x_2, X_1, X_2)\nonumber\\
  &= (\sqrt{\Gamma_1}\cos\theta_1, \sqrt{\Gamma_2}\cos\theta_2,
    \sqrt{\Gamma_1}\sin\theta_1, \sqrt{\Gamma_2}\sin\theta_2)
\end{align}

\noindent 
Define
\begin{align}
    g_1 &= f_1\sqrt{\frac{2}{\Lambda_1}} \\
    g_2 &= f_2\sqrt{\frac{2}{\Lambda_2}} \\
\end{align}

\noindent and
\begin{align}
  \mathcal{A} = \frac{1}{g_1\sqrt{g_1^2+g_2^2}}.
\end{align}

\noindent The perturbation Hamiltonian \(\mathcal H_{\rm
pert}\) becomes
\begin{align}
  \mathcal H_{\rm pert} \propto g_1 x_1 + g_2 x_2
\end{align}

\noindent
Let \(\v \Psi\) be the
counter-clockwise phase space rotation by the angle \(\psi\), where \(\tan\psi=
g_2/g_1\),
\begin{align}
  \v \Psi =  \mathcal{A}
  \begin{pmatrix}
    g_1 & g_2 \\
    -g_2 & g_1 
  \end{pmatrix}.
\end{align}

\noindent The block matrix
\begin{align}
  \v M =
  \begin{pmatrix}
    \v \Psi & \v 0 \\
    \v 0 & \v \Psi
  \end{pmatrix}
\end{align}

\noindent is symplectic \citep{goldstein_classical_2000}.
The coefficients \(g_i\) depend weakly on the semimajor axis ratio
\(\alpha\), and so \(\v M\) only represents a canonical transformation if
\(\alpha\) is stationary or varying slowly, which is a good
approximation for the systems considered in this paper.

Define the coordinates
\begin{align}
   \v W = (w_1, w_2, W_1, W_2) \equiv \v M \v X.
\end{align}

\noindent so that \(w_1 = (g_1 x_1 + g_2 x_2)\).  Hence,
\(\mathcal H_{\rm pert}\propto w_1\) only.  Finally, we revert the \(\v
W\) set back to polar coordinates
\((\hat\theta_1,\hat\theta_2,S_1,S_2)\), so that \(\mathcal H_{\rm pert}\propto S_1\cos\hat\theta_1\)
only.
The new resonance angle is
given by the equation
\begin{align}
\label{hattheta}
  \tan\hat{\theta}_1 = \frac{W_1}{w_1} = \frac{f_1 e_1\sin(\theta_1)
  + f_2e_2\sin(\theta_2)}{f_1e_1\cos(\theta_1) + f_2e_2\cos(\theta_2)}.
\end{align}

\noindent
and is conjugate to the momentum
\begin{align}
  S_1 = w_1^2 + W_1^2 = f_1^2e_1^2
  + 2f_1f_2e_1e_2\cos(\varpi_1 - \varpi_2) + f_2^2e_2^2.
\end{align}

\noindent
The key characteristic of this action-angle pair is that it does not depend
on the planetary mass ratio \citep[e.g.][]{deck13_first_order_reson_overl_stabil}.

After the reducing rotation, neither \(\mathcal H_0\) nor \(\mathcal
H_{\rm pert}\) depend on \(\hat\theta_2\), and so its conjugate momentum,
\(S_2\), is a constant of resonance:
\begin{align}
\label{eq:appS2deriv}
  S_2 = w_2^2 + W_2^2 = q\sqrt{\alpha}f_2^2e_1^2
-2\abs{f_1f_2}e_1e_2\cos(\varpi_1-\varpi_2) + \frac{f_1^2}{q\sqrt\alpha}e_2^2
,
\end{align}

\noindent
where \(\alpha=a_1/a_2\) is the ratio of semimajor axes.

Because the angle \((j+1)\lambda_2-j\lambda_1\) is
invariant under rotations of phase space,
it can be shown that
\begin{equation}
  \hat{\theta}_1 = (j+1)\lambda_2-j\lambda_1
  - \hat\varpi_1,
\end{equation}

\noindent
where
\begin{equation}
\hat\varpi_1 = \frac{\abs{f_1} e_1\sin(\varpi_1) -
  \abs{f_2}e_2\sin(\varpi_2)} {\abs{f_1}e_1\cos(\varpi_1) -
  \abs{f_2}e_2\cos(\varpi_2)}.
\end{equation}

\noindent
so that \(\hat\varpi_1\) now plays the role of \(\varpi_i\) as it appears in \(\theta_i\).

The sum
\begin{align}
  \Gamma_1 +\Gamma_2 = x_1^2+x_2^2 + X_1^2 + X_2^2
  = w_1^2+w_2^2 + W_1^2 + W_2^2 = S_1 + S_2
\end{align}

\noindent
is preserved, and so the form of \(\mathcal H_0(\Gamma_1+\Gamma_2)\)
is preserved as well.

From here on out, we will denote \(\hat\theta_1\) and \(S_1\) as
\(\hat\theta\) and \(\hat S\) to indicate that they are the single pair of
dynamical variables.  We could just as well have carried out this
analysis with \(\hat\theta_2\) and \(S_2\), but \(\hat S\) has the following
geometric interpretation:
\begin{align}
  \hat S = \left\lvert \abs{f_1}\mathbf{e}_1 - \abs{f_2}\mathbf{e}_2\right\rvert^2,
\end{align}

\noindent
where the \(\mathbf{e}_i\) are the Runge-Lenz vectors with magnitude
\(e_i\) in the direction of \(\varpi_i\).  

Altogether, we arrive at the following Hamiltonian after
dropping constant terms:
\begin{align}
  \label{eq:HShat}\mathcal H(\hat \theta, \hat S) &= \mathcal H_0(\hat S) + \mathcal H_{\rm pert}(\hat \theta, \hat S) \\
  \mathcal H_0
  &= \left( 3\mathcal NS_2 -2\mathcal M\right) \hat S
    + \frac32 \mathcal N \hat S^2 \\
  \mathcal H_{\rm pert}
  &= - q\tilde\mu\mathcal K\sqrt{\hat S}\cos\hat\theta
\end{align}

\noindent This form is valid for \(q\in (0,\infty)\), and is
naturally extended to the case of an outer test particle by taking the
limit \(q\to\infty\).  The extra factor of \(q\) in front of \(\mathcal K\)
is so \(q\tilde\mu\) reduces to \(q\tilde\mu\to q\mu_2 \to \mu_1\) in the
\(m_2=0\) test particle limit.  The coefficients in this case are given
by
\begin{align}
  \mathcal M
  &= \frac{q^3}{(1+q)^3}\frac{j-(j+1)\alpha^{3/2}}{\Lambda_1^3}\\
  \mathcal N
  &= \frac{q^3}{(1+q)^3}\left(
    \frac{j}{\Lambda_1} - \frac{(j+1)\alpha^{3/2}}{\Lambda_2}
    \right)\frac{1}{\Lambda_1^3}\\
  \mathcal K
  &= \frac{q^2}{(1+q)^2}
    \frac{\alpha^{3/2}\Lambda_2}{f_1^2\sqrt{1+\alpha^{1/2}f_1^2/f_2^2}\Lambda_1^{5/2}}.\\
\end{align}

A similar representation of \(\mathcal H, \mathcal M\), \(\mathcal N\),
and \(\mathcal{K}\) also exists for which the limit \(q\to 0\) converges
to the case of an inner test particle by exchanging the necessary
factors of \(q\) for \(\Lambda_1/\Lambda_2 = q\sqrt\alpha\).
\end{document}