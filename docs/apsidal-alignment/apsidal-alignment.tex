% Created 2021-11-11 Thu 15:04
% Intended LaTeX compiler: pdflatex
\documentclass[usenatbib]{mnras}
\usepackage[utf8]{inputenc}
\usepackage[T1]{fontenc}
\usepackage{graphicx}
\usepackage{grffile}
\usepackage{longtable}
\usepackage{wrapfig}
\usepackage{rotating}
\usepackage[normalem]{ulem}
\usepackage{amsmath}
\usepackage{textcomp}
\usepackage{amssymb}
\usepackage{capt-of}
\usepackage{hyperref}
\usepackage{caption}
\usepackage{subcaption}
\usepackage{pdfpages}
\usepackage{float}
\usepackage{booktabs}
\usepackage{enumitem}
\usepackage{graphicx}
\usepackage{tensor}
\usepackage{ wasysym }
\usepackage{mathtools}
\usepackage{xcolor}
\usepackage{cancel}
\newcommand{\note}[1]{{\color{red} \large #1 }}
\renewcommand{\O}{\mathcal{O}}
\renewcommand{\d}{\partial}
\renewcommand{\v}[1]{\boldsymbol{ #1 }}
\renewcommand{\t}[1]{\tilde{ #1 }}
\newcommand{\tg}{\t{g}}
\newcommand{\vh}[1]{\hat{\boldsymbol{ #1 }}}
\newcommand{\pp}[2]{\frac{\partial #1}{\partial #2}}
\newcommand{\dd}[2]{\frac{d #1}{d #2}}
\DeclarePairedDelimiter{\abs}{|}{|}
\DeclarePairedDelimiter{\norm}{||}{||}
\DeclarePairedDelimiter{\p}{(}{)}
\DeclarePairedDelimiter{\we}{\langle}{\rangle}
\title[MMR Architecture]{Architecture of Planetary Systems in Mean Motion Resonance}
\author[Laune et al.]{
JT Laune,$^{1}$
Laetitia Rodet,$^{1}$
and Dong Lai$^{1}$
\\
$^{1}$Department of Astronomy and Space Sciences, Cornell University\\}
\date{}
\title{Architecture of Planetary Systems in Mean Motion Resonance}
\hypersetup{
 pdfauthor={Jordan Laune},
 pdftitle={Architecture of Planetary Systems in Mean Motion Resonance},
 pdfkeywords={},
 pdfsubject={},
 pdfcreator={Emacs 27.2 (Org mode 9.4.6)}, 
 pdflang={English}}
\begin{document}

\maketitle
\maketitle

\section{Introduction}
\label{sec:org9b52561}

Even before the first detection of an exoplanetary system, planets
were predicted to migrate from their initial position due to
interactions with their natal protoplanetary disk
\cite{lin79_tidal_torques_accret_discs_binar,goldreich_excitation_1979,goldreich_disk-satellite_1980-1}.
The associated angular momentum and energy exchange acts to reshape
the orbital architecture during the first million years of a system's
life, in particular the planets' semimajor axes and eccentricities.
This process is collectively known as planetary migration
\cite{nelson_planetary_2018}.

When two planets' periods are related by an integer fraction \(j\):\(j+k\)
(e.g. 2:1, 3:2), mutual gravitational interactions can lock orbits
into mean motion resonances (MMRs).  MMRs are characterized by the
libration of two angles specific to the values of \(j\) and \(k\).  If
\(k=1\), i.e. \(j\):\(j+1\) for some \(j\), the gravitational interactions are
first order in eccentricity and inclination.  The resonance angles for
first order MMRs are given by
\begin{align}
\label{circangles1}
 \theta_1 &= (j+1)\lambda_2 - \lambda_1 - \varpi_1 \\
\label{circangles2}
 \theta_2 &= (j+1)\lambda_2 - \lambda_1 - \varpi_2.
\end{align}

\noindent The mean longitudes of the inner and outer planets
are denoted by \(\lambda_1\) and \(\lambda_2\), respectively, and
\(\varpi_1\) and \(\varpi_2\) denote their longitudes of perihelion.  If
the angles \(\theta_1\) and \(\theta_2\) librate around \(\pi\) and \(0\),
respectively, the system is said to be in resonance.  As a planet
migrates, its resonance locations sweep across the disk and capture
planets into resonant pairs and chains.  The MMR equilibrium is
governed by various conditions which are well studied in the
literature
\cite{henrard_second_1983,deck_migration_2015,goldreich_overstable_2014,xu_migration_2018,henrard_second_1983}.
The resonant pair's path across the disk, characterized by the
direction, distance traveled, and migration and eccentricity damping
timescales, determines the final orbital architecture for the system
\cite{cresswell_evolution_2006,cresswell_three-dimensional_2008}.

In the past 40 years, exoplanet surveys have given us a wealth of
systems to test our understanding of disk migration and resonant
dynamics.  Thousands of exoplanet systems have been discovered, and,
although many resonant systems have been found, the proportion of such
systems is low compared to theoretical expectations based on disk
migration \cite{fabrycky_architecture_2014}.  In addition to the paucity
of planets in MMRs, there also seems to be an overabundance of planets
just outside of mean motion resonances, with period ratios slightly
larger than 2 or 3/2
\cite{fabrycky_architecture_2014,choksi_sub-neptune_2020}.  A system's
orbital architecture can reveal details about its history.
Commensurate period ratios may indicate past resonant interactions
within the planetary system, but the present state of the resonance
can only be determined by analyzing the behavior of the angles in
equations \eqref{circangles1} and \eqref{circangles2}.  Furthermore, since
these angles are related by \(\theta_2-\theta_1=\varpi_1-\varpi_2\),
this implies that a resonant pair of planets will have
\(\Delta\varpi\equiv \varpi_1-\varpi_2\approx \pi\).

NASA's \emph{Kepler} mission observed several planet pairs near a
first-order resonance with constraints on \(\Delta\varpi\), such as
Kepler-88 \cite{weiss_discovery_2020} and Kepler-24
\cite{antoniadou_exploiting_2020}. Both of these systems are apsidally
anti-aligned (\(\Delta\varpi\approx180^\circ\)).  Kepler-9b and
Kepler-9c are near to the 2:1 resonace and apsidally anti-aligned, but
their angles \(\theta_1\) and \(\theta_2\) likely circulate.  However, a
puzzling resonant architecture has been discovered recently in the
system K2-19.

NASA's \emph{K2} mission initially discovered the planets K2-19b and c
near a 3:2 period ratio (\(P_b=7.9\) d, \(P_c=11.9\) d), and later on
planet K2-19d at \(P_d=2.5\) d
\cite{howell14_k2_mission,armstrong15_one_closes_exopl_pairs_to,sinukoff16_eleven_multip_system_fromk_masses}.
K2-19 is a solar-type star (\(M_*=0.88M_\odot\)), planet b has
\(M_{b}=10.8 M_{\oplus}\), and c has \(M_{c}<10M_{\oplus}\).  Both outer
planets had moderate eccentricities \(e_b\approx e_c\approx 0.2\).
Further observations revealed K2-19b and c to be apsidally
\emph{aligned}, with \(\Delta\varpi\equiv \varpi_c-\varpi_b \approx
2\pm 2^\circ\) \cite{petigura_k2-19b_2020}.  Hence, the K2-19 system
poses a problem for the conventional understanding of planet migration
and resonance capture.

Investigating how K2-19 could have formed with \(\Delta\varpi=0\)
through resonance capture will offer us an insight into its dynamical
history as well as a better understanding of the genesis of extrasolar
orbital configurations.  In this paper, we review the analytically
simple migration model commonly used in the literature.  In Section
\ref{sec:org0467503}, we present the classical picture of resonant capture
and explore the parameter space for the coupling between the resonant
eccentricities and the protoplanetary disk. We fail to find any disk
conditions which robustly lead to apsidal alignment.  In Section
\ref{sec:org3c31088}, we design a toy model which relies on eccentricity
diriving to a finite, non-zero value and argue that it can reproduce
the apsidal alignment between K2-19b and c. We demonstrate that the
orbital architecture of K2-19 can be explained by an external force
which pumps the eccentricities beyond their equilibrium values.

\section{Standard Picture of Resonance Capture}
\label{sec:org0467503}
Planets embedded within a protoplanetary disk interact gravitationally
with the gas and lose angular momentum, leading to inward migration
towards the central star.  Disk torques vary with planet mass as well
as across semimajor axis; large outer planets lose angular momentum
quickly and sweep up inner planets into MMRs
\cite{tanaka_three-dimensional_2004,xu_migration_2018}.  In some
cases, the planet can gain angular momentum and migrate away from the
primary.

In this paper, we will ignore the detailed physics of planet-disk
interactions and instead implement a proxy for dissipative forces
parametrized by the eccentricity damping and migration timescales,
denoted by \(T_{e,i}\) and \(T_{m,i}\) for \(i=1,2\).  We will denote all
quantities relevant to the inner planet with the subscript \(1\), and the
outer with \(2\).  The equations of motion for disk effects are:
\begin{align}\label{eq:disforce}
  \frac{\dot{e}_i}{e_i} &= -\frac{1}{T_{e,i}} \\
  \frac{\dot{a}_i}{a_i} &= -\frac{1}{T_{m,i}} -\frac{2e_i^2}{T_{e,i}}.
\end{align}

\noindent This approximate migration model has been proposed
by \citet{goldreich_disk-satellite_1980-1} and is used in most studies
of MMR capture
\cite[e.g.][]{goldreich_overstable_2014,xu_migration_2018} In our
notation, \(T_{m,i}>0\) \((<0)\) denotes inward (outward) migration.

We'll consider two planets with masses \(m_1\) and \(m_2\) around a star
of mass \(M_*\). We define \(\mu_i=m_i/M_*\) to be their mass fractions
and set \(M_*=1M_\odot\) throughout. For typical, thin disk profiles, we
have
\cite{tanaka_three-dimensional_2004,cresswell_three-dimensional_2008,xu_migration_2018}
\begin{align}
  \frac{T_{e,1}}{T_{e,2}}&= \frac1q\\
  T_{e,i}&=3.46 h^2 T_{m,i},
\end{align}

\noindent where \(q=m_1/m_2\) is the mass ratio and \(h\) is the
aspect ratio of the disk.  To scale the dissipation times in the
integrations, we choose a parameter \(T_{e,0}\) and set
\begin{align}
  T_{e,1}&=T_{e,0}\sqrt{q}\\
  T_{e,2}&= T_{e,0}/\sqrt{q}.
\end{align}

\noindent
We must have \(1/T_{m,1} - 1/T_{m,2} > 0\) for convergent
(i.e. \(\abs{a_1-a_2}\) is shrinking) inward migration, and vice versa
for outward migration. Hence, for \(q>1\), we set \(T_{m,i}> 0\); for
\(q<1\), we set \(T_{m,i}<0\).  Unless noted otherwise, we choose \(h=0.1\)
and \(T_{e,0}=1000~\rm{years}\).

\subsection{Resonant Hamiltonian}
\label{sec:org88f9d91}
\begin{figure*}
  \centering
  \includegraphics[width=0.7\textwidth]{{./standard-example-h-0.1-Tw0-1000}.png}
  \caption{Standard MMR capture process for $h=0.1$ and $q=2$. The
    outer planet $m_2$ starts wide of resonance and is captured near
    $t=2000$ yrs, after which the two angles $\theta_1\to180^\circ$
    and $\theta_2\to 0^\circ$.  While in resonance, the $e_i$ values
    are driven to equilibrium and the periapses are antialigned.}
  \label{fig:standardex}
\end{figure*}
When two planets have commensurate period ratios, \(j\):\(j+k\) where \(j,k\)
are integers, their gravitational interactions may lock them into a
mean motion resonance (MMR).  As young planets migrate within their
disk, if the migration is convergent, they cross MMR period ratios and
may be captured. In our paper, we will be considering only first order
MMRs, denoted by \(j\):\(j+1\), which occur where \(n_2/n_1 = j/j+1\), where
\(n_1,n_2\) denote the planets' mean motions.

The Hamiltonian of a system with two planets near a first order MMR is
\cite{murray_solar_2000}:
\begin{align}
\label{hamiltonian}
  H_{\rm kep} = & -\frac{G M m_{1}}{2 a_{1}}-\frac{G M m_{2}}{2 a_{2}}\\
  H_{\rm res} = & -\frac{G m_{1} m_{2}}{a_{2}}
                  \left[
                  f_{1} e_{1} \cos \theta_{1} 
                  +f_{2} e_{2} \cos \theta_{2}\right]\\
  H_{\rm sec} = &-\frac{G m_{1} m_{2}}{a_{2}}\left[f_{3} (e_1^2 + e_2^2)
                  +f_4e_1e_2\cos(\varpi_2-\varpi_1)
                  \right] \\
  H = &~ H_{\rm kep} + H_{\rm res}+ H_{\rm sec}. 
\end{align}

\noindent Here, the \(f_i\) are functions of the semimajor
axis ratio \(\alpha=a_1/a_2\) that can be found in Appendix B of
\citet{murray_solar_2000} as
\begin{align}
\label{coefficients}
  f_1 &= \frac12[2(j+1)+\alpha D]b_{1/2}^{(j+1)}(\alpha); f_1(\alpha_{2:3})\approx 2.0 \\
  f_2 &= -\frac12[-1+2(j+1)+\alpha D]b_{1/2}^{(j)}(\alpha);f_2(\alpha_{2:3}) \approx -2.5\\
  f_3 &= \frac18[2\alpha D + \alpha^2 D^2]b_{1/2}^{(0)}(\alpha); f_3(\alpha_{2:3})\approx 1.15\\
  f_4 &= \frac14[2-2\alpha D - \alpha^2 D^2]b_{1/2}^{(1)}(\alpha); f_4(\alpha_{2:3})\approx -2.0 \\
\end{align}

\noindent In our numerical integrations, we evaluate them at
the instantaneous semimajor axis ratio.  However, \(f_1\) and \(f_2\)
depend weakly on \(\alpha\), and so in our analytical treatment we may
ignore their derivatives to good approximation.  \(H_{\rm kep}\) is the
standard Keplerian Hamiltonian; \(H_{\rm res}\) the resonant
interactions between the planets of order \(\O(e_i)\); and \(H_{\rm sec}\)
the secular interactions.  The two angles are given as in equations
\eqref{circangles1} and \eqref{circangles2}.

Equation \eqref{hamiltonian} admits seven independent coupled ordinary
differential equations (\(\dot a_i, \dot e_i, \dot\theta_i,
\dot\varpi_i\)), which we may integrate together with the effects of
dissipation to simulate MMR capture.  An example of MMR capture is
given in Figure \ref{fig:standardex}.  The period ratio \(P_2/P_1\)
initially starts wide of the nominal resonance value of \(1.5\).  After
around \(2~\rm{kyr}\) of convergent migration, the planets are caught
into MMR, indicated by the stabilization of \(\theta_1\) to \(180^\circ\)
and \(\theta_2\) to \(0^\circ\).  The planets' eccentricities level off at
their equilibrium values near \(e_1\approx 0.02\) and \(e_2\approx0.04\),
and the planets become apsidally anti-aligned with
\(\varpi_1-\varpi_2\approx 180^\circ\).

In this paper, we will use the term "resonance" loosely to mean the
libration of an angle such as \(\theta_1\), \(\theta_2\), and later on
\(\hat\theta\).  We'll also use the angle itself to refer to the
resonance, i.e. the planets \(m_1\) and \(m_2\) in Figure
\ref{fig:standardex} are caught into both \(\theta_1\) and \(\theta_2\),
respectively, since those angles are librating.

During the migration phase, planets typically retain small
eccentricities. Indeed, the standard circular MMRs (angles \(\theta_1\)
and \(\theta_2\)) have resonance widths which decrease with \(e\),
and so small eccentricities are necessary for capture.  Most studies
neglect the secular terms in \(H_{\rm sec}\) because they are second
order in eccentricity.  However, if eccentricities are excited,
secular terms play an important role, and so we include them in our
analysis.

\subsection{Equilibrium}
\label{sec:orge90fb54}
\begin{figure}
  \centering
  \begin{subfigure}[t]{0.225\textwidth}
  \includegraphics[width=1\textwidth]{{standard-eeqs-Tm2--40873-Tw0-1000}.png}
  \caption{ }
  \label{fig:standardeqecc}
  \end{subfigure}
  \begin{subfigure}[t]{0.225\textwidth}
  \includegraphics[width=1\textwidth]{{standard-pomega-Tm2--40873-Tw0-1000}.png}
  \caption{ }
  \label{fig:standardDpom}
  \end{subfigure}
  \caption{\emph{(a)} Analytical equilibrium values are plotted
    as dashed lines for various values of $q$. The points
    indicate time averaged numerical results from integrating the
    time-dependent equations of motion.  Error bars indicate the
    standard deviation of the eccentricities; most fall within
    the marker for eccentricity.  Simulations without secular
    effects showed only negligible differences, and so they were
    not included.  \emph{(b)} Same as \emph{(a)}, but for
    $\Delta\varpi$. Simulations without secular effects did show
    significant differences, and so they have been included.}
\label{fig:standard}
\end{figure}
The MMR capture in Figure \ref{fig:standardex} leads to an equilibrium
state in period ratio, resonant angles, eccentricities, and
\(\Delta\varpi\).  The Hamiltonian in equation
\eqref{hamiltonian}, including the dissipative terms, admits the
following three equations for equilibrium values of
\((e_1,e_2,\theta_1,\theta_2)\):
\begin{equation}
\label{dote1}
  \dot e_1 = \frac{\mu_2}{\alpha_2} [f_1\sin(\theta_1) - De_2 \sin(\gamma_2-\gamma_1)] - \frac{e_1}{T_{e,1}}=0
\end{equation}

\begin{equation}
\label{dote2}
  \dot e_2 = \frac{q\mu_2}{\alpha_2} [f_2\sin(\theta_2) - De_1 \sin(\gamma_1-\gamma_2)]- \frac{e_2}{T_{e,2}}=0
\end{equation}

\begin{align}
\label{dotdpom}
  \frac{d}{dt}\Delta\varpi \equiv \dot\varpi_1-\dot\varpi_2
  &= \frac{\mu_2}{\alpha_2} \left[ \frac{f_1\cos\theta_1}{\alpha_1^{1/2} e_1}
     - \frac{qf_2\cos\theta_2}{\alpha_2^{1/2}e_2}\right.\nonumber \\
  &\quad+ \left.\frac{2C}{\alpha_1^{1/2}} + \frac{De_2}{\alpha_1^{1/2} e_1}
    - \frac{2qC}{\alpha_2^{1/2}} - \frac{qDe_1}{ \alpha_2^{1/2}e_2}\right]=0.
\end{align}

Moreover, absent any dissipative or secular forces, the following quantities are
conserved:
\begin{align}
  J &= \Lambda_1\sqrt{1-e_1^2} + \Lambda_2\sqrt{1-e_2^2}\\
  G &= \frac{j+1}{j} \Lambda_1 + \Lambda_2,
\end{align}

\noindent where \(\Lambda_1 = q\sqrt{a_1/a_0}\) and
\(\Lambda=\sqrt{a_2/a_0}\) assuming the Hamiltonian has been scaled by
the quantity \(GM_*m_2/a_0\).  The quantity \(J\) is the angular momentum
of the system, and \(G\) is an integral of motion for the the
Hamiltonian \(H_{\rm kep}+H_{\rm res}\) in equation \eqref{hamiltonian}.
Define \(\eta\) to be the ratio of \(J\) and \(G\),
\begin{align}
  \eta(\alpha, e_1, e_2) &\equiv - 2(q/\alpha_0+1)\p*{\frac{J}{G}-\left.\frac{J}{G}\right|_{0}},
\end{align}

\noindent
where \(\alpha_0 = (j/(j+1))^{3/2}\) and \(\left(J/G\right|_{0}\) is
evaluated at \(e_i=0\) and \(\alpha=\alpha_0\).
Thus, we have \(\eta(\alpha_0, 0, 0)=0\) and the corresponding Taylor expansion yields
\begin{align}
  \eta \approx -\frac{q(\alpha-\alpha_0)}{j\sqrt{\alpha_0}(q/\alpha_0+1)} + q\sqrt{\alpha_0}e_1^2 + e_2^2
\end{align}

\noindent
The equation of motion for \(\eta\) is then given by
\begin{align}
\label{doteta}
  \dot\eta = \frac{q\alpha_0^{1/2}}{j(q\alpha_0^{-1}+1)}&\left[ \frac{1}{T_{m2}} - \frac{1}{T_{m1}}
      + \frac{2e_1^2}{T_{e1}}- \frac{2e_2^2}{T_{e2}} \right] \nonumber\\
    &- q\alpha_0^{1/2}\frac{2e_1^2}{T_{e1}} - \frac{2e_2^2}{T_{e2}}=0.
\end{align}

\noindent
We note that the only contribution to \(\dot{\eta}\) is from dissipative effects.

By solving the four equations \eqref{dote1} -- \eqref{dotdpom} and
\eqref{doteta} , we can calculate the equilibrium values for the system.
In the standard picture and neglecting secular terms (i.e., for small
\(e_i\)), equations \eqref{dote1} and \eqref{dote2} show
\(\sin(\theta_i)\approx 0\).  Equation \eqref{dotdpom} gives us
\(\abs{\cos\theta_i} \approx 1\) and \(\cos\theta_1 = -\cos\theta_2\), but
the solution \((\theta_1,\theta_2)=(0,\pi)\) is unstable.  Hence,
\(\theta_1\approx\pi\) and \(\theta_2\approx 0\) in equilibrium.  Since
\(\theta_1-\theta_2 = \varpi_2-\varpi_1\), we therefore see that
convergent migration produces anti-aligned periapses.  We confirm this
in the time-dependent integration in Figure \ref{fig:standardex}.
The equilibrium \(e_i\)'s and \(\Delta\varpi\)'s for comparable mass
planets \((q\in[0.5,2])\) are given in Figures \ref{fig:standardeqecc} and
\ref{fig:standardDpom}.  Analytical solutions to the equilibrium
equations are plotted as dashed lines.
We compare the analytical results to a numerical integration of the 
time-dependent differential equations from Hamiltonian
\eqref{hamiltonian} and plot the average \(e_1\), \(e_2\), and
\(\Delta\varpi\) over the last 10\% of the timespan.  These results are
calculated with outward migration for \(q>1\) and inward migration for
\(q<1\).

As we can see in Figures \ref{fig:standardeqecc} and
\ref{fig:standardDpom}, the final averaged eccentricities for \(m_1\) and
\(m_2\) go approximately as \(e_2/e_1 \sim q\). As expected, the
\(\Delta\varpi\) average values are all very close to \(\pi\). The
numerical and analytical results largely agree.  In the next two
sections, we will explore slightly modified models by varying the
ratio \(T_{e,1}/T_{e,2}\) to test whether they can
produce apsidal alignment.

\subsection{Eccentricity damping timescales}
\label{sec:org8644393}
\begin{figure}
  \centering
  \includegraphics[width=0.3\textwidth]{{./varyTe-eeqs-h-0.1-Tw0-1000}.png}
  \caption{ Equilibrium eccentricity values for a range of
    $T_{e,1}/T_{e,2}\in[0.2,10]$ are plotted for three
    different values of $q=0.5,1.0,$ and $2.0$. The points and
    errorbars are calculated in the same way as
    \ref{fig:standard}.  The dashed lines indicate analytical
    estimates for $e_i$.}
  \label{fig:eqecc}
\end{figure}

\begin{figure}
  \centering
  \includegraphics[width=0.3\textwidth]{{./varyTe-pomega-h-0.1-Tw0-1000}.png}
  \caption{Same as \ref{fig:eqecc} but for $\Delta\varpi$.}
  \label{fig:eqDpom}
\end{figure}
Up until now, we have strictly been considering the standard picture
of planet migration -- with \(T_{e,1}/T_{e,2} = 1/q\) and
\(T_{e,i}=3.46h^2T_{e,i}\) -- which always gives rise to apsidal
anti-alignment for reasonable disk conditions .( \(h\sim 0.1\),
\(T_{e,i}\sim h^2 T_{m,i}\) ) This simple parametrized model will always
fail to capture all of the complicated hydrodynamics of real
astrophysical disks. We can therefore easily expect a difference in
the ratio \(T_{e,1}/T_{e,2}\) over an order of magnitude.
We would like to determine the effects of the eccentricity damping
ratio on the equilibrium values of \(e_i\) and whether such
a change could lead to apsidal alignment.

We explore this possibility in Figures \ref{fig:eqecc} and
\ref{fig:eqDpom}. The ratio \(T_{e,1}/T_{e,2}\) varies freely between
\(0.2\) and \(10\), regardless of the mass ratio.  Initially, we attempted
to extend this range to \(T_{e,1}/T_{e,2}=0.1\), but the system
eventually escapes resonance for all \(q=0.5\), \(1\), and \(2\) and no
equilibrium is reached.  The migration timescales are set to
\(\abs{T_{m,i}}=T_{e,i}/3.46 h^2\).  For \(T_{e,1}<T_{e,2}\), then, we set
\(T_{m,i}>0\), corresponding to outward migration, and vice versa for
\(T_{e,1}>T_{e,2}\).

For comparable mass planets with \(q=0.5\), \(1\), and \(2\), varying the
ratio \(T_{e,1}/T_{e,2}\) around \(1/q\) modifies the final equilibrium
eccentricities by a roughly similar factor, as seen in Figure
\ref{fig:eqecc}. The eccentricity ratio \(e_1/e_2\) is largely unchanged,
yet the magnitudes \(e_1\) and \(e_2\) are larger for more extreme values
of \(T_{e,1}/T_{e,2}\).  The dashed lines plot the analytic results from
solving equations \eqref{dote1} -- \eqref{doteta}; these findings
reproduce the numerical results.

The corresponding values for \(\Delta\varpi\) are shown in Figure
\ref{fig:eqDpom}. In all cases, the analytic equilibrium
equations predict \(\Delta\varpi\approx 180^\circ\), and the numerical
integrations agree.  We note that the equilibrium solutions given in
Figures \ref{fig:standard} - \ref{fig:eqDpom} are not continuous across
the line \(T_{e,1}/T_{e_2} = 1\) (i.e. \(q=1\) in \ref{fig:standard}), which
is where we reverse the migration direction to ensure it is
convergent. Variations in the eccentricity damping ratio cannot
account for apsidal alignment.  

\section{Apsidal Alignment}
\label{sec:org3c31088}
As we have seen, capture into the \(\theta_1\) and \(\theta_2\) resonance
always leads to \(\Delta\varpi\approx 180^\circ\) due to their equilibrium
values being close to \(180^\circ\) and \(0^\circ\), respectively.  The apsidally
anti-aligned K2-19 system therefore poses a problem for our standard
model.  In order to match this observation, either \(\theta_1\),
\(\theta_2\), or both angles must circulate.

\subsection{Eccentricity driving forces}
\label{sec:orgb6dbc6f}
\begin{figure*}
  \centering
  \includegraphics[width=0.7\textwidth]{{driving-example-h-0.03-Tw0-1000}.png}
  \caption{Here we have set $e_{2,d}=0.3$ with $h=0.1$ and $q=2$.  After
    about 10~kyr, the system escapes the circular resonances and becomes
    apsidally aligned.}
  \label{fig:drivingex}
\end{figure*}
The planets K2-19b and c are moderately eccentric, with
\(e_{b}\approx e_c\approx 0.2\) \citep{petigura_k2-19b_2020}.
One way of escaping the circular \(\theta_i\) resonances is to
artificially drive the eccentricity of the system to larger values,
where equation \eqref{dotdpom} may be broken.  We modify the
eccentricity damping for \(m_1\) in equation \eqref{eq:disforce} to be
\begin{equation}
  \frac{\dot e_1}{e_1} = -\frac{(e_1-e_{1,d})}{T_{e,1}},
\end{equation}

\noindent so that planet \(m_1\) is  driven to
\(e_{1,d}\) with a timescale of \(T_{e,1}\).

In Figure \ref{fig:drivingex}, we demonstrate the feasibility of this
approach, where we integrate the time-dependent equations with an
eccentricity driven to \(e_{1,d}=0.1\) for \(q=2\).  We initalize the
system close to resonance, where it is caught (\(\theta_1\) and
\(\theta_2\) librate) for around 8,000 years. Between \(t=8,000\) and
\(10,000\) years, \(e_1\) and \(e_2\) grow and the system subsequently
breaks out of both the \(\theta_1\) and \(\theta_2\) resonances.  At this
point, both planets' ecentricities are excited to about \(e_i\approx
0.2\) and the planets become apsidally aligned as \(\Delta\varpi\)
librates around \(0^\circ\) with a large amplitude.  Despite the
circulation of both resonance angles, the period ratio remains locked
very close to the nominal resonance location (\(P_2/P_1= 1.5\)). The
system is caught in a different type of resonance which we will study
in the following subsection.

\subsection{Reducing the Hamiltonian}
\label{sec:orgc2c62d5}
\begin{figure}
  \centering
  \includegraphics[width=0.45\textwidth]{{./S2-conserved}.png}
  \caption{ }
  \label{fig:S2cons}
\end{figure}

\begin{figure*}
  \centering
  \includegraphics[width=0.7\textwidth]{{driving-perpendicular-example-h-0.03-Tw0-1000}.png}
  \caption{}
  \label{fig:perpex}
\end{figure*}

\begin{figure}
  \centering
  \includegraphics[width=0.45\textwidth]{{./relative-geometry}.png}
  \caption{ }
  \label{fig:relgeom}
\end{figure}
A detailed analysis of the MMR Hamiltonian \eqref{hamiltonian}
illustrates the underlying dynamics behind the capture processes in
Figure \ref{fig:drivingex} which lead to apsidal alignment.  We show
that \(\theta_1\) and \(\theta_2\) are actually subresonances of a
resonance \(\hat\theta\) which arises after transforming the system's
Hamiltonian so that it has only a single degree of freedom.  This
treatment naturally explains why \(P_2/P_1\) is locked near \(1.5\).


If we assume that the semimajor axis ratio \(\alpha\) is stationary or
varying adiabatically, we may transform the resonant Hamiltonian
\(H_{\rm Kep} + H_{\rm res}\) in equation \eqref{hamiltonian} into the
form
\begin{equation}
  \label{hhat}
  \hat H(\hat R,\hat\theta) = -3(\delta+1) \hat R + \hat R^2 - 2\sqrt{2\hat R} \cos(\hat\theta)
\end{equation}

\noindent through a series of rotations in phase space.  For
the details of these transformations, see Appendix \ref{sec:orgfd19f7f}.  The parameter \(\delta\)
quantifies the system's depth into resonance.  We do not include
\(H_{\rm sec}\) in this analysis because it is second order in
eccentricities and does not qualitatively change the outcome of our
numerical integrations.

First order MMRs are typically treated as the circular restricted
three body problem (CR3BP), where one of the planets is a test
particle and the other is on a circular orbit. These are limiting
cases of the Hamiltonian given in equation \eqref{hamiltonian} for
comparable mass planets. For an inner (outer) test particle, \(m_1=0\)
(\(m_2=0\)), \(m_2\) (\(m_1\)) is finite, and \(q\) approaches infinity
(zero).  In the CR3BP limit, the action angles are \(\theta_1\) and
\(\theta_2\) in equations \eqref{circangles1}, \eqref{circangles2}, and
\eqref{hamiltonian} for the internal and external cases, respectively.
The actions for the CR3BP are proportional to
\begin{align}
\label{eq:R2limit}
  R_1  \propto \frac{e_1^2}{\mu_2^{2/3}}
\end{align}

\noindent 
and
\begin{align}
\label{eq:R1limit}
  R_2  \propto  \frac{e_2^2}{\mu_1^{2/3}} .
\end{align}

One can see that the actions given above for the CR3BP are functions
of the test particles' eccentricities only, as the massive planets'
eccentricities are assumed to be zero.  On the other hand, for the
case where \(q>0\) is finite, both planets will be on eccentric
osculating orbits.
Hence, we expect the action \(\hat R\) in equation \eqref{hhat}
to be a function of both \(e_1\) and \(e_2\).

Define \(\v{\hat e} = \abs{f_1}\v e_1 - \abs{f_2}\v e_2\) and \(\hat e =
\abs{\v{\hat e}}\), where \(\v e_i\) is the Runge-Lenz vector, i.e. the
vector with magnitude \(e_i\) in the direction of perihelion.  The
action \(\hat R\) then takes the form \(\hat R \propto \tilde \mu
\hat e^2\), where \(\tilde\mu = m_1m_2/(m_1+m_2)\) is the reduced mass of the
planets. In order to recover the actions given above for the
appropriate CR3BP limits, we have
\begin{align}
\label{eq:R1limit}
  \hat R \propto \frac{\hat e^2}{q^{2/3}\hat\mu^{2/3}}
\end{align}

\noindent 
for \(q < 1\)
and
\begin{align}
\label{eq:R2limit}
  \hat R \propto \frac{q^{2/3}\hat e^2}{\hat\mu^{2/3}}
\end{align}

\noindent
for \(q>1\).
The action angle is \(\hat\theta\) which is given by
\begin{align}
\label{hattheta}
  \tan\hat{\theta}_1 = \frac{W_1}{w_1} = \frac{f_1 e_1\sin(\theta_1)
  + f_2e_2\sin(\theta_2)}{f_1e_1\cos(\theta_1) + f_2e_2\cos(\theta_2)}.
\end{align}

\noindent
Appendix \ref{sec:orgfd19f7f}
describes the detailed behavior of \(\hat H\), \(\hat \theta\),
and \(\hat R\) in the CR3BP limits,

\subsection{Three modes of resonance}
\label{sec:org2d0fcd6}
Under the assumption that the semimajor axes are stationary (or slowly
varying), the Hamiltonian in equation \eqref{hamiltonian} has two
degrees of freedom (dof): the angles \(\theta_1\) and \(\theta_2\).  By
transforming this Hamiltonian into the single dof form of equation
\eqref{hhat} the resonance admits the following conserved quantity:
\begin{align}
  \label{eq:s2}
  S_2 = q\sqrt{\alpha}f_2^2e_1^2
+2\abs{f_1f_2}e_1e_2\cos(\varpi_1-\varpi_2) + \frac{f_1^2}{q\sqrt\alpha}e_2^2.
\end{align}

By enforcing \(dS_2/dt = 0\) together with the assumption \(d\alpha/dt
\approx 0\), we arrive at the following equilibrium condition:
\begin{align}
  \label{eq:S2eq}
  \frac{dS_2}{dt} \approx e_1^2\left(\frac{e_1-e_{1\rm d}}{T_{e,1}}\right)
  \mathcal{S}_1
  + e_2^2\left(\frac{e_2-e_{2\rm d}}{T_{e,2}}\right)
  \mathcal{S}_2
  = 0.
\end{align}

\noindent
where
\begin{align}
  \label{eq:sS1}
  \mathcal{S}_1=&\left[
                  q^2\alpha\abs*{\frac{f_2}{f_1}}
                  + \frac{e_2}{e_1}q\sqrt{\alpha}\cos(\varpi_1-\varpi_2)
                  \right]\\
  \label{eq:sS2}
  \mathcal{S}_2=&\left[
                  \abs*{\frac{f_2}{f_1}} q\sqrt{\alpha}
                  \frac{e_1}{e_2}\cos(\varpi_1-\varpi_2) + 1
                  \right].
\end{align}

In Figure \ref{fig:S2cons}, we have plotted \(\mathcal{S}_1\) and
\(\mathcal{S}_2\) as well as \(\dot S_2/S_2\) for three different
combinations of \(e_{1d}\) and \(e_{2d}\).  The top row is for the
standard eccentricity damping case where \(e_{1d}=e_{2d}=0\).  Once the
system equilibrates, \(S_2\approx 10^{-4}\) is well conserved (top left) and
small. Both \(\mathcal{S}_1\) and \(\mathcal{S}_2\) are also close to
zero. From equations \eqref{eq:sS1} and \eqref{eq:sS1}, we see this
corresponds to \(e_2/e_1 \sim q\), as we found in Section \ref{sec:org0467503}.

The second row corresponds to the integration shown in
\ref{fig:drivingex}, where \(e_{1d}=0.1\) and \(e_{2d}=0\).  Near the
beginning, while the system is still caught in the \(\theta_1\) and
\(\theta_2\) resonances, \(\mathcal{S}_1\), \(\mathcal{S}_2\), and \(S_2\) are
small.  Once the \(\theta_i\) resonances are broken, and only
\(\hat\theta\) librates, they are excited to larger values.  The
quantities \(\mathcal{S}_1\) and \(\mathcal{S}_2\) undergo large periodic
oscillations away from zero, while \(S_2\) grows and then stabilizes at
its new equilibrium value. The system's eccentricities
oscillate in such a way to conserve equation \eqref{eq:S2cons}.

The last row presents a third mode of resonance, where \(e_{2d}=0.2\)
and \(e_{1d}=0\).  We have plotted the details of this system in Figure
\ref{fig:perpex}.  The system is immediately caught in the \(\hat\theta\)
resonance, which persists throughout the integration.  The angle
\(\theta_2\) librates with large amplitude around its resonant value of
\(0^\circ\), while \(\theta_1\) librates around \(270^\circ\) rather than
\(180^\circ\).  As a result, \(\Delta\varpi\) approaches \(90^\circ\) and so
the planets' perihelia are now perpendicular to eachother.  For this
system, \(\mathcal{S}_2\) is conserved close to 0, while \(\mathcal{S}_1\)
grows to a magnitude similar to its value in the apsidally aligned
case.

The systems in Figures \ref{fig:standardex}, \ref{fig:drivingex}, and
\ref{fig:perpex} are representative of three different modes of
resonance, ones where \(\Delta\varpi=180^\circ\),
\(\Delta\varpi=90^\circ\), and \(\Delta\varpi=90^\circ\),
respectively. These correspond to three different behaviors of the
quantities \(\mathcal{S}_1\) and \(\mathcal{S}_2\) while in resonance
under the influence of eccentricity forcing.

In Figure \ref{fig:relgeom}, we plot 

\subsection{Saturation eccentricities}
\label{sec:orgb132153}
\begin{figure}
    \centering
    \includegraphics[width=0.4\textwidth]{{./Rhat-grid}.png}
    \caption{\emph{Left:} 
\emph{Right:}}
    \label{fig:Rhat-grid}
  \end{figure}
Now that we have identified these three resonant modes, we briefly
explore the \((e_{1d},e_{2d})\) parameter space for the saturation
eccentricities for moderate values between \(0\) and \(0.3\).
In the left panel of Figure \ref{fig:Rhat-grid},
we summarize the resonant behavior for each system on a grid
within the parameter space. Here, we have excluded
runs which become unstable and escape the resonance on the timescale
of our integrations. Roughly, for \(e_{1d}\gtrsim e_{2d}\),
the system becomes apsidally aligned. For \(e_{2d} > e_{1d}\),
\(\Delta\varpi=90^\circ\).

We have plotted the averaged final eccentricities in the right
panel of Figure \ref{fig:Rhat-grid}.

\section{Conclusion}
\label{sec:org272dfcb}

\clearpage

\onecolumn
\appendix
\section{Reducing the Hamiltonian to a single degree of freedom}
\label{sec:orgfd19f7f}
\subsection{Scaling the Hamiltonian}
\label{sec:org63b789b}
The Hamiltonian for two comparable mass planets near a first order \(j:j+1\)
resonance is
\begin{align}
  H = -\frac{G M m_{1}}{2 a_{1}}-\frac{G M m_{2}}{2 a_{2}}
                 -\frac{G m_{1} m_{2}}{a_{2}}
                  \left[
                  f_{1} e_{1} \cos \theta_{1} 
                  +f_{2} e_{2} \cos \theta_{2}\right].
\end{align}

\noindent Define \(m_{\rm tot} = m_1+m_2\) and \(a_0\) to be the
scale length of the problem.  We will then scale the Hamiltonian by
\(H_0 = GMm_{\rm tot}/a_0\), the time by the frequency \(\omega_0 =
\sqrt{GM/a_0^3}\), and the canonical momenta by \(\Lambda_0 = m_{\rm
tot} \sqrt{GMa_0}\).  The dimensionless Hamiltonian \(\mathcal{H}\) is
then
\begin{align}
  \mathcal{H} \equiv \frac{H}{H_0}
  = -\frac{m_1/m_{\rm tot}}{2a_1/a_0}
    -\frac{m_2/m_{\rm tot}}{2a_2/a_0}
  -\frac{\tilde m}{M (a_2/a_0)}\left[
    f_1e_1\cos\theta_1+f_2e_2\cos\theta_2
    \right],
\end{align}

\noindent
where \(\tilde m = m_1m_2/m_{\rm tot}\) is the reduced mass.
The canonical momenta then become
\begin{align}
  \Lambda_1 &= \frac{m_1}{m_{\rm tot}}\sqrt{\frac{a_1}{a_0}} \\
  \Lambda_2 &= \frac{m_2}{m_{\rm tot}}\sqrt{\frac{a_2}{a_0}} \\
  \Gamma_1 &= \frac{m_1}{m_{\rm tot}}\sqrt{\frac{a_1}{a_0}}
             \left(1-\sqrt{1-e_2^2}\right) \\
  \Gamma_2 &= \frac{m_2}{m_{\rm tot}}\sqrt{\frac{a_2}{a_0}}
             \left(1-\sqrt{1-e_2^2}\right)
\end{align}

\noindent
Restoring \(\mathcal{H}\) with these momenta, we have
\begin{align}
\label{eq:H_1}
  \mathcal{H}
  = -\frac{q^3}{2(1+q)^3 \Lambda_1^2}
    - \frac{1}{2(1+q)^3\Lambda_2^2}
   - \frac{\tilde\mu}{(1+q)^2 \Lambda_2^2}\left[
    f_1\sqrt{\frac{2\Gamma_1}{\Lambda_1}}\cos\theta_1
    +f_2\sqrt{\frac{2\Gamma_2}{\Lambda_2}}\cos\theta_2
    \right],
\end{align}

\noindent where we have defined \(\tilde\mu=\tilde m/M\) to be
the reduced mass ratio.  We note that, as equation \eqref{eq:H_1} is
written, the \(\theta_i\) are not conjugate to \(\Lambda_i\) or
\(\Gamma_i\).  For the limiting cases of \(q\to \infty\) (\(m_2=0\)) or
\(q\to 0\) (\(m_1=0\)), \(\mathcal{H}\) reduces to the standard test
particle Hamiltonian found in \citet{murray_solar_2000}.
\subsection{Reducing rotation}
\label{sec:org1d9ec58}
Now, we would like to find the momenta conjugate to the fast
coordinates \(\lambda_i\) while keeping the slowly varying \(\theta_i\).
A canonical transformation preserves the form
\begin{align}
  \label{eq:dH} 
  d\mathcal{H}
  &= \Lambda_1 d\lambda_1+\Lambda_2d\lambda_2
    + \Gamma_1d\gamma_1+\Gamma_2d\gamma_2\nonumber\\
  &= \Gamma_1 d\theta_1 + \Gamma_2 d\theta_2
    +J_1 d\lambda_1+J_2d\lambda_2 .
\end{align}

\noindent
We can solve the set of equations in \eqref{eq:dH} for
\begin{align}
\label{eq:J1}
J_1 &= \Lambda_1 + j(\Gamma_1+\Gamma_2)\\
\label{eq:J2}
J_2 &= \Lambda_2 - (j+1)(\Gamma_1+\Gamma_2),
\end{align}

\noindent where \(\Gamma_i\) and \(J_i\) are conjugate to
\(\theta_i\) and \(\lambda_i\), respectively.
The coordinates \(\lambda_1\) and \(\lambda_2\)
no longer appear in the Hamiltonian,
which means \(J_1\) and \(J_2\) are constants of motion and
equation \eqref{eq:H_1} may be written
in the following form:
\begin{align}
\label{eq:H_2}
  \mathcal{H}
  = \mathcal{H}_0(\Gamma_1+\Gamma_2; J_1, J_2, q)
                  + \mathcal{H}_{\rm pert}(\Gamma_1,\Gamma_2; J_1, J_2, q),
\end{align}

\noindent
where
\begin{align}
  \label{eq:H01}
  \mathcal{H}_0(\Gamma_1+\Gamma_2; J_1, J_2, q)
  = -\frac{q^3}{2(1+q)^3(J_1-j(\Gamma_1+\Gamma_2))^2}
  -\frac{1}{2(1+q)^3(J_2+(j+1)(\Gamma_1+\Gamma_2))^2} 
\end{align}

\noindent
and
\begin{align}
  \label{eq:Hpert1}
  \mathcal{H}_{\rm pert}(\Gamma_1,\Gamma_2; J_1, J_2, q)
  = -\frac{\tilde\mu}{(1+q)^2(J_2+(j+1)(\Gamma_1+\Gamma_2))^2}
  \left[
    f_1\sqrt{\frac{2\Gamma_1}{J_1 - j(\Gamma_1+\Gamma_2)}}\cos\theta_1
  +f_2\sqrt{\frac{2\Gamma_2}{J_2 + (j+1)(\Gamma_1+\Gamma_2)}}\cos\theta_2
    \right].
\end{align}

\noindent We have \(\Gamma_i \ll \Lambda_i\) for small
eccentricities.  Under this assumption, we may drop terms smaller than
\(\mathcal{O}(\Gamma_i^2/\Lambda_i^4)\).  Equation \eqref{eq:H01} becomes
\begin{align}
  \label{eq:H02}
  \mathcal{H}_0
  = -\frac{1}{(1+q)^3}\left[
     \frac{q^3}{2\Lambda_1^2} + \frac{1}{2\Lambda_2^2}
   + 2\left(
     \frac{jq^3}{\Lambda_1^3} - \frac{(j+1)}{\Lambda_2^3}
     \right)(\Gamma_1+\Gamma_2)
   -\frac32\left( 
     \frac{jq^3}{\Lambda_1^4} - \frac{(j+1)}{\Lambda_2^4}\right)
     (\Gamma_1+\Gamma_2)^2
     \right]
\end{align}

\noindent
and equation \eqref{eq:Hpert1} reduces to its original form
\begin{align}
\label{eq:H_3}
  \mathcal{H}_{\rm pert}
  =-\frac{\tilde\mu}{(1+q)^2\Lambda_2^2}
  \left[
  f_1\sqrt{\frac{2\Gamma_1}{\Lambda_1}}\cos\theta_1
  +f_2\sqrt{\frac{2\Gamma_2}{\Lambda_2}}\cos\theta_2
  \right].
\end{align}

\noindent Absent any dissipation, \(\Lambda_1\) and
\(\Lambda_2\) are approximately constant in resonance.  Hence, we may
drop the first two terms in parentheses in equation \eqref{eq:H02},
leaving only the terms which include factors of \((\Gamma_1+\Gamma_2)\):
\begin{align}
  \label{eq:H03}
  \mathcal{H}_0
  = -\frac{1}{(1+q)^3}\left[
   2\left(
     \frac{jq^3}{\Lambda_1^3} - \frac{(j+1)}{\Lambda_2^3}
     \right)(\Gamma_1+\Gamma_2)
   -\frac32\left( 
     \frac{jq^3}{\Lambda_1^4} - \frac{(j+1)}{\Lambda_2^4}\right)
     (\Gamma_1+\Gamma_2)^2
     \right].
\end{align}

Following \cite{henrard86_reduc_trans_apocen_librat}
\citep[or equivalently][]{wisdom_canonical_1986}, let \(\v X\) be the
cartesian formulation
\begin{align}
  \v X &= (x_1, x_2, X_1, X_2)\nonumber\\
  &= (\sqrt{\Gamma_1}\cos\theta_1, \sqrt{\Gamma_2}\cos\theta_2,
    \sqrt{\Gamma_1}\sin\theta_1, \sqrt{\Gamma_2}\sin\theta_2)
\end{align}

\noindent 
Define
\begin{align}
    g_1 &= f_1\sqrt{\frac{2}{\Lambda_1}} \\
    g_2 &= f_2\sqrt{\frac{2}{\Lambda_2}} \\
\end{align}

\noindent and
\begin{align}
  \mathcal{A} = \frac{1}{g_1\sqrt{g_1^2+g_2^2}}.
\end{align}

\noindent The perturbation Hamiltonian \(\mathcal H_{\rm
pert}\) becomes
\begin{align}
  \mathcal H_{\rm pert} \propto g_1 x_1 + g_2 x_2
\end{align}

\noindent
Let \(\v \Psi\) be the
counter-clockwise rotation by angle \(\psi\) defined by \(\tan\psi=
g_2/g_1\):
\begin{align}
  \v \Psi =  \mathcal{A}
  \begin{pmatrix}
    g_1 & g_2 \\
    -g_2 & g_1 
  \end{pmatrix}.
\end{align}

\noindent The block matrix
\begin{align}
  \v M =
  \begin{pmatrix}
    \v \Psi & \v 0 \\
    \v 0 & \v \Psi
  \end{pmatrix}
\end{align}

\noindent is symplectic \citep{goldstein_classical_2000}.
The coefficients \(g_i\) depend weakly on the semimajor axis ratio
\(\alpha\), and so \(\v M\) only represents a canonical transformation if
\(\alpha\) is stationary or varying slowly, which is a good
approximation for the systems considered in this paper.

Define the coordinates
\begin{align}
   \v W = (w_1, w_2, W_1, W_2) \equiv \v M \v X.
\end{align}

\noindent so that \(w_1 = (g_1 x_1 + g_2 x_2)\).  Hence,
\(\mathcal H_{\rm pert}\propto w_1\) only.  Finally, we revert the \(\v
W\) set back to polar coordinates
\((\hat\theta_1,\hat\theta_2,S_1,S_2)\). The new resonance angle is
given by the equation
\begin{align}
\label{hattheta}
  \tan\hat{\theta}_1 = \frac{W_1}{w_1} = \frac{f_1 e_1\sin(\theta_1)
  + f_2e_2\sin(\theta_2)}{f_1e_1\cos(\theta_1) + f_2e_2\cos(\theta_2)}.
\end{align}

\noindent
and is conjugate to the momentum
\begin{align}
  S_1 = w_1^2 + W_1^2 = f_1^2e_1^2
  + 2f_1f_2\cos(\varpi_1 - \varpi_2) + f_2^2e_2^2.
\end{align}

\noindent
It can be shown that
\begin{equation}
  \hat{\theta}_1 = (j+1)\lambda_2-j\lambda_1
  - \hat\varpi,
\end{equation}

\noindent
where
\begin{equation}
\hat\varpi = \frac{\abs{f_1} e_1\sin(\varpi_1) -
  \abs{f_2}e_2\sin(\varpi_2)} {\abs{f_1}e_1\cos(\varpi_1) -
  \abs{f_2}e_2\cos(\varpi_2)}
\end{equation}

\noindent
We note that the other angle, \(\hat\theta_2\),
is cyclic, and so its conjugate momentum \(S_2\) is constant:
\begin{align}
  S_2 = w_2^2 + W_2^2 = q\sqrt{\alpha}f_2^2e_1^2
-2f_1f_2e_1e_2\cos(\varpi_1-\varpi_2) + \frac{f_1^2}{q\sqrt\alpha}e_2^2
,
\end{align}

\noindent
where \(\alpha=a_1/a_2\) is the ratio of semimajor axes.
The sum
\begin{align}
  \Gamma_1 +\Gamma_2 = x_1^2+x_2^2 + X_1^2 + X_2^2
  = w_1^2+w_2^2 + W_1^2 + W_2^2 = S_1 + S_2
\end{align}

\noindent
is preserved, and so the form of \(\mathcal H_0(\Gamma_1+\Gamma_2)\)
is preserved as well.

From here on out, we will denote \(\hat\theta_1\) and \(S_1\) as
\(\hat\theta\) and \(\hat S\) to indicate that they are the single pair of
dynamical variables.  We could just as well have carried out this
analysis with \(\hat\theta_2\) and \(S_2\), but \(\hat S\) has the following
geometric interpretation:
\begin{align}
  \hat S = \left\lvert \abs{f_1}\mathbf{e}_1 - \abs{f_2}\mathbf{e}_2\right\rvert^2,
\end{align}

\noindent
where the \(\mathbf{e}_i\) are the Runge-Lenz vectors with magnitude
\(e_i\) in the direction of \(\varpi_i\).  

Altogether, we arrive at the following Hamiltonian after
dropping constant terms:
\begin{align}
  \label{eq:HShat}\mathcal H(\hat \theta, \hat S) &= \mathcal H_0(\hat S) + \mathcal H_{\rm pert}(\hat \theta, \hat S) \\
  \mathcal H_0
  &= \left( 3\mathcal NS_2 -2\mathcal M\right) \hat S
    + \frac32 \mathcal N \hat S^2 \\
  \mathcal H_{\rm pert}
  &= - q\tilde\mu\mathcal K\sqrt{\hat S}\cos\hat\theta
\end{align}

\noindent This form is valid for \(q\in (0,\infty)\), and is
naturally extended to the case of an outer test particle by taking the
limit \(q\to\infty\).  The extra factor of \(q\) in front of \(\mathcal K\)
is so \(q\tilde\mu\) reduces to \(q\tilde\mu\to q\mu_2 \to \mu_1\) in the
\(m_2=0\) test particle limit.  The coefficients in this case are given
by
\begin{align}
  \mathcal M
  &= \frac{q^3}{(1+q)^3}\frac{j-(j+1)\alpha^{3/2}}{\Lambda_1^3}\\
  \mathcal N
  &= \frac{q^3}{(1+q)^3}\left(
    \frac{j}{\Lambda_1} - \frac{(j+1)\alpha^{3/2}}{\Lambda_2}
    \right)\frac{1}{\Lambda_1^3}\\
  \mathcal K
  &= \frac{q^2}{(1+q)^2}
    \frac{\alpha^{3/2}\Lambda_2}{f_1^2\sqrt{1+\alpha^{1/2}f_1^2/f_2^2}\Lambda_1^{5/2}}.\\
\end{align}

A similar representation of \(\mathcal H, \mathcal M\), \(\mathcal N\),
and \(\mathcal{K}\) also exists for which the limit \(q\to 0\) converges
to the case of an inner test particle by exchanging the necessary
factors of \(q\) for \(\Lambda_1/\Lambda_2 = q\sqrt\alpha\).


\subsection{Second fundamnetal model of resonance}
\label{sec:org6ccb5f1}
Now, all that is left to do is to rescale equation \eqref{eq:HShat}
into what is referred to as
the "second fundamental model of resonance" \citep{henrard_second_1983}.



\note{
Now transform into $\hat R$
}


\section{Elliptic restricted 3 body problem}
\label{sec:orgfa4a612}

\twocolumn
\bibliography{references}
\bibliographystyle{mnras}
\end{document}